\RequirePackage{xcolor}
\documentclass[fleqn]{fcup-thesis}

\usepackage[english]{babel}
\usepackage{lipsum}

\usepackage[utf8]{inputenc}
\usepackage[T1]{fontenc}
\usepackage{lmodern}
\usepackage{amsmath,amssymb,bm}
\usepackage{mathtools}
\usepackage{graphicx}
\usepackage[separate-uncertainty=true]{siunitx}
\usepackage[version=3]{mhchem}
\usepackage{wasysym}
\usepackage[final]{microtype}
\usepackage[hang,multiple]{footmisc}
\let\newfloat\relax
\usepackage{float}
\usepackage{placeins}
\usepackage{xkvltxp}
\usepackage{fixme}
\usepackage{color}
\usepackage{rotate}
\usepackage{rotating}
\usepackage{threeparttable}
\usepackage{threeparttablex}            % Lets threeparttable work with longtable
\usepackage{longtable}
\usepackage{pdflscape}

\usepackage{caption}
\captionsetup{
  labelfont=bf,
}
\captionsetup[table]{%
  justification=justified,
  singlelinecheck=false,
}


\definecolor{flatblue}{RGB}{41, 128, 185}
\definecolor{flatgrey}{RGB}{52, 73, 94}
\definecolor{flatgreen}{RGB}{44, 201, 144}
\definecolor{flatpurple}{RGB}{142, 68, 173}
\definecolor{flatorange}{RGB}{211, 84, 0}
\definecolor{flatred}{RGB}{207, 0, 15}
\usepackage{soul}  % highlight text with \hl{}, strikethrough with \st{}
\sethlcolor{flatgreen}
\setstcolor{flatred}
\usepackage{epigraph} %for quotes
\setlength\epigraphrule{0pt}

%% TODO NOTES
\usepackage{xargs}                      % Use more than one optional parameter in a new commands
\usepackage[colorinlistoftodos,prependcaption,textsize=tiny]{todonotes}
\newcommandx{\change}[2][1=]{\todo[linecolor=flatblue,backgroundcolor=flatblue!25,bordercolor=flatblue,#1]{#2}}
\newcommandx{\reference}[2][1=]{\todo[linecolor=flatgreen,backgroundcolor=flatgreen!25,bordercolor=flatgreen,#1]{#2}}
\newcommandx{\unfinished}[2][1=]{\todo[linecolor=flatorange,backgroundcolor=flatorange!25,bordercolor=flatorange,#1]{#2}}
\newcommandx{\improvement}[2][1=]{\todo[linecolor=flatpurple,backgroundcolor=flatpurple!25,bordercolor=flatpurple,#1]{#2}}


%%%%%%%%%%% Literature %%%%%%%%%%%%%%%
% When in doubt visit this page:
% http://adsabs.harvard.edu/abs_doc/aas_macros.html
\usepackage[sort,longnamesfirst]{natbib}
\def\aap{A\&A}
\def\aapr{Astronomy and Astrophysics Reviews}
\def\eprint{e-prints}
\def\apj{ApJ}
\def\apjs{ApJS}
\def\apjl{ApJL}
\def\mnras{MNRAS}
\def\aj{AJ}
\def\nat{Nature}
\def\aaps{A\&A Supp.}
\def\pasp{Publications of the ASP}
\def\prd{Phys. Rev. D}
\def\prl{Phys. Rev. Lett.}
\def\araa{ARA\&A}
\def\actaa{Acta Astronomica}
\def\procspie{Proceedings of the SPIE}
\def\pasj{PASJ}
\def\icarus{Icarus}


\usepackage{fix-cm} % Allows increasing the f
\usepackage[colorlinks,
           linkcolor=flatorange,
           citecolor=flatblue,
           urlcolor=flatpurple]{hyperref}

\usepackage[Lenny]{fncychap}

%%%%%%%%%%%%% new commands %%%%%%%%%%%%%%%
\newcommand{\cref}[1]{Chapter~\ref{#1}}
\newcommand{\sref}[1]{Section~\ref{#1}}
\newcommand{\tref}[1]{\tablename~\ref{#1}}
\newcommand{\fref}[1]{\figurename~\ref{#1}}
\newcommand{\aref}[1]{Appendix~\ref{#1}}
\newcommand{\eref}[1]{Equation~\ref{#1}}
\renewcommand{\epsilon}{\varepsilon}
\renewcommand{\bf}{\textbf}
\newcommand{\nicebreak}{\newline\newline\noindent}
\newcommand{\code}[1]{\texttt{#1}}
\newcommand{\object}[1]{#1}
\newcommand{\Mjup}{M_\mathrm{Jup}}

%%%%%%%%%%%%%%%% Math %%%%%%%%%%%%%%%%%%%%
\newcommand{\F}{\mathcal{F}}
\newcommand{\tm}[1]{\textnormal{#1}}
\newcommand{\pd}[2]{\frac{\partial #1}{\partial #2}}

%%%%%%%%%%%%% Layout %%%%%%%%%%%%%%%%%%%%%
\setcounter{secnumdepth}{3}
\setcounter{tocdepth}{3}

%% A&A stuff
\DeclareRobustCommand{\ion}[2]{\textup{#1\,\textsc{\lowercase{#2}}}}
\newcommand*\element[1][]{%
  \def\aa@element@tr{#1}%
  \aa@element
}


%% *** Change this example to appropriate values. ***
\degree{Doctor of Philosophy}
\department{Departamento de Fisica e Astronomia}
\gradyear{2017}
\author{Daniel Thaagaard Andreasen}
\title{Determination of stellar parameters for M-dwarf stars: the NIR approach}


%% Make each page fill up the entire page.
\flushbottom


%%%%%%%%%%%%      MAIN  DOCUMENT      %%%%%%%%%%%%
\includeonly{introduction}

\begin{document}

%% This sets the page style and numbering for preliminary sections.
\begin{preliminary}

%% This generates the title page from the information given above.
\maketitle
\cleardoublepage


\begin{dedication}
\centering \huge \itshape
To Linnea, Henriette, Rico, and Else\\For always supporting me
\end{dedication}


\begin{acknowledgements}

When doing a PhD it is important to remember it is more a team effort than the work of an
individual. This is something I learned quickly during the last four years. Therefore there are
several people I would like to thank.

First and most importantly are my two supervisors, Sérgio and Nuno. They were after me in the
beginning of my studies because I was too shy to ask for help; something that I quickly learned I
needed to do. They always had their door open for me and all my small questions. It goes without
saying that I am thankful for all their guidance during my studies. However, what I am most thankful
for is the freedom I have had to explorer paths and ideas on my own, and with them safely on the
sideline. This sometimes led to failures and dead ends, but it make me grow as a researcher both by
learning from my mistake, but also by prioritising my time.

When I thank Sérgio and Nuno, my official supervisors, I also have to thank Elisa. She has been my
third unofficial supervisor almost from the first day. Although she did not have any experience with
NIR spectroscopy, she was never afraid of giving her opinion and trying to help. Elisa, your
kindness, calmness, and help is priceless! It is extremely reassuring to go on this adventure with
three capable supervisors ready to help at any time.

While Sérgio, Nuno, and Elisa are all very capable, they were not alone to answer all my technical
questions. Often I went to Vardan to discuss science. Vardan's knowledge is immense, and it has been
a pleasure to receive advise and answers when necessary. Similar has the support from Maria in my
first two years been valuable. She was my friend here when I arrived, and the one I went to in the
beginning with small questions since I was too shy to go with ``stupid'' questions to my
supervisors. Thanks for all the answers you both gave me and the fruitful collaborations.

Luckily my time here in Porto has not only been about work. I have had a lot of good times with new
and old friends here. In a random order I would like to thank Elisa, Maria, João, Jorge, Jason,
Vardan, Mahmoud, Guilherme, Giancarlo, Andressa, Luisa, Solène, and the entire football team.

A very special thank you to Sofia who made my life easier here in Portugal, and has been a fantastic
support and friend. All your time and effort will never be forgotten, and I appreciate all you have
done for me.

Now, I know being abroad has been difficult but exciting for me. However, it has mostly been
difficult for my family home in Denmark. The last part is specifically dedicated to them: Tusind tak
for al jeres tålmodighed, mine kære venner og familie i Danmark. Jeg ved jeg ikke altid har været
den bedste til at sende et par beskeder hjem, men alligevel har I altid mødt mig med kram når I har
hentet mig i lufthavnen. Det har været en svær tid for mig her, og at tænke på jer derhjemme har
virkelig holdt mit mod oppe! Jeg vil i sær gerne takke mine søstre, Henriette og Linnea for altid at
have tid til at sludre lidt. Det samme gælder Rico, som også har hentet mig et par gange i
lufthavnen. Selvfølgelig bliver jeg nødt til at sige tak til min kære mormor. Din støtte har været
uden lige. Der har også altid været mange venner derhjemme som har fundet tid til at bruge et par
timer med mig. Tusind tak til i sær Ellen Marie, Maria, Ditte, og Sabrina.

\vspace{10}\noindent
Tusind tak alle sammen/thank you very much everybody!


\end{acknowledgements}



\begin{abstract}
\lipsum[2]
%% (At most 150 words for M.Sc. or 350 words for Ph.D.)
\end{abstract}

\begin{abstract-pt}
\lipsum[2]
%% (At most 150 words for M.Sc. or 350 words for Ph.D.)
\end{abstract-pt}



\tableofcontents
\addcontentsline{toc}{chapter}{\listtablename}
\listoftables
\addcontentsline{toc}{chapter}{\listfigurename}
\listoffigures

\end{preliminary}


%% *** Include chapter files here. ***
%!TEX root = thesis.tex
\chapter{Introduction}
\label{cha:introduction}

Ever since the dawn of time, the humankind have looked at the stars and wondered if we are alone in
this Universe. To answer this question, one must look toward the field of extrasolar planets
(exoplanets). This is a rapidly growing field in astronomy and science in general. Since the first
confirmed discovery of an exoplanet around a millisecond pulsar in 1992 by \citet{Wolszczan1992} and
three years later, the more interesting exoplanet 51 Peg b discovered around a solar-type star by
\citet{Mayor1995}, more than 3600 exoplanets have been discovered at the time of writing, July
2017\footnote{\url{http://exoplanet.eu/}}.

With the discoveries of exoplanets, the main focus is now mainly on finding the twin of Earth, that
is a planet that can harbour life as we know it. However, it is not enough to simply discover small
rocky exoplanets. Accurate and precise determination of the stellar parameters are crucial as the
planetary parameters (radius, mass, bulk density, etc.) are directly derived from their host's
parameters.

In this chapter there will be a general introduction to exoplanets, detection methods, and
characterisation (\sref{sec:exoplanets}). Then a throughout introduction on the exoplanet host
stars (\sref{sec:planet_host_stars}), which is the main focus on this thesis. While learning about
host stars, and stars in general, the results have wide-spread applications, where some will briefly
be discussed in the end of this chapter (\sref{sec:stars_application}) before an introduction on
what this thesis will consists of (\sref{sec:this_thesis}).



\section{Exoplanets}
\label{sec:exoplanets}



\section{Planet host stars}
\label{sec:planet_host_stars}

With the present diversity of exoplanets it becomes increasingly important to get an accurate and
precise characterisation of the planets in order to study them in samples and on an individual
level. An accurate and precise characterisation can give us an idea whether the planet is rocky,
composed of water or gaseous.



\section{Applications from knowing the stars}
\label{sec:stars_application}





\section{This thesis}
\label{sec:this_thesis}

\chapter{Theory}

To encompass all theory regarding stellar structure, evolution, and their
atmosphere is far beyond the scope of this thesis. Rather the theory needed is
presented below with highlights on the most important aspects.

\section{Stellar structure}

The structure of a non-rotating spherical stars can be described by five rather
simple differential equations \citep[see e.g.][]{kippenhahn} presented below:
\begin{enumerate}
    \item \textbf{Equation of Continuity}
        \nicebreak
        Relation between the mass, $m$, the density, $\rho$, at a symmetric
        shell at radius $r$
        \begin{align}
            \Aboxed{\pd{r}{m} &= \frac{1}{4\pi r^2\rho}.}
        \end{align}

    \item \textbf{Equation of Hydrostatic Equilibrium}
        \nicebreak
        The equation of hydrostatic equilibrium shows how a star in equilibrium
        is balanced between two forces. The inward force from gravity and the
        outward force from pressure, $P$,
        \begin{align}
            \Aboxed{\pd{P}{m} &= -\frac{Gm}{4\pi r^4}.}
        \end{align}
        When working with asteroseismology a time dependent perturbation to this
        equation is added \citep[see e.g.][for a thorough discussion]{Aerts2010}.
        However, this term is neglected here.


    \item \textbf{Equation of Energy Conservation}
        \nicebreak
        The equation of energy conservation shows how the energy is produced and
        lost throughout the star.
        \begin{align}
            \Aboxed{\pd{l}{m} = \epsilon - \epsilon_\nu + \epsilon_g,}
        \end{align}
        where $\epsilon$ is the energy production in the centre of the star,
        $\epsilon_\nu$ is the energy lost by neutrinos which is always
        positive, $\epsilon_g$ is a source function of time-dependent terms,
        and $l$ is the luminosity at $m$. $\epsilon_g$ comes from the fact that
        non-stationary shells can change its internal energy, and thus exchange
        mechanical energy with neighbouring shells.

    \item \textbf{Equation of Energy Transport}
        \nicebreak
        Energy transportation throughout the star is described with the
        following equation
        \begin{align}
            \Aboxed{\pd{T}{m} &= -\frac{GmT}{4\pi r^4P}\nabla_\tm{rad},}
        \end{align}
        where $\nabla_\tm{rad}$ is the radiative temperature gradient, and $T$
        is the temperature. The value of the temperature gradient compared to
        the radiative temperature gradient tells if the energy is transported by
        convection or radiation. In our Sun the outer layer are convective while
        the inner layer are radiative.

    \item \textbf{Equation of Chemical Composition}
        \nicebreak
        In this last equation we see the evolution of an element, $X_i$, when
        it reacts with other elements with reaction rates $r_{ji}$ and $r_{ik}$
        \begin{align}
            \Aboxed{\pd{X_i}{t} &= \frac{m_i}{\rho} \Bigl( \sum_j r_{ji} - \sum_k r_{ik}\Bigr).}
        \end{align}
        Note that this is the only time-dependent equation of the five
        presented.
\end{enumerate}

These five fundamental equations are implemented in stellar evolutionary codes,
which we will use in later chapters. The many different codes that exist take
other things into account, e.g the star can rotate, and it may not always be in
hydrostatic equilibrium (this is important if we want our star to pulsate). For
simplicity we have only presented time-dependence in the Equation of Chemical
Composition since timescales of rotation, pulsations, and activity are much
shorter than the long timescale found in chemical composition changes.


\section{Stellar atmosphere}

Much of this Section is inspired by \citet{Gray2006}. While all the figures here
were made by the author of this thesis, most of them can be found in
\citet{Gray2006} as well.

% \\\\
Stellar atmospheres are rather complicated. This is where the light produced in
the interior of the stars are released. However, the atmosphere of a star is not
transparent to all light, and some of the light is absorbed in the atmosphere
and later emitted in a random direction. The different elements in the
atmosphere is the reason for absorbing light at specific wavelength. The
strength of the absorption depends on the physical conditions in the atmosphere,
the effective temperature ($T_\mathrm{eff}$), the pressure/gravity ($\log g$),
the overall metallicity ($[\ion{Fe}/\ion{H}]$), the specific abundance of a
given element if different from the overall metallicity ($A$), and the atomic
characteristics of the transition coursing the absorption line.

It is important to know the fraction of atoms excited to the $n$th energy level,
$N_n$. This fraction is proportional to the statistical weight $g_n$ and the
Boltzmann factor and is described as:
\begin{align}
    \Aboxed{ \frac{N_n}{N} = \frac{g_n}{u(T)} 10^{-\theta\chi_n} \tag*{Boltzmann}}
\end{align}
This equation is also called the Boltzmann equation. Here $N$ is the total
number of atoms per unit volume, $u(T)=\Sigma g_i e^{-\chi_i/kT}$ is the
partition function, $k$ is Boltzmann's constant, $T$ is the temperature, and
$\chi_n$ is the excitation potential for the lower energy level.

While atoms can get excited following Boltzmann's equation above, they can also
get ionised. The ionisation for a collision dominated gas (which is a good
approximation for FGKM stars), the ratio of neutral atoms to single ionised
atoms is described by the Saha equation:
\begin{align}
  \Aboxed{ \frac{N_1}{N_0}P_e = \frac{(2\pi m_e)^{3/2}(kT)^{5/2}}{h^3} \frac{2u_1(T)}{u_0(T)} e^{-I/kT}  \tag*{Saha} \label{eq:saha}}
\end{align}
here $m_e$ is the electron mass, $h$ is Planck's constant, and $I$ is the
ionisation potential of the ion.

The atomic lines are characterised by few quantum mechanical descriptors.
\begin{itemize}
  \item The wavelength describes between which energy levels there is a
        transition, i.e. at which wavelength the light is absorbed.
  \item The ionisation state, i.e. is it a atom element absorbing or a ionised
        atom.
  \item The excitation potential. This gives an idea how deep in the atmosphere
        a line is formed. If $\chi$ is high, then higher temperatures (i.e.
        higher random motion and more collisions between the atoms) is required
        for forming the absorption line. These higher temperatures are found
        deeper in the atmosphere.
  \item The oscillator strength, $\log \mathit{gf}$, is related to the atomic
        transition probability.
  \item The damping coefficients, is a natural damping (also known as radiation
        damping) caused by the uncertainty of lifetime in an energy level
        according to Heisenberg's uncertainty principle. This is related to a
        uncertainty in the energy level and thus a natural broadening is
        introduced.
\end{itemize}




\subsection{The equivalent width}

Measuring the equivalent width (EW) of spectral lines are important for some
analysis of stellar spectra. The EW is a measure of the strength of the line,
and dependent on the atmospheric conditions in where the spectral line is
formed, such as $T_\mathrm{eff}$, $\log g$, $[\ion{Fe}/\ion{H}]$, and
$\xi_\mathrm{micro}$.

\begin{figure}[htpb!]
    \centering
    \includegraphics[width=1.0\linewidth]{figures/ewTheoretical.pdf}
    \caption{An absorption line centred at $\lambda_0$ normalised at the flux
             level $F_c$. The area of the absorption line to the left is equal
             to the blue shaded area in the rectangle to the right with width
             EW.}
    \label{fig:ewTheoretical}
\end{figure}

The EW is mathematically described as integrating over the entire line, and
assign this area to a rectangle from 0 to the continuum flux ($F_c$) with the
width, EW. This is illustrated in \fref{fig:ewTheoretical} and the equation
below: \begin{align} EW = \int_{0}^{\infty} \frac{F_c-F(\lambda)}{F_c} d\lambda,
\end{align} where $\lambda$ is the wavelength. This is integral is assuming
there is only one single line, hence the integral is over all wavelength. In
practice the integral is calculated in small windows around a spectral line. See
\sref{sec:measureEW} for more details on how this is performed in practice. The
unit of the EW is the same as the wavelength used. Throughout this thesis we
will use \AA{}ngstr\"{o}m (1\AA$=\SI{0.1}{nm}$) for the wavelength, and m\AA{}
for the EW.


\subsubsection{Temperature dependence}

As mentioned above the EW depends on the atmospheric parameters. The dependence
on $T_\mathrm{eff}$ is the strongest dependence. At low $T_\mathrm{eff}$ neutral
elements, say \ion{Fe}{I}, are the strongest lines as the number of ionised
atoms are too small to contribute significantly to the EW. This is the result
the Saha equation. As $T_\mathrm{eff}$ increases \ion{Fe}{I} is converted into
ionised \ion{Fe}{II}. Slowly, as the number of \ion{Fe}{I} decreases so does the
EW, and likewise as the number of \ion{Fe}{II} increases so does the EW. This
goes on until second ionised atoms, \ion{Fe}{III}, are formed and the same
situation arise again. This is illustrated in \fref{fig:ewTeff} where the EW of
two iron lines, one neutral and one ionised, are plotted against
$T_\mathrm{eff}$. These two lines have similar EW in the Sun:
$\SI{46.2}{m}$\AA{} and $\SI{53.9}{m}$\AA{} for the \ion{Fe}{I} and \ion{Fe}{II}
line respectively.

\begin{figure}[htpb!]
    \centering
    \includegraphics[width=1.0\linewidth]{figures/ewTeff.pdf}
    \caption{The EW for a \ion{Fe}{I} and \ion{Fe}{II} line with increasing
             $T_\mathrm{eff}$. The two lines have similar EW in the Sun and are
             found in the optical part of the spectrum.}
    \label{fig:ewTeff}
\end{figure}


\subsubsection{Pressure dependence}

Pressure dependence in the stellar atmosphere can be related to the gravity
dependence. There are many ways to measure the pressure, and thus the gravity
which is what is ultimately the goal with the measurement of $\log g$. Below are
listed some of the most common methods to measure $\log g$ from spectroscopy.

\begin{itemize}
  \item Continuum: The Balmer jump is the only continuum feature sensitive
        enough to estimate the $\log g$.
  \item Hydrogen lines: Hydrogen profiles are pressure sensitive and can
        therefore be used to estimate $\log g$. However, the gravity dependence
        rapidly diminishes for temperatures above $\SI{10\,000}{K}$.
  \item Other strong lines: There exists other strong lines with
        pressure-broadened wings such as the \ion{Ca}{II} H and K lines. These
        are better for cooler stars than the hydrogen lines described above.
  \item Weak lines: By comparing two stages of ionisation for the same element
        it is possible to obtain $\log g$ using weaker or modestly strong lines.
\end{itemize}
In this thesis weak lines are used to measure $\log g$. More specifically a
comparison between \ion{Fe}{I} and \ion{Fe}{II} lines are used. For FGK stars,
as the atmosphere contracts (i.e. $\log g$ increases) the pressure likewise
increases, which in turn means that both the gas pressure, $P_g$, and electron
pressure, $P_e$, increases. Since hydrogen is the main electron contributor, but
not fully ionised for these stars, the electron pressure is much smaller that
the gas pressure. The gas pressure follow a simply empirical approximation with
gravity:
\begin{align}
  P_g \simeq \mathrm{constant}\, g^{2/3},
\end{align}
where $g$ is the gravity.

NEED TO WRITE MORE HERE...

In \fref{fig:ewGravity} is shown how the \ion{Fe}{II} line used previously
change with $\log g$. The curve of growth is shown in the upper panel, while a
synthetic spectrum for each $\log g$ is presented in the lower left panel. It is
clear that the ionised line is sensitive to $\log g$ as shown in the lower right
panel, where the correlation between the abundance and $\log g$ is 0.40. This is
expected as can be seen in \citet[][Table 16.1]{Gray2006}. There is used
$\delta\log A/\delta\log g$ as an indicator, and for neutral elements the
correlation is much weaker. It is important to note that the correlation does
change with $T_\mathrm{eff}$ and the element used.

\begin{figure}[htpb!]
    \centering
    \includegraphics[width=1.0\linewidth]{figures/ewGravity.pdf}
    \caption{\emph{Upper panel}: Curve of growth for same \ion{Fe}{II} used in
             \fref{fig:ewTeff} for four different $\log g$ values. Here it is
             the weak lines mostly affected by the change in $\log g$.
             \emph{Lower left panel}: Synthetic spectra of the same line. The
             colour scale is the same.
             \emph{Lower right}: The abundance for the line at different
             $\log g$. A strong correlation (0.40) is seen.}
    \label{fig:ewGravity}
\end{figure}




\subsubsection{Abundance dependence}

The abundance of a given element obviously has an effect on the EW. The more
abundant an element is, the more photons can be absorbed thus increasing the EW.
However, the relationship is not strictly linear. For weak lines (GIVE RANGE) EW
is approximately linear with the abundance, however it reach a plateau where the
core of the line saturates. In this regime the EW only increases slowly, until
the absorption "spills" into the wings and the increase is again linear.
However, for these strong lines the profile is no longer Gaussian. The curve of
growth for the same \ion{Fe}{I} line used in \fref{fig:ewTeff} is shown in
\fref{fig:cog}. Instead of EW it is common to use the reduced EW, $\log
(EW/\lambda)$\footnote{The reduced EW is useful since it normalises
Doppler-dependent phenomena, such as microturbulence and thermal broadening.},
which we will use more later. Instead of the abundance of a line, the oscillator
strength, $\log \mathit{gf}$, is used.

\begin{figure}[htpb!]
    \centering
    \includegraphics[width=1.0\linewidth]{figures/cog.pdf}
    \caption{\emph{Upper panel:} Curve of growth of the same \ion{Fe}{I} line as
             used in \fref{fig:ewTeff}. Four points are marked which is shown in
             the \emph{lower panel} as a synthetic spectral line. The RW (proxy
             for EW) is clearly increasing with $\log \mathit{gf}$ (proxy for
             abundance).}
    \label{fig:cog}
\end{figure}


\subsubsection{Microturbulence}

Small-scale motion, that is motion of material at length scales small compared
to the unit optical depth, are called microturbulence, $\xi_\mathrm{micro}$.
This is not to be confused with macroturbulence, which is motion of material at
scales larger than the unit optical depth. The latter is associated with
granulation and will not be discussed further in this thesis.
$\xi_\mathrm{micro}$ comes into play when looking at the curve of growth for
saturated lines (i.e. between green and red points in \fref{fig:cog}). If no
$\xi_\mathrm{micro}$ is assumed, then the measured abundance is higher than
predicted by models based on thermal and damping broadening alone. In
\fref{fig:cog_vt} is shown three curves of growth with
$\xi_\mathrm{micro}={\SI{0.5}{km/s}, \SI{2.5}{km/s}, \SI{5.0}{km/s}}$. As
$\xi_\mathrm{micro}$ increases, so does the EW and hence the abundance.

\begin{figure}[htpb!]
    \centering
    \includegraphics[width=1.0\linewidth]{figures/cog_vt.pdf}
    \caption{Curve of growth for three different values of $\xi_\mathrm{micro}$.
             The EW is increasing with increasing $\xi_\mathrm{micro}$.}
    \label{fig:cog_vt}
\end{figure}

The broadening of an absorption line measured by the shift in wavelength,
$\Delta\lambda$, when $\xi_\mathrm{micro}$ is included is defined as:
\begin{align}
  \Delta\lambda = \frac{\lambda_0}{c} \sqrt{\frac{2kT}{m} + \xi_\mathrm{micro}^2},
\end{align}
where $c$ is the speed of light, $\lambda_0$ is the rest wavelength of the given
line, $k$ is Boltzmann's constant, $T$ is the temperature, and $m$ is the mass
of the atom. Setting $\xi_\mathrm{micro}=\SI{0}{km/s}$, we end up with thermal
broadening.


\section{Stellar parameters for FGK stars - the EW method}

\subsection{Line list and atomic data}

\subsection{Measuring EW}
\label{sec:measureEW}

\subsection{Determining abundances with MOOG}

%!TEX root = thesis.tex

\chapter{Deriving stellar parameters}
\label{cha:method}


\section{Different methods for atmospheric parameters}

\subsection{IRFM}

\subsection{Photometry}

\subsection{Spectroscopy}


\section{Other stellar parameters}

\subsection{Asteroseismology}


\section{\FASMA}
\label{sec:parameters}

In this section, the process from a spectrum to atmospheric parameters will be
explained in details. There are two classic methods, synthetic fitting and
curve-of-growth analysis.

The synthetic fitting method is in simple terms a comparison between the
observed spectrum and a synthetic spectrum, which is either calculated on the
fly like SME \citep{Valenti1996}, or using a pre-calculated grid. By analysing
the $\chi^2$ the synthetic spectrum that best match the observed spectrum can be
found. This technique works for all ranges of spectral resolutions and can work
for many rotational profiles as well \citep[see e.g.][]{Tsantaki2017}. However,
this method is often time-consuming compared to the curve-of-growth analysis.

Here the curve-of-growth analysis will be explained in detail. In particular
Fast Analysis of Spectra Made Automatically (\FASMA) which was developed during
this thesis. \FASMA is made of three \emph{drivers}, 1) EW measurement driver,
2) stellar atmospheric parameters driver, and 3) abundance driver. An additional
driver is under development; a synthetic fitting driver \citep{Tsantaki2017}.
\FASMA has been made available to the community via a web
application\footnote{\url{http://www.iastro.pt/fasma/}} \citep{Andreasen2017a}.


\subsection{Ingredients}

\FASMA is written in the Python programming language and glue together other
software and models necessary for obtaining stellar atmospheric parameters from
high quality spectra. These software and models are described in greater detail
in the following sections. In short, the curve-of-growth analysis require
measured EWs where the latest version of \ARES is used \citep{Sousa2015a}. These
EWs are used to derive line abundances using model atmosphere like the ATLAS9
\citep{Kurucz1993}, MARCS models \citep{Gustafson2008}, and PHOENIX models
\citep{Husser2013} to mention the most popular for this analysis. Note that the
PHOENIX models are relative new and not as widely used yet. In tandem with model
atmospheres a radiative transfer code is also needed. \FASMA uses \MOOG
\citep{Sneden1973} for this, however there are other codes available, e.g.
\unfinished{Mention some other codes here}. The model atmosphere usually comes
in a pre-calculated grid in the $\{T_\mathrm{eff},\,\log
g,\,[\ion{Fe}/\ion{H}]\}$ parameter space. These are interpolated in order to
access the requested combination of parameters. Last, \FASMA consist of a
minimization routine which looks for the right parameters given a spectrum.



\subsection{Wrapper for \ARES}
\label{sec:measureEW}

There are two ways to measure the EW of an absorption line, manual or automatic.
Both of these methods are used here. There are advantages and disadvantages for
both method. For the manual, an advantage is that we can inspect the lines and
try to measure lines in different ways (which is useful if it is blended). We
have more control over how blended lines are fitted, and which profiles are
used. Disadvantages are that it is very time consuming, and it is prone to
errors, as a measurement might change drastically by the eyes measuring it. Even
for the same person, the measurement can change. By mentioning the advantages
and disadvantages of the manual method, it should be clear that the advantages
and disadvantages of the automatic method is the opposite of those. Especially
the time to measure the lines are orders of magnitudes faster, which is crucial
when dealing with more than a handful of spectra.

When a line is measurement by hand (manually) it is in this thesis done using
the splot command in IRAF. Here the deblending mode is used whenever necessary.
It is often necessary to fit one spectral lines with several Gaussians, as
neighbouring lines might contaminate the line of interest slightly.

When a line is measurement automatically it is in this thesis done with \ARES
\citep{Sousa2007,Sousa2015a}. When using \ARES it is important to use a correct
value of the \code{rejt} parameter. This parameter is used for placing the
continuum level, and is thus directly related to the final measurement EW. It is
difficult to get this parameter right, however the newest version of \ARES has
the option to analyse a few absorption free regions and measure the S/N. The
\code{rejt} is then calculated as:
\begin{align*}
  \mathtt{rejt} = 1 - \frac{1}{\mathrm{S/N}}.
\end{align*}

\ARES is used via the first driver of \FASMA. All the options available for
\ARES can be accessed by \FASMA. The options are setting the spectral window,
$\lambda_\mathrm{min}$ and $\lambda_\mathrm{max}$, the RV correction to be
applied or a mask to measure the RV and automatic make this correction, the
minimum and maximum EW to be considered ($\SI{5}{m}$\AA{} and $\SI{150}{m}$\AA{}
respectively), the minimum distance between two consecutive lines, the smoothing
applied with a \emph{boxcar} filter before measuring the EWs. An in-depth
description of these options can be found in \citet{Sousa2007,Sousa2015a}.

Sometimes \ARES crash when measuring an absorption line. The reason is not
clear, however when dealing with a large amount of spectra, it is important that
the analysis moves on. To deal with this problem, \FASMA finds the last line
which \ARES tried to measure in the log file. This line is temporarily removed
from the line list and \ARES is restarted. The line list used for deriving
parameters consists of numerous iron lines, thus removing one line will have a
negligible effect on the final derived parameters.



\subsection{Interpolation of atmosphere models}
\label{sec:interpolation}

\FASMA has access to both ATLAS9 models by \citet{Kurucz1993} and MARCS models by
\citet{Gustafson2008}, both are in a pre-calculated grid as described above. Let
this grid be described by $\{T_\mathrm{eff,g},\,\log
g_g,\,[\ion{Fe}/\ion{H}]_g\}$, where $g$ is one of the grid points.
\unfinished{Make a table or plot describing the grid}. The requested value will
be $\{T_\mathrm{eff,r},\,\log g_r,\,[\ion{Fe}/\ion{H}]_r\}$. The task is now to
find the surrounding grid points in the parameter space of the requested
parameters. For $\log g$ and $[\ion{Fe}/\ion{H}]$ two neighbouring grid point
are used, and for $T_\mathrm{eff}$ four surrounding grid point are used, in
total $4\times2\times2=16$ model atmospheres for the interpolation. \FASMA use
the four surrounding grid points for $T_\mathrm{eff}$ instead of two, since the
model atmosphere changes most with $T_\mathrm{eff}$. This is common in other
interpolations\reference{Give ref here} as well.

When the 16 model atmosphere have been located, the interpolation goes through
each layer of the model atmosphere, where there typical are 72 layers, and each
column of which there are six. The columns are described in
\sref{sec:atmospheremodels}. The interpolation are done using the
\code{griddata} function from \code{SciPy}\footnote{\url{https://scipy.org/}}.
The interpolation is linear in the parameter space. After the interpolation, the
result is saved to a file in the format expected by \MOOG.






\subsection{Minimization}

With the measured EWs for all the lines in the line list, we choose an
atmosphere model to determine the abundances. If there is no prior knowledge of
the star it is common simple choose a solar atmosphere model as a starting
point. Next the correlation between the abundances and the reduced EWs, and the
abundances and the excitation potential is calculated. If there is a correlation
it means the atmosphere model used is wrong. Moreover, we also have to check if
the mean abundance of \ion{Fe}{I} and \ion{Fe}{II} lines are equal, and last if
mean abundance of the \ion{Fe}{I} lines is equal to the input
$[\ion{Fe}/\ion{H}]$ of the atmosphere model\footnote{We use \ion{Fe}{I} instead
of \ion{Fe}{II} lines for this, since they are more numerous.}. If one of these
four criteria does not pass, then the atmosphere model is wrong, and we have to
search for a new one. A common way to do this, is by combining the indicators
into a scalar value:
\begin{align}
  f(\{T_\mathrm{eff}, \log g, [Fe/H], \xi_\mathrm{micro}\}) &= \sqrt{a_\mathrm{EP}^2 + a_\mathrm{RW}^2 + \Delta\ion{Fe}{}^2},
\end{align}
where $a_\mathrm{EP}$ is the correlation between abundances and excitation
potential, $a_\mathrm{RW}$ is the correlation between abundances and reduced EW,
and $\Delta\ion{Fe}{}$ is the difference between the mean abundances of
\ion{Fe}{I} and \ion{Fe}{II}. This scalar function can be minimized using
standard minimization procedures as the simplex downhill among others. However,
there is another approach that takes into the account the information stored in
these indicators. For example, if $a_\mathrm{EP}$ is positive it means
$T_\mathrm{eff}$ has to be increased by an amount correlated by the numerical
value of $a_\mathrm{EP}$. In the same way, a non-zero $a_\mathrm{RW}$ means
$\xi_\mathrm{micro}$ has to be changed, and $\Delta\ion{Fe}{}$ is an indicator
for $\log g$. In the end it is a vector function being minimized which are more
difficult, however we are not minimizing this using standard mathematical
methods, but rather using the physical knowledge. This minimization is useless
for anything else, but it is excellent for this.
The vector function has the form:
\begin{align}
    f(\{T_\mathrm{eff}, \log g, [Fe/H], \xi_\mathrm{micro}\}) = \{a_\mathrm{EP}, a_\mathrm{RW}, \Delta\ion{Fe}, \ion{Fe}{I}\}.
\end{align}

In each iteration where convergence is not reached, the input metallicity is
changed to that of the mean output metallicity using the \ion{Fe}{I} lines. The
minimization is depicted in \fref{fig:minimization}. This minimization is
written in the Python programming language and is also a wrapper around both
\ARES and \MOOG. The entire package is called \FASMA\footnote{Greek for
spectrum} \citep{Andreasen2017a, Tsantaki2017}. \FASMA is able to fix one or all
of the four atmospheric parameters, and when it reach convergence it checks if
there are any outliers in the abundances. These will be removed, either all at
once, all iteratively, meaning that after removing the outliers the minimization
is restarted at the previous best parameters, and this process is continued
until there can be removed no other outliers, or last is removing one outlier
iteratively. An optical line list like the ones by
\citet{Sousa2008a,Tsantaki2013} have been tested thoroughly and it is safe to
remove a larger amount of lines and still obtain reliable parameters. However,
with a less tested line list, like the one by \citet{Andreasen2016} (and refined
in \citet{Andreasen2017b}), one should remove outliers more carefully, and it is
recommended that one outlier is removed iteratively.

In \fref{fig:minimization} there is a flag with \emph{autofixvt}. This was an
option introduced since we see that some spectra does not converge, however the
usual way to proceed is to fix the $\xi_\mathrm{micro}$. This is done at the end
of the minimization if the $\xi_\mathrm{micro}$ value is close to either
$0\si{km/s}$ or $5\si{km/s}$ and $|a_\mathrm{RW}| > 0.050$. When fixing
$\xi_\mathrm{micro}$ with \FASMA, its value is changed in each iteration
following a simple empirical relation:
\begin{align}
  vt = \begin{cases}
    6.935 \cdot 10^{-4}\; T_\mathrm{teff} - 0.348 \log g - 1.437     & \text{For $\log g \ge 3.95$} \\
    2.72 - 0.457 \log g + 0.072 \cdot [\ion{Fe}/\ion{H}]             & \text{For $\log g < 3.95$},
\end{cases}
\end{align}
where the first case is from \citet{Tsantaki2013} and the latter case is from
\citet{Adibekyan2015}.

Last there is an option, \emph{refine}. This apply more strict criteria for the
indicators to reach convergence, thus making the minimization less sensitive to
the initial guess since it could otherwise reach convergence from one "side" of
the parameter space. The default criteria are:
\begin{align*}
  a_\mathrm{EP}     &= 0.001\\
  a_\mathrm{RW}     &= 0.003\\
  \Delta\mathrm{Fe} &= 0.001.
\end{align*}
The criteria for $a_\mathrm{RW}$ is not as strict as $a_\mathrm{EP}$ since this
indicator can change rapidly with small changes in $\xi_\mathrm{micro}$, thus a
very strict criteria might never lead to convergence. Convergence is reached
once all of the above criteria are met, and the input and output metallicity are
identical. If one or more of the parameters are fixed, the corresponding
criterion is simply set to 0 and effectively ignored, thus not changing the
parameter.

For each iteration, the change to be applied for the atmospheric parameters are
defined by adding the following:
\begin{align}
  T_\mathrm{eff}     &: \SI{2000}{K} \cdot a_\mathrm{EP}   \\
  \xi_\mathrm{micro} &: \SI{1.5}{km/s} \cdot a_\mathrm{RW} \\
  \log g             &: -\Delta\mathrm{Fe}
\end{align}
to each parameter. Note again that metallicity is simply changed to the the
output metallicity of the previous iteration. These are empirical relations.
Note that by changing e.g. $T_\mathrm{eff}$ not only is $a_\mathrm{EP}$
affected, but the other indicators as well. So there is a inter-dependency
between the parameters, however this is ignored by \FASMA as it is not a simple
problem to solve. The stepping presented above is chosen to rapidly reach
convergence, without causing problems for the inter-dependency.

\begin{figure}[htpb!]
    \centering
    \includegraphics[width=1.0\linewidth]{figures/FASMA_minimization.pdf}
    \caption{Overview of the minimization for \FASMA. Credit: \citet{Andreasen2017a}.}
    \label{fig:minimization}
\end{figure}

\subsection{Error estimate}

%!TEX root = thesis.tex
\chapter{Results for FGK stars}
\label{cha:results}


\section{Parameter dependence on EP cut}
\label{sec:EPcut}


\section{HD20010}
\label{sec:HD20010}


\section{Arcturus}
\label{sec:arcturus}


\section{10 Leo}
\label{sec:10Leo}


\section{Synthetic cool stars}
\label{sec:synthetic_spectra}

%!TEX root = thesis.tex


\chapter{SWEET-Cat}
\label{sec:SWEET-Cat}

Part of the work during the thesis has been dedicated to regularly update
SWEET-Cat\footnote{\url{https://www.astro.up.pt/resources/sweet-cat/}}, a catalogue with all
discovered planet hosts, and the stellar parameters.

In this chapter a detailed description of SWEET-Cat will be presented. Moreover an analysis of 50
planet hosts was performed during the thesis with updated planetary parameters (mass and radius).


\section{What is SWEET-Cat?}

As mentioned above, SWEET-Cat is a catalogue of planet host stars. However, the strength of
SWEET-Cat is the homogeneously analysed stars utilising the method described in
\sref{sec:parameters} with \code{FASMA} or a similar tool before the creation of \code{FASMA}.

In the era with a large number of discovered exoplanets (more than 3500 confirmed exoplanet at the
moment of writing), the time for in-depth statistical studies has arrived. However, when conducting
these studies it is crucial to have consistent measurements of e.g. stellar atmospheric parameters.
This can be obtained by using a single analysis to obtain these parameters, as it is know that
different methods will lead to different results \citep[see e.g.][for a recent review]{Hinkel2016}.

To obtain stellar atmospheric parameters from one method is an on-going goal with SWEET-Cat, where
high quality spectra are obtained for stars hosting planets. These are used to determine the stellar
parameters in a homogeneous way. All stars in SWEET-Cat analysed with the method from our group are
marked with a flag showing whether it is analysed homogeneously or not. The columns provided in
SWEET-Cat are summarised in \tref{tab:sweetcat}. It is important to note that SWEET-Cat does not
include any planetary parameters.

\begin{table}[htb!]
    \caption{Columns in SWEET-Cat}
    \label{tab:sweetcat}
    \centering
    \begin{tabular}{lrl}
      \hline\hline
      Column                         & Unit      & Description \\
      \hline
      Name                           &           & Popular stellar name                                 \\
      HD number                      &           & HD number                                            \\
      RA                             & \si{deg}  & Right ascension                                      \\
      Dec                            & \si{deg}  & Declination                                          \\
      $\mathrm{Vmag}$                & \si{mag}  & V magnitude                                          \\
      $\sigma(\mathrm{Vmag})$        & \si{mag}  & Error on V magnitude                                 \\
      $\pi$                          & \si{mas}  & Parallax                                             \\
      $\sigma(\pi)$                  & \si{mas}  & Error on parallax                                    \\
      Source of $\pi$                &           & Source of parallax measurement                       \\
      $T_\mathrm{eff}$               & \si{K}    & Effective temperature                                \\
      $\sigma(T_\mathrm{eff})$       & \si{K}    & Error on effective temperature                       \\
      $\log g$                       & \si{dex}  & Surface gravity                                      \\
      $\sigma(\log g)$               & \si{dex}  & Error on surface gravity                             \\
      $\log g_{\mathrm{LC}}$         & \si{dex}  & Surface gravity corrected from light curves          \\
      $\sigma(\log g_{\mathrm{LC}})$ & \si{dex}  & Error on surface gravity corrected from light curves \\
      $\xi_\mathrm{micro}$           & \si{km/s} & Micro turbulence                                     \\
      $\sigma(\xi_\mathrm{micro})$   & \si{km/s} & Error on micro turbulence                            \\
      $[\ion{Fe}/\ion{H}]$           & \si{dex}  & Metallicity                                          \\
      $\sigma([\ion{Fe}/\ion{H}])$   & \si{dex}  & Error on metallicity                                 \\
      Mass                           & $M_\odot$ & Stellar mass                                         \\
      $\sigma(\mathrm{Mass})$        & $M_\odot$ & Error on stellar mass                                \\
      Reference                      &           & Reference for parameters                             \\
      Homogeneity flag               & 0/1       & 0 for not homogeneous analysis, 1 otherwise          \\
      Last updated                   & date      & Last updated                                         \\
      Comments                       &           & Any special remarks/comments (e.g. M star)           \\
      \hline
    \end{tabular}
\end{table}

SWEET-Cat is updated on a weekly basis if new planet hosts are discovered, and whenever planet hosts
have been analysed with the method from our group, as described in this thesis.


\section{Data for 50 planet hosts}

In this section the data for a large update to SWEET-Cat will be described. The majority of the data
comes as a result from proposals submitted for observational time, while some of the data was found
in the archive. In the next section the analysis of the 50 spectra will be presented along with the
results.


\subsection{Proposals for observation time}



\subsection{Data collected from proposals}



\subsection{Data collected from archive}



\section{Analysis}


The method of determining atmospheric parameters from the curve of growth analysis has been applied
several times in the optical \citep[see e.g.][]{Mortier2013b,Tsantaki2013,Sousa2011,Santos2013}.
When studying stars with planets and any correlations between stellar and planetary parameters it is
important to have a homogeneous characterisation of the stars. An effort to create such a sample was
initiated by \citet{Santos2013} with the SWEET-Cat database. The motivation to homogenise the
stellar hosts is mainly to compare the hosts and make statistical studies on one consistent scale.
When doing these statistical studies, the results might otherwise suffer from offsets between
different methods.

The skills acquired during the NIR studies as mentioned above were directly translated into deriving
parameters for a sample of 50 known planet host stars that were not previously analysed by our group
\citep{Andreasen2017a}. The spectra of these stars were required at UVES, FIES, HARPS, and ESPaDOnS
with the mean S/N higher than 200\unfinished{Write more about the data acquisition here}.

A Hertzsprung-Russell diagram of the sample can be seen in \fref{fig:sweetcat}. The sample covers a
large range of $T_\mathrm{eff}$, FGK, while there are both dwarf, sub-giant, and some giant stars.
The colours indicate the $\log g$. In order to determine the luminosity of each star the simple
relation
\begin{align*}
  L = 4\pi R^2 \sigma T^4_\mathrm{eff}
\end{align*}
is used, where $L$ is the luminosity, $R$ is the stellar radius, and $\sigma$ is the
Stefan-Boltzmann constant. In solar units this relation is simply:
\begin{align*}
  \frac{L}{L_\odot} = \left(\frac{R}{R_\odot}\right)^2 \left(\frac{T_\mathrm{eff}}{T_{\mathrm{eff},\odot}}\right)^4
\end{align*}
In order to determine the the stellar radius, the empirical relation from \citet{Torres2010} was
used.

\begin{figure}[htpb!]
    \centering
    \includegraphics[width=1.0\linewidth]{figures/HR.pdf}
    \caption{A Hertzsprung-Russell diagram of the sample of 50 planet host stars added to SWEET-Cat.
             The parameters were derived using optical high resolution and high S/N spectra in
             tandem with \code{FASMA} and an optical line list. The colour scale shows the derived
             $\log g$ for each star.}
    \label{fig:sweetcat}
\end{figure}

The parameters were derived using \code{FASMA} with the optical line list compiled by
\citet{Sousa2008a} and \citet{Tsantaki2013} for stars where $T_\mathrm{eff}$ was below \SI{5200}{K}.
All the new derived parameters were added to SWEET-Cat, available for the community.

With these updated parameters the completeness of SWEET-Cat for stars brighter than V magnitude 10
is 85\% (77\% for stars brighter than 12). For fainter stars it is time expensive to acquire spectra
of the quality needed for this method. Moreover, many of the fainter planet host stars have been
observed with the \emph{Kepler} space mission, where most stars are faint.

SWEET-Cat was recently combined with planetary masses to see two distinctive populations for giant
planets by \citet{Santos2017}. This can be seen in the mass histogram in \fref{fig:giantpopulations}
for the full sample of giant planets, with masses higher than 1 Jupiter mass and lower than 20
Jupiter masses, and for a sample constrained by: $\SI{4000}{K}\leq T_\mathrm{eff} \leq\SI{6500}{K}$
in order to have reliably atmospheric parameters from spectroscopic data, orbital periods above
\SI{10}{days} to avoid hot jupiters whose formation and migration process is debated \citep[see
e.g.]{Ngo2016}, orbital periods below 5 years to allow for the sample to be reasonable complete.
Last only stars brighter than 13 magnitude were included to ensure that the planetary masses can
have been derived with reasonable confidence using the radial velocities.

\begin{figure}[htpb!]
    \centering
    \includegraphics[width=1.0\linewidth]{figures/giantPopulation.pdf}
    \caption{Giant planet masses for the full sample and constrained sample (see text for details).
             This study was performed by \citet{Santos2017} to distinct two giant planet populations.}
    \label{fig:giantpopulations}
\end{figure}

By separating the distribution into two at $4M_{Jup}$, it can be shown \citep[see][for
details]{Santos2017} that the stars hosting the more massive giant planets are in average more
metal-poor compared to the stars hosting the lower mass giant planets. This suggest two different
stellar populations forming giant planets.


\section{Future work}

%!TEX root = thesis.tex
\chapter{Conclusions and future work}
\label{cha:future}

\section{Conclusions}

This thesis consisted of two separated analysis. 1) The analysis of NIR stellar spectra and the
compilation of a new iron line list for determining stellar atmospheric parameters, and 2) the
analysis of 50 planet-hosting stars and update of SWEET-Cat. These two different analyses (NIR and
optical) were both done using the same method; determination of stellar atmospheric parameters using
high quality spectra by imposing ionization and excitation equilibrium. In a few summarised points,
this thesis brings:

\begin{itemize}
  \item A new NIR line list consisting of 84 \ion{Fe}{I} lines and 5 \ion{Fe}{II} lines. This line
        list is optimised for usage with high quality spectra. The goal is to analyse FGK stars and
        making the bridge towards the M stars. Especially are the dwarf stars of most interest, in
        the context of exoplanets and the search of habitable worlds. ``Know the star, know the
        exoplanet'' is the key.

        The line list has been tested on few NIR spectra, mostly due to the lack of spectra and the
        difficulties it is to currently obtain any NIR spectra. This is about to change in the
        near future. The tests performed have been increasingly successful, and progress is evident.
  \item To analyse the data a new tool was created, \code{FASMA}. This tool provides many different
        options, can be used with different atmosphere models, is available to the community as a
        web application, and provides three different drivers for analysis of spectra: EW
        measurements, deriving parameters, and deriving abundances of a range of elements. A fourth
        driver is under construction which is to derive parameters using the synthesis method
        \citep{Tsantaki2017}.
  \item The analysis of 50 planet-host stars to increase the number of homogeneously analysed stars
        was also the first appearance and official usage of \code{FASMA}. This analysis increased
        the completeness to 85\% for stars brighter than 10 V magnitude. It is time consuming to
        obtain high quality spectra for stars fainter than this magnitude, however the completeness
        is still at 77\% for stars brighter than 12 V magnitude. Out of the 50 planet host stars
        analysed, eight changed either the radius or mass by more than 25\%. These eight systems
        were carefully analysed, and the updated planetary parameters (radius, mass, and density)
        were derived when possible.
\end{itemize}


\section{Future work}

While the tests with the NIR line list has been successful, there will still be room for
improvements. The same is the case for the \code{FASMA} and SWEET-Cat. Here are some of the main
points for (possible) future work.

\begin{itemize}
  \item The $\log g$ problem
  \item The model atmosphere used is reliable in the Porto group, however newer versions are
        available. It would be interesting to use more recent versions of the ATLAS9 atmosphere
        models, and to explorer other atmosphere models altogether, as the PHOENIX library.
  \item While \code{FASMA} is in a state where it work well and is stable, there are still ways to
        improve it. One way is to use machine learning to get a better guess of initial parameters.
        This work has already been initiated, and details on this can be seen in \cref{cha:fasmaML},
        however the work is not yet connected to \code{FASMA}.\unfinished{Write about non-LTE here?}
  \item After obtaining stellar atmospheric parameters from a high quality spectrum it is natural to
        explorer the spectrum with another purpose: to obtain abundances for other elements. The
        individual abundances of other elements than iron might be used to explorer the history of
        the star, and can be used in connection with the plant-star correlations. This is something
        that is already explored in detail in the optical, but is yet a relative new approach in the
        NIR from high quality spectra.
  \item One of the main limitations during this thesis have been the limited access to data. This
        lead to the analysis of synthetic spectra. It will be very interesting to explorer real data
        in the future. Particular interesting is the CARMENES library which is under constructions.
        This library will contain a high S/N spectrum of all the stars observed by CARMENES. This
        can be used to accurately explorer different part of the parameter space and find weak
        points in the methodology presented here in this thesis.
  \item This was already discussed previously, however, it will be interesting to update SWEET-Cat.
        These updates can include more columns such as abundances of the planet host, modelled mass,
        radius, and age, rotational velocity, etc. The update to SWEET-Cat can also be for the web
        interface. A wish is to update this and include easy-plotting capabilities for users to
        quickly explorer planet-star correlations.
\end{itemize}



\appendix

%!TEX root = thesis.tex
\chapter{NIR line list}
\label{cha:linelist}

\begin{longtable}{rrrrlr}
    \caption{\label{tab:nirll} The NIR iron line list. The $\log g$ are calibrated with \code{FASMA}
             using the ATLAS9 model atmospheres by \citet{Kurucz1993}.}\\
    \hline\hline
    Wavelength  & Atomic number & Excitation potential & $\log \mathit{gf}$ &  Element     &  Solar EW  \\[-0.3em]
      \AA{}     &               &  \si{eV}             &                    &              &  m\AA{}    \\
    \hline
    \endfirsthead
    \caption{continued.}\\
    \hline\hline
    Wavelength  & Atomic number & Excitation potential & $\log \mathit{gf}$ &  Element     &  Solar EW  \\[-0.3em]
      \AA{}     &               &  \si{eV}             &                    &              &  m\AA{}    \\
    \hline
    \endhead
    \hline
    \endfoot
      10065.05  &      26.0     &          4.83        &      -0.279        &  \ion{Fe}{I} &    94.0    \\
      10080.42  &      26.0     &          5.10        &      -1.964        &  \ion{Fe}{I} &     5.9    \\
      10081.39  &      26.0     &          2.42        &      -4.512        &  \ion{Fe}{I} &     6.9    \\
      10086.24  &      26.0     &          2.95        &      -3.978        &  \ion{Fe}{I} &     7.0    \\
      10137.10  &      26.0     &          5.09        &      -1.736        &  \ion{Fe}{I} &     9.8    \\
      10142.84  &      26.0     &          5.06        &      -1.554        &  \ion{Fe}{I} &    14.9    \\
      10145.56  &      26.0     &          4.80        &      -0.118        &  \ion{Fe}{I} &   109.0    \\
      10155.16  &      26.0     &          2.18        &      -4.336        &  \ion{Fe}{I} &    16.2    \\
      10156.51  &      26.0     &          4.59        &      -2.109        &  \ion{Fe}{I} &    12.2    \\
      10167.47  &      26.0     &          2.20        &      -2.319        &  \ion{Fe}{I} &   125.7    \\
      10195.11  &      26.0     &          2.73        &      -3.608        &  \ion{Fe}{I} &    22.6    \\
      10216.31  &      26.0     &          4.73        &       0.047        &  \ion{Fe}{I} &   129.9    \\
      10218.41  &      26.0     &          3.07        &      -2.893        &  \ion{Fe}{I} &    40.9    \\
      10265.22  &      26.0     &          2.22        &      -4.648        &  \ion{Fe}{I} &     8.1    \\
      10307.45  &      26.0     &          4.59        &      -2.432        &  \ion{Fe}{I} &     6.4    \\
      10332.33  &      26.0     &          3.63        &      -3.131        &  \ion{Fe}{I} &    10.5    \\
      10340.89  &      26.0     &          2.20        &      -3.665        &  \ion{Fe}{I} &    46.6    \\
      10347.97  &      26.0     &          5.39        &      -0.717        &  \ion{Fe}{I} &    37.0    \\
      10353.81  &      26.0     &          5.39        &      -0.989        &  \ion{Fe}{I} &    24.2    \\
      10364.06  &      26.0     &          5.45        &      -1.100        &  \ion{Fe}{I} &    18.0    \\
      10379.00  &      26.0     &          2.22        &      -4.236        &  \ion{Fe}{I} &    18.7    \\
      10388.75  &      26.0     &          5.45        &      -1.471        &  \ion{Fe}{I} &     8.7    \\
      10395.80  &      26.0     &          2.18        &      -3.435        &  \ion{Fe}{I} &    61.3    \\
      10423.03  &      26.0     &          2.69        &      -3.658        &  \ion{Fe}{I} &    22.9    \\
      10423.74  &      26.0     &          3.07        &      -3.119        &  \ion{Fe}{I} &    29.9    \\
      10469.65  &      26.0     &          3.88        &      -1.277        &  \ion{Fe}{I} &    89.3    \\
      10532.24  &      26.0     &          3.93        &      -1.650        &  \ion{Fe}{I} &    64.4    \\
      10555.65  &      26.0     &          5.45        &      -1.282        &  \ion{Fe}{I} &    13.1    \\
      10577.14  &      26.0     &          3.30        &      -3.222        &  \ion{Fe}{I} &    17.2    \\
      10616.72  &      26.0     &          3.27        &      -3.306        &  \ion{Fe}{I} &    15.6    \\
      10725.19  &      26.0     &          3.64        &      -2.948        &  \ion{Fe}{I} &    15.7    \\
      10753.00  &      26.0     &          3.96        &      -2.077        &  \ion{Fe}{I} &    39.7    \\
      10780.69  &      26.0     &          3.24        &      -3.553        &  \ion{Fe}{I} &    10.4    \\
      10783.05  &      26.0     &          3.11        &      -2.786        &  \ion{Fe}{I} &    47.0    \\
      10818.28  &      26.0     &          3.96        &      -2.160        &  \ion{Fe}{I} &    35.6    \\
      10863.52  &      26.0     &          4.73        &      -0.877        &  \ion{Fe}{I} &    67.1    \\
      10884.26  &      26.0     &          3.93        &      -2.129        &  \ion{Fe}{I} &    39.1    \\
      10896.30  &      26.0     &          3.07        &      -2.911        &  \ion{Fe}{I} &    42.9    \\
      11013.24  &      26.0     &          4.80        &      -1.240        &  \ion{Fe}{I} &    42.4    \\
      11026.79  &      26.0     &          3.94        &      -2.517        &  \ion{Fe}{I} &    21.2    \\
      11119.80  &      26.0     &          2.85        &      -2.452        &  \ion{Fe}{I} &    84.8    \\
      11641.80  &      26.0     &          4.58        &      -2.116        &  \ion{Fe}{I} &    15.6    \\
      11778.42  &      26.0     &          5.34        &      -1.708        &  \ion{Fe}{I} &     8.4    \\
      12053.08  &      26.0     &          4.56        &      -1.602        &  \ion{Fe}{I} &    41.3    \\
      12119.50  &      26.0     &          4.59        &      -1.897        &  \ion{Fe}{I} &    25.0    \\
      12213.34  &      26.0     &          4.64        &      -2.006        &  \ion{Fe}{I} &    19.1    \\
      12227.11  &      26.0     &          4.61        &      -1.408        &  \ion{Fe}{I} &    51.5    \\
      12244.92  &      26.0     &          3.64        &      -3.222        &  \ion{Fe}{I} &    11.8    \\
      12340.48  &      26.0     &          2.28        &      -4.680        &  \ion{Fe}{I} &     9.4    \\
      12342.92  &      26.0     &          4.64        &      -1.545        &  \ion{Fe}{I} &    42.1    \\
      12510.52  &      26.0     &          4.96        &      -1.930        &  \ion{Fe}{I} &    12.9    \\
      12557.00  &      26.0     &          2.28        &      -4.026        &  \ion{Fe}{I} &    33.8    \\
      12615.93  &      26.0     &          4.64        &      -1.686        &  \ion{Fe}{I} &    35.7    \\
      12638.70  &      26.0     &          4.56        &      -0.679        &  \ion{Fe}{I} &   112.3    \\
      12807.15  &      26.0     &          3.64        &      -2.649        &  \ion{Fe}{I} &    37.1    \\
      12808.24  &      26.0     &          4.99        &      -1.811        &  \ion{Fe}{I} &    16.4    \\
      12824.86  &      26.0     &          3.02        &      -3.612        &  \ion{Fe}{I} &    20.1    \\
      12840.57  &      26.0     &          4.96        &      -1.612        &  \ion{Fe}{I} &    25.3    \\
      12879.77  &      26.0     &          2.28        &      -3.525        &  \ion{Fe}{I} &    68.7    \\
      12896.12  &      26.0     &          4.91        &      -1.713        &  \ion{Fe}{I} &    23.2    \\
      12933.01  &      26.0     &          5.02        &      -1.879        &  \ion{Fe}{I} &    13.9    \\
      12934.67  &      26.0     &          5.39        &      -1.103        &  \ion{Fe}{I} &    30.9    \\
      13014.84  &      26.0     &          5.45        &      -1.542        &  \ion{Fe}{I} &    12.3    \\
      13352.17  &      26.0     &          5.31        &      -0.355        &  \ion{Fe}{I} &    94.4    \\
      13392.10  &      26.0     &          5.35        &      -0.105        &  \ion{Fe}{I} &   115.1    \\
      15194.49  &      26.0     &          2.22        &      -4.808        &  \ion{Fe}{I} &    14.1    \\
      15201.57  &      26.0     &          5.49        &      -1.315        &  \ion{Fe}{I} &    29.0    \\
      15207.53  &      26.0     &          5.38        &       0.311        &  \ion{Fe}{I} &   215.9    \\
      15335.38  &      26.0     &          5.41        &       0.252        &  \ion{Fe}{I} &   205.2    \\
      15490.34  &      26.0     &          2.20        &      -4.787        &  \ion{Fe}{I} &    16.1    \\
      15593.74  &      26.0     &          5.03        &      -1.796        &  \ion{Fe}{I} &    28.0    \\
      15611.15  &      26.0     &          3.42        &      -2.966        &  \ion{Fe}{I} &    51.6    \\
      15631.95  &      26.0     &          5.35        &       0.171        &  \ion{Fe}{I} &   207.0    \\
      15648.51  &      26.0     &          5.43        &      -0.633        &  \ion{Fe}{I} &    93.8    \\
      15676.58  &      26.0     &          5.11        &      -1.848        &  \ion{Fe}{I} &    22.3    \\
      16198.50  &      26.0     &          5.41        &      -0.376        &  \ion{Fe}{I} &   131.4    \\
      17420.83  &      26.0     &          3.88        &      -3.628        &  \ion{Fe}{I} &     6.7    \\
      19923.34  &      26.0     &          5.02        &      -1.536        &  \ion{Fe}{I} &    49.7    \\
      21851.38  &      26.0     &          3.64        &      -3.578        &  \ion{Fe}{I} &    12.7    \\
      22257.11  &      26.0     &          5.06        &      -0.704        &  \ion{Fe}{I} &   132.5    \\
      22380.80  &      26.0     &          5.03        &      -0.377        &  \ion{Fe}{I} &   179.4    \\
      22392.88  &      26.0     &          5.10        &      -1.330        &  \ion{Fe}{I} &    60.8    \\
      22619.84  &      26.0     &          4.99        &      -0.564        &  \ion{Fe}{I} &   158.2    \\
      23308.48  &      26.0     &          4.08        &      -2.705        &  \ion{Fe}{I} &    31.3    \\
      10427.31  &      26.1     &          6.08        &      -1.575        & \ion{Fe}{II} &    13.7    \\
      10501.50  &      26.1     &          5.55        &      -1.861        & \ion{Fe}{II} &    19.5    \\
      10862.64  &      26.1     &          5.59        &      -2.006        & \ion{Fe}{II} &    15.3    \\
      11125.58  &      26.1     &          5.62        &      -2.213        & \ion{Fe}{II} &    10.5    \\
      13251.14  &      26.1     &          9.41        &       0.768        & \ion{Fe}{II} &    13.4    \\
\end{longtable}

\begin{landscape}
\chapter{SWEET-Cat update of 50 planet hosts}
\label{cha:SCtables}

\begin{ThreePartTable}
% \advance\vsize5cm
% \csname @colroom\endcsname=\vsize
\textheight=\vsize
\csname @colht\endcsname=\vsize
% \setlength\extrarowheight{4pt}
\hskip-4em  % push the table a bit to the left...
\begin{longtable}{lllrlclr}

\caption{\label{tab:SCresults}
         Derived parameters for the 50 stars in our sample. The S/N was measured by ARES.}\\

 \hline\hline
 Star  & $T_\mathrm{eff}$ & $\log g$ & $[\ion{Fe}/\ion{H}]$ &  $\xi_\mathrm{micro}$ &  $\xi_\mathrm{micro}$ fixed?  &  Instrument  &  S/N  \\[-0.3em]
       &       [K]        &  [cgs]   &                      &     [km/s]            &                               &              &       \\
 \hline
 \endfirsthead
 \caption{continued.}\\
 \hline\hline
 Star  & $T_\mathrm{eff}$ & $\log g$ & $[\ion{Fe}/\ion{H}]$ &  $\xi_\mathrm{micro}$ &  $\xi_\mathrm{micro}$ fixed?  &  Instrument  &  S/N  \\[-0.3em]
       &       [K]        &  [cgs]   &                      &     [km/s]            &                               &              &       \\
 \hline
 \endhead
 \hline
 \endfoot
 \endlastfoot
 \hline
      \object{BD -11 4672}    &   \num{4553(75)}    &  \num{4.87(51)}             &  \num{-0.30(02)}  &  \num{0.14(07)}  & yes  &  FIES             &  487  \\
      \object{BD +49  828}    &   \num{5015(36)}    &  \num{2.87(9)}\tnote{a}     &  \num{-0.01(03)}  &  \num{1.48(04)}  & no   &  FIES             &  567  \\
      \object{GJ 785}         &   \num{5087(48)}    &  \num{4.42(10)}             &  \num{-0.01(03)}  &  \num{0.69(10)}  & no   &  HARPS            &  801  \\
      \object{HATS-1}         &   \num{5969(46)}    &  \num{4.39(6)}              &  \num{-0.04(04)}  &  \num{1.06(08)}  & no   &  UVES             &  155  \\
      \object{HATS-5}         &   \num{5383(91)}    &  \num{4.41(22)}             &  \num{ 0.08(06)}  &  \num{0.91(14)}  & no   &  UVES             &  158  \\
      \object{HAT-P-12}       &   \num{4642(106)}   &  \num{4.53(27)}             &  \num{-0.26(06)}  &  \num{0.28(63)}  & no   &  FIES             &  185  \\
      \object{HAT-P-24}       &   \num{6470(181)}   &  \num{4.33(27)}             &  \num{-0.41(10)}  &  \num{1.40(03)}  & yes  &  UVES             &  158  \\
      \object{HAT-P-39}       &   \num{6745(236)}   &  \num{4.39(47)}             &  \num{-0.21(12)}  &  \num{1.53(04)}  & yes  &  UVES             &  127  \\
      \object{HAT-P-42}       &   \num{5903(66)}    &  \num{4.29(10)}\tnote{a}    &  \num{ 0.34(05)}  &  \num{1.19(08)}  & no   &  UVES             &  130  \\
      \object{HAT-P-46}       &   \num{6421(121)}   &  \num{4.53(14)}\tnote{a}    &  \num{ 0.16(09)}  &  \num{1.67(18)}  & no   &  UVES             &  208  \\[5pt]
      \object{HD 120084}      &   \num{4969(40)}    &  \num{2.94(14)}\tnote{a}    &  \num{ 0.12(03)}  &  \num{1.41(04)}  & no   &  ESPaDOnS         &  852  \\
      \object{HD 192263}      &   \num{4946(46)}    &  \num{4.61(14)}             &  \num{-0.05(02)}  &  \num{0.66(12)}  & no   &  HARPS            &  415  \\
      \object{HD 219134}      &   \num{4767(70)}    &  \num{4.57(17)}             &  \num{ 0.00(04)}  &  \num{0.59(24)}  & no   &  ESPaDOnS         &  725  \\
      \object{HD 220842}      &   \num{5999(39)}    &  \num{4.30(6)}\tnote{a}     &  \num{-0.08(03)}  &  \num{1.21(05)}  & no   &  FIES             &  459  \\
      \object{HD 233604}      &   \num{4954(46)}    &  \num{2.86(11)}\tnote{a}    &  \num{-0.14(04)}  &  \num{1.61(05)}  & no   &  FIES             &  314  \\
      \object{HD 283668}      &   \num{4841(73)}    &  \num{4.51(18)}             &  \num{-0.74(04)}  &  \num{0.16(61)}  & no   &  FIES             &  592  \\
      \object{HD 285507}      &   \num{4620(126)}   &  \num{4.72(61)}             &  \num{ 0.04(06)}  &  \num{0.74(43)}  & no   &  UVES             &  239  \\
      \object{HD 5583}        &   \num{4986(35)}    &  \num{2.87(9)}\tnote{a}     &  \num{-0.35(03)}  &  \num{1.62(04)}  & no   &  FIES             &  933  \\
      \object{HD 81688}       &   \num{4903(21)}    &  \num{2.70(5)}\tnote{a}     &  \num{-0.21(02)}  &  \num{1.54(02)}  & no   & \tnote{b}         & 1350, 860  \\
      \object{HD 82886}       &   \num{5123(18)}    &  \num{3.30(4)}\tnote{a}     &  \num{-0.25(01)}  &  \num{1.16(02)}  & no   & \tnote{c}         & 1198,1294  \\[5pt]
      \object{HD 87883}       &   \num{4917(68)}    &  \num{4.53(19)}             &  \num{ 0.02(03)}  &  \num{0.46(21)}  & no   &  ESPaDOnS         &  753  \\
      \object{HIP 107773}     &   \num{4957(49)}    &  \num{2.83(9)}\tnote{a}     &  \num{ 0.04(04)}  &  \num{1.49(05)}  & no   &  UVES             &  218  \\
      \object{HIP 11915}      &   \num{5770(14)}    &  \num{4.33(3)}              &  \num{-0.06(01)}  &  \num{0.95(02)}  & no   &  HARPS            &  709  \\
      \object{HIP 116454}     &   \num{5042(72)}    &  \num{4.69(15)}             &  \num{-0.16(03)}  &  \num{0.71(17)}  & no   &  UVES             &  412  \\
      \object{HR 228}         &   \num{5042(42)}    &  \num{3.30(9)}\tnote{a}     &  \num{ 0.07(03)}  &  \num{1.14(04)}  & no   &  UVES             &  400  \\
      \object{KELT-6}         &   \num{6246(88)}    &  \num{4.22(9)}\tnote{a}     &  \num{-0.22(06)}  &  \num{1.66(13)}  & no   &  FIES             &  374  \\
      \object{Kepler-37}      &   \num{5378(53)}    &  \num{4.47(12)}             &  \num{-0.23(04)}  &  \num{0.58(13)}  & no   &  FIES             &  205  \\
      \object{Kepler-444}     &   \num{5111(43)}    &  \num{4.50(13)}             &  \num{-0.51(03)}  &  \num{0.37(15)}  & no   &  FIES             &  675  \\
      \object{mu Leo}         &   \num{4605(94)}    &  \num{2.61(26)}\tnote{a}    &  \num{ 0.25(06)}  &  \num{1.64(11)}  & no   &  ESPaDOnS         &  354  \\[5pt]
      \object{ome Ser}        &   \num{4928(35)}    &  \num{2.69(6)}\tnote{a}     &  \num{-0.11(03)}  &  \num{1.55(04)}  & no   &  FIES             & 1168  \\
      \object{omi UMa}        &   \num{5499(52)}    &  \num{3.36(7)}\tnote{a}     &  \num{-0.01(05)}  &  \num{1.98(06)}  & no   &  ESPaDOnS         &  527  \\
      \object{Qatar-2}        &   \num{4637(316)}   &  \num{4.53(62)}             &  \num{ 0.09(17)}  &  \num{0.63(83)}  & no   &  UVES             &   97  \\
      \object{SAND364}        &   \num{4457(104)}   &  \num{2.26(20)}\tnote{a}    &  \num{-0.04(06)}  &  \num{1.60(11)}  & no   &  UVES             &  220  \\
      \object{TYC+1422-614-1} &   \num{4908(41)}    &  \num{2.90(12)}\tnote{a}    &  \num{-0.07(03)}  &  \num{1.57(05)}  & no   &  FIES             &  506  \\
      \object{WASP-37}        &   \num{5917(72)}    &  \num{4.25(15)}             &  \num{-0.23(05)}  &  \num{0.59(13)}  & no   &  FIES             &  232  \\
      \object{WASP-44}        &   \num{5612(80)}    &  \num{4.39(30)}             &  \num{ 0.17(06)}  &  \num{1.32(13)}  & no   &  UVES             &  125  \\
      \object{WASP-52}        &   \num{5197(83)}    &  \num{4.55(30)}             &  \num{ 0.15(05)}  &  \num{1.16(14)}  & no   &  UVES             &  125  \\
      \object{WASP-58}        &   \num{6039(55)}    &  \num{4.23(10)}             &  \num{-0.09(04)}  &  \num{1.12(08)}  & no   &  FIES             &  310  \\
      \object{WASP-61}        &   \num{6265(168)}   &  \num{4.21(21)}\tnote{a}    &  \num{-0.38(11)}  &  \num{1.44(02)}  & yes  &  UVES             &  163  \\[5pt]
      \object{WASP-72}        &   \num{6570(85)}    &  \num{4.25(13)}             &  \num{ 0.15(06)}  &  \num{2.30(15)}  & no   &  UVES             &  174  \\
      \object{WASP-73}        &   \num{6203(32)}    &  \num{4.16(6)}\tnote{a}     &  \num{ 0.20(02)}  &  \num{1.66(04)}  & no   & \tnote{d}         & 193,231 \\
      \object{WASP-75}        &   \num{6203(46)}    &  \num{4.42(22)}\tnote{a}    &  \num{ 0.24(03)}  &  \num{1.45(06)}  & no   &  UVES             &  189  \\
      \object{WASP-76}        &   \num{6347(52)}    &  \num{4.29(8)}\tnote{a}     &  \num{ 0.36(04)}  &  \num{1.73(06)}  & no   &  UVES             &  165  \\
      \object{WASP-82}        &   \num{6563(55)}    &  \num{4.29(10)}\tnote{a}    &  \num{ 0.18(04)}  &  \num{1.93(08)}  & no   &  UVES             &  239  \\
      \object{WASP-88}        &   \num{6450(61)}    &  \num{4.24(6)}\tnote{a}     &  \num{ 0.03(04)}  &  \num{1.79(09)}  & no   &  UVES             &  174  \\
      \object{WASP-94 A}      &   \num{6259(34)}    &  \num{4.34(7)}\tnote{a}     &  \num{ 0.35(03)}  &  \num{1.50(04)}  & no   &  UVES             &  356  \\
      \object{WASP-94 B}      &   \num{6137(21)}    &  \num{4.42(5)}\tnote{a}     &  \num{ 0.33(02)}  &  \num{1.29(03)}  & no   &  UVES             &  397  \\
      \object{WASP-95}        &   \num{5799(31)}    &  \num{4.29(5)}\tnote{a}     &  \num{ 0.22(03)}  &  \num{1.18(04)}  & no   &  UVES             &  247  \\
      \object{WASP-97}        &   \num{5723(52)}    &  \num{4.24(7)}              &  \num{ 0.31(04)}  &  \num{1.03(08)}  & no   &  UVES             &  219  \\[5pt]
      \object{WASP-99}        &   \num{6324(89)}    &  \num{4.34(12)}             &  \num{ 0.27(06)}  &  \num{1.83(12)}  & no   &  UVES             &  249  \\
      \object{WASP-100}       &   \num{6853(209)}   &  \num{4.15(26)}\tnote{a}    &  \num{-0.30(12)}  &  \num{1.87(02)}  & yes  &  UVES             &  166  \\
  \hline
\end{longtable}
  \begin{tablenotes}
    \item [a] Spectroscopic $\log g$.
    \item [b] Weighted average of ESPaDOnS and FIES results.
              The parameters are (FIES in parentheses):
              $T_\mathrm{eff}=4870(4934)\pm30(29)$,
              $\log g=2.50(2.73)\pm0.14(0.05)$,
              $[\ion{Fe}/\ion{H}]=-0.26(-0.19)\pm0.03(0.02)$, and
              $\xi_\mathrm{micro}=1.50(1.59)\pm0.03(0.03)$.
    \item [c] Weighted average of ESPaDOnS and FIES results.
              The parameters are (FIES in parentheses):
              $T_\mathrm{eff}=5124(5121)\pm22(29)$,
              $\log g=3.30(3.31)\pm0.05(0.07)$,
              $[\ion{Fe}/\ion{H}]=-0.25(-0.24)\pm0.02(0.02)$, and
              $\xi_\mathrm{micro}=1.15(1.17)\pm0.03(0.04)$.
    \item [d] Weighted average of UVES and FEROS results.
              The parameters are (FEROS in parentheses):
              $T_\mathrm{eff}=6313(6162)\pm61(37)$,
              $\log g=4.26(4.14)\pm0.15(0.06)$,
              $[\ion{Fe}/\ion{H}]=0.22(0.19)\pm0.04(0.03)$, and
              $\xi_\mathrm{micro}=1.85(1.61)\pm0.08(0.04)$.
  \end{tablenotes}
\end{ThreePartTable}

\clearpage
\begin{longtable}{lllrll}
    \caption{\label{tab:oldSC} Previous parameters from SWEET-Cat.}\\
    \hline\hline
    Star  & $T_\mathrm{eff}$ & $\log g$ & $[\ion{Fe}/\ion{H}]$ &  $\xi_\mathrm{micro}$ &  Reference  \\[-0.3em]
          &       [K]        &  [cgs]   &                      &     [km/s]            &             \\
    \hline
    \endfirsthead
    \caption{continued.}\\
    \hline\hline
    Star  & $T_\mathrm{eff}$ & $\log g$ & $[\ion{Fe}/\ion{H}]$ &  $\xi_\mathrm{micro}$ &  Reference  \\[-0.3em]
          &       [K]        &  [cgs]   &                      &     [km/s]            &             \\
    \hline
    \endhead
    \hline
    \endfoot
    \object{BD-114672}       &    $4475 \pm 100$   &    $4.10 \pm 0.36$   &    $-0.48 \pm 0.05$   &    $0.67 \pm 0.16$   &    \citet{Moutou2015}       \\
    \object{BD +49 828}      &    $4943 \pm  30$   &    $2.85 \pm 0.09$   &    $-0.19 \pm 0.06$   &          ...         &    \citet{Niedzielski2015}  \\
    \object{GJ 785}          &    $5144 \pm  50$   &    $4.60 \pm 0.06$   &    $ 0.08 \pm 0.03$   &          ...         &    \citet{Howard2011}       \\
    \object{HATS-1}          &    $5780 \pm 100$   &    $4.40 \pm 0.08$   &    $-0.06 \pm 0.12$   &          ...         &    \citet{Penev2013}        \\
    \object{HATS-5}          &    $5304 \pm  50$   &    $4.53 \pm 0.02$   &    $ 0.19 \pm 0.08$   &          ...         &    \citet{Zhou2014}         \\
    \object{HAT-P-12}        &    $4650 \pm  60$   &    $4.61 \pm 0.02$   &    $-0.29 \pm 0.05$   &          ...         &    \citet{Lee2014}          \\
    \object{HAT-P-24}        &    $6373 \pm  80$   &    $4.29 \pm 0.04$   &    $-0.16 \pm 0.08$   &          ...         &    \citet{Kipping2010}      \\
    \object{HAT-P-39}        &    $6340 \pm 100$   &    $4.16 \pm 0.03$   &    $ 0.19 \pm 0.10$   &          ...         &    \citet{Hartman2012}      \\
    \object{HAT-P-46}        &    $6120 \pm 100$   &    $4.25 \pm 0.11$   &    $ 0.30 \pm 0.10$   &    $0.85 \pm  ...$   &    \citet{Hartman2014}      \\
    \object{HAT-P-42}        &    $5743 \pm  50$   &    $4.14 \pm 0.07$   &    $ 0.27 \pm 0.08$   &          ...         &    \citet{Boisse2013}       \\
    \object{HD 120084}       &    $4892 \pm  22$   &    $2.71 \pm 0.08$   &    $ 0.09 \pm 0.05$   &    $1.31 \pm 0.10$   &    \citet{Sato2013}         \\
    \object{HD 192263}       &    $4906 \pm  57$   &    $4.36 \pm 0.17$   &    $-0.07 \pm 0.02$   &    $0.78 \pm 0.12$   &    \citet{Tsantaki2013}     \\
    \object{HD 219134}       &    $4699 \pm  16$   &    $4.63 \pm 0.10$   &    $ 0.11 \pm 0.04$   &    $0.35 \pm 0.19$   &    \citet{Motalebi2015}     \\
    \object{HD 220074}       &    $3935 \pm 110$   &    $1.30 \pm 0.50$   &    $-0.25 \pm 0.25$   &    $1.60 \pm 0.30$   &    \citet{Lee2013}          \\
    \object{HD 220842}       &    $5920 \pm  20$   &    $4.24 \pm 0.02$   &    $-0.17 \pm 0.02$   &          ...         &    \citet{Hebrard2016}      \\
    \object{HD 233604}       &    $4791 \pm  45$   &    $2.55 \pm 0.18$   &    $-0.36 \pm 0.04$   &          ...         &    \citet{Nowak2013}        \\
    \object{HD 283668}       &    $4845 \pm  66$   &    $4.35 \pm 0.12$   &    $-0.75 \pm 0.12$   &    $0.02 \pm 0.30$   &    \citet{Wilson2016}       \\
    \object{HD 285507}       &    $4503 \pm  73$   &    $4.67 \pm 0.06$   &    $ 0.13 \pm 0.01$   &          ...         &    \citet{Quinn2014}        \\
    \object{HD 5583}         &    $4830 \pm  45$   &    $2.53 \pm 0.14$   &    $-0.50 \pm 0.18$   &          ...         &    \citet{Niedzielski2016}  \\
    \object{HD 81688}        &    $4753 \pm  15$   &    $2.22 \pm 0.05$   &    $-0.36 \pm 0.02$   &    $1.43 \pm 0.05$   &    \citet{Sato2008}         \\
    \object{HD 82886}        &    $5112 \pm  44$   &    $3.40 \pm 0.06$   &    $-0.31 \pm 0.03$   &          ...         &    \citet{Johnson2011}      \\
    \object{HD 87883}        &    $4958 \pm  44$   &    $4.56 \pm 0.06$   &    $ 0.07 \pm 0.03$   &          ...         &    \citet{Valenti2005}      \\
    \object{HIP 107773}      &    $4945 \pm 100$   &    $2.60 \pm 0.20$   &    $ 0.03 \pm 0.10$   &          ...         &    \citet{Jones2015}        \\
    \object{HIP 11915}       &    $5760 \pm   4$   &    $4.46 \pm 0.01$   &    $-0.06 \pm 0.00$   &          ...         &    \citet{Bedell2015}       \\
    \object{HIP 116454}      &    $5089 \pm  50$   &    $4.59 \pm 0.03$   &    $-0.16 \pm 0.08$   &          ...         &    \citet{Vanderburg2015}   \\
    \object{HR 228}          &    $4959 \pm  25$   &    $3.16 \pm 0.08$   &    $ 0.01 \pm 0.04$   &    $1.12 \pm 0.07$   &    \citet{Sato2013b}        \\
    \object{KELT-6}          &    $6102 \pm  43$   &    $4.07 \pm 0.06$   &    $-0.28 \pm 0.04$   &          ...         &    \citet{Collins2014}      \\
    \object{Kepler-37}       &    $5417 \pm  70$   &    $4.57 \pm 0.01$   &    $-0.32 \pm 0.07$   &          ...         &    \citet{Barclay2013}      \\
    \object{Kepler-444}      &    $5046 \pm  74$   &    $4.60 \pm 0.06$   &    $-0.55 \pm 0.07$   &          ...         &    \citet{Campante2015}     \\
    \object{mu Leo}          &    $4538 \pm  27$   &    $2.40 \pm 0.10$   &    $ 0.36 \pm 0.05$   &    $1.40 \pm 0.10$   &    \citet{Lee2014}          \\
    \object{ome Ser}         &    $4770 \pm  10$   &    $2.32 \pm 0.04$   &    $-0.24 \pm 0.02$   &    $1.34 \pm 0.04$   &    \citet{Sato2013}         \\
    \object{omi UMa}         &    $5242 \pm  10$   &    $2.64 \pm 0.03$   &    $-0.09 \pm 0.02$   &    $1.51 \pm 0.07$   &    \citet{Sato2012}         \\
    \object{Qatar-2}         &    $4645 \pm  50$   &    $4.60 \pm 0.02$   &    $-0.02 \pm 0.08$   &          ...         &    \citet{Bryan2012}        \\
    \object{SAND364}         &    $4284 \pm   9$   &    $2.20 \pm 0.06$   &    $-0.02 \pm 0.04$   &          ...         &    \citet{Brucalassi2014}   \\
    \object{TYC+1422-614-1}  &    $4806 \pm  45$   &    $2.85 \pm 0.18$   &    $-0.20 \pm 0.08$   &          ...         &    \citet{Niedzielski2015a} \\
    \object{WASP-37}         &    $5940 \pm  55$   &    $4.39 \pm 0.02$   &    $-0.40 \pm 0.12$   &          ...         &    \citet{Simpson2011}      \\
    \object{WASP-44}         &    $5400 \pm 150$   &    $4.48 \pm 0.07$   &    $ 0.06 \pm 0.10$   &          ...         &    \citet{Anderson2012}     \\
    \object{WASP-52}         &    $5000 \pm 100$   &    $4.58 \pm 0.01$   &    $ 0.03 \pm 0.12$   &          ...         &    \citet{Hebrard2013}      \\
    \object{WASP-58}         &    $5800 \pm 150$   &    $4.27 \pm 0.09$   &    $-0.45 \pm 0.09$   &          ...         &    \citet{Hebrard2013}      \\
    \object{WASP-61}         &    $6250 \pm 150$   &    $4.26 \pm 0.01$   &    $-0.10 \pm 0.12$   &          ...         &    \citet{Hellier2012}      \\
    \object{WASP-72}         &    $6250 \pm 100$   &    $4.08 \pm 0.13$   &    $-0.06 \pm 0.09$   &    $1.60 \pm 0.10$   &    \citet{Gillon2013}       \\
    \object{WASP-73}         &    $6030 \pm 120$   &    $3.92 \pm 0.08$   &    $ 0.14 \pm 0.14$   &    $1.10 \pm 0.20$   &    \citet{Delrez2014}       \\
    \object{WASP-75}         &    $6100 \pm 100$   &    $4.50 \pm 0.10$   &    $ 0.07 \pm 0.09$   &    $1.30 \pm 0.10$   &    \citet{Gomez2013}        \\
    \object{WASP-76}         &    $6250 \pm 100$   &    $4.13 \pm 0.02$   &    $ 0.23 \pm 0.10$   &    $1.40 \pm 0.10$   &    \citet{West2016}         \\
    \object{WASP-82}         &    $6490 \pm 100$   &    $3.97 \pm 0.02$   &    $ 0.12 \pm 0.11$   &    $1.50 \pm 0.10$   &    \citet{West2016}         \\
    \object{WASP-88}         &    $6430 \pm 130$   &    $4.03 \pm 0.09$   &    $-0.08 \pm 0.12$   &    $1.40 \pm 0.10$   &    \citet{Delrez2014}       \\
    \object{WASP-94 A}       &    $6170 \pm  80$   &    $4.27 \pm 0.07$   &    $ 0.26 \pm 0.15$   &          ...         &   \citet{Neveu2014}         \\
    \object{WASP-94 B}       &    $6040 \pm  90$   &    $4.26 \pm 0.06$   &    $ 0.23 \pm 0.14$   &          ...         &   \citet{Neveu2014}         \\
    \object{WASP-95}         &    $5630 \pm 130$   &    $4.38 \pm 0.03$   &    $ 0.14 \pm 0.16$   &          ...         &    \citet{Hellier2014}      \\
    \object{WASP-97}         &    $5640 \pm 100$   &    $4.43 \pm 0.03$   &    $ 0.23 \pm 0.11$   &          ...         &    \citet{Hellier2014}      \\
    \object{WASP-99}         &    $6180 \pm 100$   &    $4.12 \pm 0.03$   &    $ 0.21 \pm 0.15$   &          ...         &    \citet{Hellier2014}      \\
    \object{WASP-100}        &    $6900 \pm 120$   &    $4.04 \pm 0.11$   &    $-0.03 \pm 0.10$   &          ...         &    \citet{Hellier2014}      \\
\end{longtable}


\end{landscape}

%!TEX root = thesis.tex
\chapter{\code{FASMA} and machine learning}
\label{cha:fasmaML}

With the increasing amount of astrophysical data, it is important to perform a rapid and trustworthy
analysis. This is one of the strengths for \code{FASMA} when dealing with high quality spectroscopic
data for determination of stellar atmospheric parameters. Here a classic method to analyse these
large amount of data has been modernised with a new minimisation technique that utilise the physical
knowledge about the system. However smart it may sound like, it can be improved. The time consuming
part in \code{FASMA} are the calls to \code{MOOG}. In the end a couple of minutes are spent on the
minimisation on a modern computer.

This lead to a side-project: explorer the use of machine learning to determine stellar atmospheric
parameters. Machine learning (ML) is not a new topic in computer science, but it is a tool that is
steadily becoming more and more popular in today's science. The effort of using ML here is meant as
a proof of concept, and something that can be improved upon in the future.

The idea was to remove the expensive calls to \code{MOOG} completely. Two things are required in
order to do so in the approach presented here:
\begin{enumerate}
  \item Stellar atmospheric parameters of a large sample of stars with a big span in the parameter
        space
  \item The EW measurements of as many absorption lines as possible (in this case \ion{Fe}{I} and
        \ion{Fe}{II})
\end{enumerate}
Additionally it is here required that the above two points are obtained in a homogeneous way.
Luckily such a sample was already analysed during this thesis as a test of \code{FASMA} (see
\sref{sec:fasma_test}) where a sample of 583 stars were analysed; all of which meet the above
required criteria.

This data set of measurements of EWs and the parameters were organised and prepared in a big table
as follows:
\begin{itemize}
  \item Each row contains both the measurements of the EWs (first N columns) and the parameters
        (last four columns). The parameters are $T_\mathrm{eff}$, $\log g$, $[\ion{Fe}/\ion{H}]$,
        and $\xi_\mathrm{micro}$
  \item The columns excluding the last four are labelled with the wavelength of the absorption line
  \item All wavelength columns which contained at least one missing measurement of the EW for any
        of the 583 stars were removed
\end{itemize}
There are 58 wavelength columns after removing wavelength columns with missing measurements. Before
the removal there were 299 columns with wavelength. In this case the \code{scikit-learn}
package\footnote{\url{http://scikit-learn.org/}} from the Python ecosystem was used to train the
data set. This also package also include a tool to split the data into a training part and testing
part. This is particular useful when trying to evaluate the accuracy of the model used. Here the
training consist of 2/3 of the data, and the rest is for testing. This splitting is done randomly,
giving slight different results each time the script is executed.

The training itself is quite fast (less than \SI{10}{s}). However, the real power comes when the
trained model is saved to the disk, which can later be loaded again. In this way the training will
only be done once. Using the model to obtain the parameters is much less than \SI{1}{s}, which makes
is many orders of magnitudes faster\footnote{Improvements in the order of millions have been
obtained here.} than a more traditional approach as with \code{FASMA}.

When testing the model trained, the 1/3 data set is used to derive parameters. Those derived
parameters are then compared to the actual parameters with a mean absolute error. This gives an idea
of the accuracy. In \fref{fig:ml} the script was run 1000 times (the splitting was done randomly in
each run), and the error for each parameter is shown as a histogram. The mean errors for each
parameters are roughly $T_\mathrm{eff}:\SI{47}{K}$, $\log g:\SI{0.10}{dex}$,
$[\ion{Fe}/\ion{H}]:\SI{0.038}{dex}$, and $\xi_\mathrm{micro}:\SI{0.12}{km/s}$.

\begin{figure}[htpb!]
    \centering
    \includegraphics[width=1.0\linewidth]{figures/ML.pdf}
    \caption{Mean absolute error on each parameter after 1000 runs.}
    \label{fig:ml}
\end{figure}

There exists many different algorithms within \code{scikit-learn} to train the final model. In the
test here a simple \code{LinearRegression} was used. Other algorithms were tested as well, such as
\code{Ridge} and \code{Lasso}, however with very similar results. The main difference between the
different algorithms lies in the details on how the minimisation is done. This is described in great
detail in the online documentation.



%% This adds a line for the Bibliography in the Table of Contents.
\addcontentsline{toc}{chapter}{Bibliography}
% \bibliographystyle{plain}
%% *** Set the bibliography file. ***
\bibliography{thesis}
\bibliographystyle{astron}
\nocite{} %include everything.


\end{document}
