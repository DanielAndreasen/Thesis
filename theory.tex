\chapter{Theory}

To encompass all theory regarding stellar structure, evolution, and their
atmosphere is far beyond the scope of this thesis. Rather the theory needed is
presented below with highlights on the most important aspects.

\section{Stellar structure}

The structure of a non-rotating spherical stars can be described by five rather
simple differential equations \citep[see e.g.][]{kippenhahn} presented below:
\begin{enumerate}
    \item \textbf{Equation of Continuity}
        \nicebreak
        Relation between the mass, $m$, the density, $\rho$, at a symmetric
        shell at radius $r$
        \begin{align}
            \Aboxed{\pd{r}{m} &= \frac{1}{4\pi r^2\rho}.}
        \end{align}

    \item \textbf{Equation of Hydrostatic Equilibrium}
        \nicebreak
        The equation of hydrostatic equilibrium shows how a star in equilibrium
        is balanced between two forces. The inward force from gravity and the
        outward force from pressure, $P$,
        \begin{align}
            \Aboxed{\pd{P}{m} &= -\frac{Gm}{4\pi r^4}.}
        \end{align}
        When working with asteroseismilogy a time dependent pertubation to this
        equation is added \citep[see e.g.][for a thorough discussion]{aerts}.
        However, this term is neglected here.


    \item \textbf{Equation of Energy Conservation}
        \nicebreak
        The equation of energy conservation shows how the energy is produced and
        lost throughout the star.
        \begin{align}
            \Aboxed{\pd{l}{m} = \epsilon - \epsilon_\nu + \epsilon_g,}
        \end{align}
        where $\epsilon$ is the energy production in the center of the star,
        $\epsilon_\nu$ is the energy lost by neutrinos which is always
        positive, $\epsilon_g$ is a source function of time-dependent terms,
        and $l$ is the luminosity at $m$. $\epsilon_g$ comes from the fact that
        non-stationary shells can change its internal energy, and thus exchange
        mechanical energy with neighboring shells.

    \item \textbf{Equation of Energy Transport}
        \nicebreak
        Energy transportation throughout the star is described with the
        following equation
        \begin{align}
            \Aboxed{\pd{T}{m} &= -\frac{GmT}{4\pi r^4P}\nabla_\tm{rad},}
        \end{align}
        where $\nabla_\tm{rad}$ is the radiative temperature gradient, and $T$
        is the temperature. The value of the temperature gradient compared to
        the radiative temperature gradient tells if the energy is transported by
        convection or radiation. In our Sun the outer layer are convective while
        the inner layer are radiative.

    \item \textbf{Equation of Chemical Composition}
        \nicebreak
        In this last equation we see the evolution of an element, $X_i$, when
        it reacts with other elements with reaction rates $r_{ji}$ and $r_{ik}$
        \begin{align}
            \Aboxed{\pd{X_i}{t} &= \frac{m_i}{\rho} \Bigl( \sum_j r_{ji} - \sum_k r_{ik}\Bigr).}
        \end{align}
        Note that this is the only time-dependent equation of the five
        presented.
\end{enumerate}

These five fundamental equations are implemented in stellar evolutionary codes,
which we will use in later chapters. The many different codes that exist take
other things into account, e.g the star can rotate, and it may not always be in
hydrostatic equilibrium (this is important if we want our star to pulsate). For
simplicity we have only presented time-dependence in the Equation of Chemical
Composition since timescales of rotation, pulsations, and activity are much
shorter than the long timescale found in chemical composition changes.


\section{Stellar atmosphere}
