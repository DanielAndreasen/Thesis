%!TEX root = master.tex
\chapter{Spectrographs}
\label{cha:spectrographs}

This chapter will be focused on some of the available spectrographs for deriving
stellar atmospheric parameters using the method described in
\sref{sec:parameters}. For this we need a spectrum of both high resolution and
high signal-to-noise ratio, S/N, which combined will be called a high quality
spectrum. In order to get reliable results, a spectral resolution of at least
$R=\frac{\lambda}{\Delta\lambda}=25\,000$ is needed. A $S/N\approx 100$ is at
least needed to obtain the parameters. These are approximately values since it
can vary across different spectral classes, e.g. it is often relatively easy to
obtain parameters of a solar-twin while it gets increasingly difficult as
especially $T_\mathrm{eff}$ diverges in either direction from the solar value.
Naturally the higher quality the spectrum is (both resolution and S/N), the
better the results will be obtained. It is common practise to increase the S/N
for a spectrum by co-adding a sample of lower S/N spectra of the same star from
the same spectrograph. This is often used if the star of interest is so dim that
it takes several observations to reach a sufficient high S/N. Another case is
when spectra have been obtained from the archive. The scientific goal can be
very different, e.g. obtaining radial velocities (RV) where a much lower S/N is
needed, however here numerous spectra are needed. It is common to search for
exoplanets with the RV detection method, where multiple spectra are obtained,
and the stellar parameters are then obtained after co-adding the multiple
spectra. It is of course important to put the spectra on a common RV, usually at
\SI{0}{km/s}.

Spectrographs work in different wavelength regions. The most used region is the
optical part of the spectrum, which is also ideal for studying FGK stars with
relative low line blending and low telluric contamination. In the recent years
there has been an increase of NIR spectrographs. These will mainly be used to
study the distant Universe, i.e. at high red-shifts, and cool objects in our own
galaxy. Especially interesting are the M stars which consist of around 70\% of
all stars in our galaxy \citep{Bochanski2010}. These stars are intrinsic dim
because of their low $T_\mathrm{eff}$, ranging from \SIrange{2200}{3500}{K},
hence most of their light emitted will be in the NIR. It is advantageous to
collect as much light as possible from these dim stars, since reaching a similar
S/N in the optical would be more time expensive. The cool M stars have more
molecules in the atmosphere than their hotter counterparts. This can be seen in
the spectrum, where molecular lines greatly depress the continuum (and
me\change{Remove stupid joke!}), making EW measurements difficult. This
continuum depression is much larger in the optical than the NIR, giving another
motivation for studying stars in this wavelength region.
