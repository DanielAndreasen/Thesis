%!TEX root = thesis.tex
\chapter{Conclusions and future work}
\label{cha:future}
\epigraph{Don't cry because it's over, smile because it happened.}{Dr. Seuss}

\section{Conclusions}

This thesis consisted of two separated analysis. 1) The analysis of NIR stellar spectra and the
compilation of a new iron line list for determining stellar atmospheric parameters, and 2) the
analysis of 50 planet-hosting stars and update of SWEET-Cat. These two different analyses (NIR and
optical) were both done using the same method; determination of stellar atmospheric parameters using
high quality spectra by imposing ionization and excitation equilibrium. In a few summarised points,
this thesis brings:

\begin{itemize}
  \item A new NIR line list consisting of 84 \ion{Fe}{I} lines and 5 \ion{Fe}{II} lines. This line
        list is optimised for usage with high quality spectra. The goal is to analyse FGK stars and
        making the bridge towards the M stars. Especially are the dwarf stars of most interest, in
        the context of exoplanets and the search of habitable worlds. ``Know the star, know the
        exoplanet'' is the key.

        The line list has been tested on few NIR spectra, mostly due to the lack of spectra and the
        difficulties it is to currently obtain any NIR spectra. This is about to change in the
        near future. The tests performed have been increasingly successful, and progress is evident.
  \item To analyse the data a new tool was created, \code{FASMA}. This tool provides many different
        options, can be used with different atmosphere models, is available to the community as a
        web application, and provides three different drivers for analysis of spectra: EW
        measurements, deriving parameters, and deriving abundances of a range of elements. A fourth
        driver is under construction which is to derive parameters using the synthesis method
        \citep{Tsantaki2017}.
  \item The analysis of 50 planet-host stars to increase the number of homogeneously analysed stars
        was also the first appearance and official usage of \code{FASMA}. This analysis increased
        the completeness to 85\% for stars brighter than 10 V magnitude. It is time consuming to
        obtain high quality spectra for stars fainter than this magnitude, however the completeness
        is still at 77\% for stars brighter than 12 V magnitude. Out of the 50 planet host stars
        analysed, eight changed either the radius or mass by more than 25\%. These eight systems
        were carefully analysed, and the updated planetary parameters (radius, mass, and density)
        were derived when possible.
\end{itemize}


\section{Future work}

While the tests with the NIR line list has been successful, there will still be room for
improvements. The same is the case for the \code{FASMA} and SWEET-Cat. Here are some of the main
points for (possible) future work.

\begin{itemize}
  \item One of the most difficult parameters to derive from high quality NIR spectra is the surface
        gravity. This has also been known for the optical part of the spectrum
        \citep[see e.g.][]{Mortier2014}. However, with the iron line list presented here, the
        problem is more severe than the optical. This is mainly due to the lack of \ion{Fe}{II}
        lines. There might be different ways to solve this problem.
        \begin{enumerate}
          \item Set the $\log g$ to a fixed value determined from other sources (this can e.g. be
                asteroseismology).
          \item Obtain more \ion{Fe}{II} lines in the NIR. With the increased experience, it might
                prove valuable to have a second look at the \ion{Fe}{II} lines, thereby increase the
                numbers.
          \item Explorer another element than iron. While iron is both abundant in numbers and have
                a lot of transitions, there might exists a better element for this.
          \item Find other pressure sensitive features in the NIR to derive $\log g$. In this case
                it will then be combined with the first point raised, and iteratively obtaining
                $T_\mathrm{eff}$, $[\ion{Fe}/\ion{H}]$, and $\xi_\mathrm{micro}$, before then
                obtaining $\log g$.
        \end{enumerate}
        This points requires some consideration before anything is done, as they might be time
        consuming or not work as intended. As an example, it might not always be a advisable to
        set the surface gravity (or any other parameters) to a fixed value.
  \item The model atmosphere used is reliable, however newer versions are available. It would be
        interesting to use more recent versions of the ATLAS9 atmosphere models, and to explorer
        other atmosphere models altogether, as the PHOENIX library.
  \item While \code{FASMA} is in a state where it work well and is stable, there are still ways to
        improve it. One way is to use machine learning to get a better guess of initial parameters.
        This work has already been initiated, and details on this can be seen in \cref{cha:fasmaML},
        however the work is not yet connected to \code{FASMA}. An additional plugin to \code{FASMA}
        could be to correct line abundances for non-LTE effects using a pre-calculated grid of these
        corrections. The main part of this has already been done in another project, but is not yet
        implemented into \code{FASMA}.
  \item After obtaining stellar atmospheric parameters from a high quality spectrum it is natural to
        explorer the spectrum with another purpose: to obtain abundances for other elements. The
        individual abundances of other elements than iron might be used to explorer the history of
        the star, and can be used in connection with the planet-star correlations. This is something
        that is already explored in detail in the optical, but is yet a relative new approach in the
        NIR from high quality spectra.
  \item One of the main limitations during this thesis have been the limited access to data. This
        lead to the analysis of synthetic spectra. It will be very interesting to explorer real data
        in the future. Particular interesting is the CARMENES library which is under constructions.
        This library will contain a high S/N spectrum of all the stars observed by CARMENES. This
        can be used to accurately explorer different part of the parameter space and find weak
        points in the methodology presented here in this thesis.
  \item This was already discussed previously, however, it will be interesting to update SWEET-Cat.
        These updates can include more columns such as abundances of the planet host, modelled mass,
        radius, and age, rotational velocity, etc. The update to SWEET-Cat can also be for the web
        interface. A wish is to update this and include easy-plotting capabilities for users to
        quickly explorer planet-star correlations.
\end{itemize}
