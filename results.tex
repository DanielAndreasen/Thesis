%!TEX root = thesis.tex
\chapter{Results for FGK stars}
\label{cha:results}
\epigraph{Don't cry because it's over, smile because it happened.}{Dr. Seuss}

It is time to apply the theory and methodology on some data. In this chapter results from studies
during the thesis will be presented. There is the analysis of three stars which lead to two papers,
HD20010 \citep{Andreasen2016} and Arcturus \& 10 Leo \citep{Andreasen2017b}. Additionally, there was
a analysis of how a cut in EP might affect the final parameters of a star. This will be discussed in
\sref{sec:EPcut}, and last the analysis of a range of synthetic spectra from the PHOENIX spectral
library \citep{Husser2013}, with $T_\mathrm{eff}$ lower than analysed in previous stars with the
methodology described above in \sref{sec:parameters}.


\section{HD20010}
\label{sec:HD20010}

HD 20010 is a star that has been analysed twice with the methodology described here in this thesis.
First time this star was analysed was in \citet{Andreasen2016} when the NIR line list was published.
Later this line list was revised leading to a removal of several lines. This resulted in the second
analysis in \citet{Andreasen2017b}; both described below.

\subsection{First analysis}

To test our new line list we search for a well-studied solar-type star. The spectrum for such a
target needs to be available in the NIR at both high resolution and high S/N. An ideal place to look
for such a star is the CRIRES-POP database (Lebzelter et al. 2012). Here, the best target for
testing is HD 20010, an F8 subgiant star. This star has been part of many surveys and is therefore
well studied. Different parameters from the literature are listed in Table 3. The data available at
CRIRES-POP are in the raw format and pipeline reduced, while three small pieces of the spectra are
fully reduced on the web page 3 . The data is in the standard CRIRES format with each fits file
including four binary tables with the data from the four detectors. In the future, the final reduced
data will be presented by the CRIRES-POP team. In contrast to the pipeline reduced data, this will
be of higher quality, a better wavelength calibration, and telluric correction. We measured the EWs
of the pipeline reduced spectra, and where there was an overlap with the fully reduced spectrum, we
measured both as a consistency check. The measured EWs from the fully reduced spectra were
consistent with the measured EWs from the pipeline reduced spectra. As mentioned above, we use the
Y, J, H, and K-bands which are all available for this star. The spectra come in pieces of 50 Å to
120 Å. These pieces over- lap each other, and we were able to measure the EW for a single line up to
five times. Unfortunately, wavelength calibration is a difficult task for CRIRES owing to the rather
small spectral regions measured on each detector. Each calibration was performed separately for each
detector and required the avail- ability of a sufficient number of calibration lines in the
respective spectral region. This was not always the case and a default linear solution was applied.
A pipeline reduced spectrum shows up as a stretched spectrum if the wavelength calibration is poor
compared to a model spectrum or a solar spectrum, for example. The wavelength calibration does not
have any effect on the signal-to-noise ratio, which is generally high for the spectrum of HD 20010.
The signal-to-noise varies between 200 and 400 for different chunks. The pipeline reduced spectra
for HD 20010 contains tellurics and the wavelength is shifted in radial velocity. All of these
factors make the line identification very difficult, and so we developed a program to properly
identify the lines, which does the following:

\begin{enumerate}
  \item Plotting the observed spectrum
  \item Overplotting a model spectrum. In this particular case the solar spectrum was used since the
        atmospheric parameters are close enough, so the sun was able to serve as a model
  \item Overplotting a telluric spectrum from the TAPAS web page \citep{Bertaux2014}
  \item Overplotting vertical lines at the location of lines in the list
  \item Calculating the cross-correlation function (CCF) for the telluric spectrum with respect to
        the observed spectrum, locating the maximum value by a Gaussian fit, and using this to shift
        the telluric spectrum with the found RV;
  \item Performing the same as step 5, but for the model
  \item Shifting the lines with the same RV as found for the model/solar spectrum.
\end{enumerate}

The final plot shows the shifted spectra, and the CCFs at the sides. An example of the software in
use is shown in Fig. 6. The two RVs are part of the title of the plot. Once the lines were
identified, the EWs were measured with the splot routine in Image Reduction and Analysis Facility
(IRAF). The reason not to choose ARES for this task was to visually confirm the identification of
the line given the relative poor wavelength calibration. We were able to measure 249 \ion{Fe}{I}
lines and 5 \ion{Fe}{II} lines compared to 344 \ion{Fe}{I} lines and 13 \ion{Fe}{II} lines for the
Sun over the whole NIR spectral region. Whenever we had more than one measurement of a line, the
average was used for the final EW. We derived the stellar parameters using the standard procedure
(see Sect. 2.6) as done for the Sun. Given the relatively low quality of the spectrum of HD 20010
(see below) and be- cause it is not corrected for telluric contamination, we made a cut in EW at
\SI{5}{m\AA{}} in order to remove the lines which are most affected by contamination from either
telluric or other line blends. Additionally, we made a cut in EP at \SI{5.5}{eV} because the
\ion{Fe}{I} and \ion{Fe}{II} lines usually used for stellar parameter determination in the optical
regime are also limited to similar values \citep[see e.g][]{Sousa2008a}. Higher excitation potential
lines are also more likely to be affected by non-LTE effects. When deriving the atmospheric
parameters, we made a $3\sigma$ outlier removal in the abundance iteratively until there were no
more outliers present. Since we could only measure 5 \ion{Fe}{II} lines, for comparison we also
decided to derive parameters using the same method, but we fixed the surface gravity to the
reference value. The resulting atmospheric parameters and iron abundances are presented in Table 4.
The effective temperature, surface gravity, and metallicity agree within the errors with the
literature values. Similar parameters are obtained by fixing log g to the average literature value
or by leaving it free.

The errors on the atmospheric parameters for HD 20010 are much higher than what is achievable with
other measurements in the literature, as presented above in Table 3. In order to explain these
errors, we calculated the abundances for all lines which have at least two measurements of the EW.
We then calculated the abundances for the highest measured EW and the lowest. The differences in
abundances are presented in Fig. 7. The very large differences (more than 0.1 dex) translate to the
high errors in the parameters.

The source of the large errors on the parameters can be seen more clearly where abundances are
compared to excitation potential or abundances versus reduced EW. Here the dispersion on the
abundances can be seen clearly, as shown in Fig. 8.

This test strongly suggest that errors in the EWs, likely due to the poor quality of this spectrum,
are responsible for the relatively large error bars in the derived stellar parameters. Systematic
errors (e.g. due to a possible non-optimal reduction of the spectrum) may be the reason for these
large error bars. As the CRIRES-POP team continue their great efforts in reducing the optimal
spectra, it will be interesting to re-visit this star once the entire spectrum has been fully
reduced.

\subsection{Second analysis}

As a first step we revisit HD 20010 for which we derived atmospheric stellar
parameters in Paper I using the newly revised line list presented in this paper.
The results are shown in Table \ref{tab:results} along with the {\bf results for
the two other stars analysed in this work}. We see better agreement with the
average literature values adopted (especially $[\ion{Fe}/\ion{H}]$ and $\log g$),
and smaller errors with the updated results. This suggests that the new line
list is more reliable.

\begin{table}[htb!]
    \caption{Results for the three stars with first set of parameters are the
             literature values as presented in Table.~\ref{tab:stars}, second
             set of parameters are results with $\log g$ set to the same value
             during the minimization procedure as found in the literature
             (fixed), and last set of parameters are with all parameters free
             during the minimization procedure.}
    \label{tab:results}
    \centering
    \begin{tabular}{llll}
      \hline\hline
                                    & HD 20010          &  10 Leo           &  Arcturus        \\
      \hline
        Literature                  &                   &                   &                  \\
        $T_\mathrm{eff}$ (lit.)     & $6152 \pm  95$    &  $4741 \pm  60$   & $4300 \pm 110$   \\
        $\log g$ (lit.)             & $3.96 \pm 0.19$   &  $2.76 \pm 0.17$  & $1.60 \pm 0.29$  \\
        $[\ion{Fe}/\ion{H}]$ (lit.) & $-0.27 \pm 0.06$  &  $-0.03 \pm 0.02$ & $-0.54 \pm 0.11$ \\
        $\xi_\mathrm{micro}$ (lit.) & $1.17 \pm 0.24$   &  $1.45 \pm 0.08$  & $1.93 \pm 0.13$  \\
      \hline
        $\log g$ fixed              &                   &                   &                  \\
        $T_\mathrm{eff}$            & $6161 \pm 164$    &  $4761 \pm 118$   & $4357 \pm  74$   \\
        $\log g$                    & 3.96 (fixed)      &  2.76 (fixed)     & 1.60 (fixed)     \\
        $[\ion{Fe}/\ion{H}]$        & $-0.18 \pm 0.11$  &  $ 0.01 \pm 0.07$ & $-0.55 \pm 0.04$ \\
        $\xi_\mathrm{micro}$        & $1.72 \pm 0.44$   &  $1.25 \pm 0.11$  & $1.55 \pm 0.10$  \\
      \hline
        All free                    &                   &                   &                  \\
        $T_\mathrm{eff}$            & $6162 \pm 184$    &  $4805 \pm  98$   & $4439 \pm  62$   \\
        $\log g$                    & $4.08 \pm 0.77$   &  $2.42 \pm 0.61$  & $1.20 \pm 0.20$  \\
        $[\ion{Fe}/\ion{H}]$        & $-0.18 \pm 0.11$  &  $-0.01 \pm 0.07$ & $-0.58 \pm 0.06$ \\
        $\xi_\mathrm{micro}$        & $1.59 \pm 0.49$   &  $1.23 \pm 0.10$  & $1.55 \pm 0.10$  \\
        \hline\hline
    \end{tabular}
\end{table}

{\bf The parameters for the three stars, omitting the Sun since the derived
parameters are trivial with a calibrated line list\footnote{The solar parameters
used were: $T_\mathrm{eff}=\SI{5777}{K}$, $\log g=4.44\,$dex,
$[\ion{Fe}/\ion{H}]=0.00\,$dex, and $\xi_\mathrm{micro}=\SI{1.00}{km/s}$.}, are
presented in Fig.~\ref{fig:parameters}. We show the literature values (blue),
derived parameters with $\log g$ fixed to the literature value (green), and
derived parameters when $\log g$ is free during the minimization procedure (red
points).}

\begin{figure}[htpb!]
    \centering
    \includegraphics[width=1.0\linewidth]{figures/parameters.pdf}
    \caption{Parameters for Arcturus, 10 Leo, and HD 20010 (revisited in this
             paper). The blue points show the literature values from the PASTEL
             database as discussed in the text. The green points are the
             derived values with $\log g$ fixed to the literature value, and the
             red points show the derived parameters when $\log g$ is also
             derived.}
    \label{fig:parameters}
\end{figure}



\section{Arcturus}
\label{sec:arcturus}

Arcturus is one of the brightest stars on the night sky with a V magnitude of
-0.05 \citep{Ducati2002}. Hence it is a prime target for testing with the
numerous measurements of the atmospheric parameters as mentioned above.

The atlas consists of both a summer observation set and a winter observation
set. The two data sets have been obtained in order to minimise the effect of
tellurics at different spectral regions. A comparison between the two sets of
measured EWs - both the manual measurements using IRAF and the automatic
measurements using ARES - are shown in Fig.~\ref{fig:EWcomp}. The automatic EW
measurements for the summer set and winter set show excellent agreement {\bf
with a dispersion of 7m\AA{}}. This means that the two data sets are very
similar, thus we decided to only manually measure the EWs for one set (summer).
We did, however, measure a few lines from the winter data set to verify the
agreement. {\bf Since the EWs are very similar we chose to only derive
parameters of the summer set with EWs measured with ARES.} Parameters were
derived with and without $\log g$ set to a fixed value (1.60\,dex, the average
literature value adopted). The derivation of the parameters followed the
procedure presented in Paper I, although we used the minimization routine of
FASMA \citep{Andreasen2017a}. After we reached convergence using all the iron
lines we were able to measure, one outlier above $3\sigma$ in abundance were
removed, and the minimization routine was restarted. This process was done
iteratively until there were no more outliers. The final results are presented
in Table \ref{tab:results} together with mean parameters from the literature.

We generally see good agreement between the derived parameters and the average
values from the literature adopted (see Table \ref{tab:results}). The only
parameter being difficult to measure is the surface gravity due to the low
number of \ion{Fe}{II} lines in the NIR. It is very important to derive the
metallicity accurately, and we report consistent results overall.


\section{10 Leo}
\label{sec:10Leo}

The approach for determining the atmospheric stellar parameters for 10 Leo is
identical to Arcturus. We use ARES on each band (YJ, H, and K-band) separately.
For the small gaps in the spectrum, we simply set the flux to 1, since the
spectrum is already normalised. This will also prevent ARES to identify and
measure any lines in these regions. The EWs from the three regions are combined
to one final line list used for the determination of the parameters. {\bf The
final results can be seen in Fig.~\ref{fig:parameters} and
Table~\ref{tab:results}.}

Generally the derived parameters are in excellent agreement with the literature
values listed here. {\bf For $T_\mathrm{eff}$ we were \SI{64}{K} off with $\log
g$ set as a free parameter, well within the errors. The only parameter that show
a discrepancy compared to the literature value is $\xi_\mathrm{micro}$ with a
difference of \SI{0.22}{km/s}, which is at the limit of the errors reported. We
note that this parameter is not reported in the PASTEL database, and this was a
derived parameter from a empirical relation.} We were able to derive good $\log
g$ values, although with larger errors compared to the results from the
literature.


\section{Synthetic cool stars}
\label{sec:synthetic_spectra}



\section{Parameter dependence on EP cut}
\label{sec:EPcut}

It is common practise, as in this case, to make a cut in EP for a line list when deriving
parameters. This was suggested in \citet{Andreasen2016} \citep[later done in][]{Andreasen2017b} for
the NIR line list used here. This cut was made at \SI{5.5}{eV}, inspired by a similar cut in the
optical \citep{Sousa2008a}, including only lines below this limit. The lines with higher EP might
also be affected by non-LTE effects which is not treated in this methodology.

As the parameters are dependent on this cut in EP\unfinished{Make a figure that shows this}, it
seemed interesting to divide a line list in two with upper and lower EPs and analyse those
separately. Before going into that analysis it is important to note, that the parameters should not
depend on any cut in EP if the theory is right, the radiative transfer code is working properly, and
the atmospheric models are correct. Lines with different EP are likely to be formed in different
layers of the atmosphere as discussed in \sref{sec:stellar_atmosphere}, however this should not
effect the final derived abundance, which is the problem here.
