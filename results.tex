%!TEX root = thesis.tex
\chapter{Results for FGK stars}
\label{cha:results}
\epigraph{Don't cry because it's over, smile because it happened.}{Dr. Seuss}

It is time to apply the theory and methodology on some data. In this chapter results from studies
during the thesis will be presented. There is the analysis of three stars which lead to two papers,
HD20010 \citep{Andreasen2016} and Arcturus \& 10 Leo \citep{Andreasen2017b}. A small summary of the
stars and their characteristics can be found in \tref{tab:stars}. More details will be provided in
the individual sections below for each star. Additionally, there was an analysis of how a cut in
excitation potential might affect the final parameters of a star. This will be discussed in
\sref{sec:EPcut}, and last the analysis of a range of synthetic spectra from the PHOENIX spectral
library \citep{Husser2013}, with $T_\mathrm{eff}$ lower than analysed in previous stars with the
methodology described above in \sref{sec:parameters}. However, before diving into the results of the
analysis it is important to describe how the NIR line list were compiled and why and how it was
later refined.

\begin{table*}[htb!]
    \caption{Summary of the four stars used in this thesis. The stellar parameters
             are from the PASTEL catalogue \citep{Soubiran2016} (see text for
             details), except the parameters for the Sun.}
    \label{tab:stars}
    \centering
    \begin{tabular}{lllllll}
      \hline\hline
        Star        & Spectrographs  & Resolution  & $T_\mathrm{eff}$ (K) &  $\log g$ (dex)  &   $\xi_\mathrm{micro}$ (km/s)   & [Fe/H] (dex)      \\
      \hline
        Sun         & FTS            & 600\,000    & $5777$               &  $4.44$          &    $1.00$                       & $ 0.00$          \\
        Arcturus    & FTS            & 100\,000    & $4300 \pm 111$       &  $1.60 \pm 0.29$ &    $1.93 \pm 0.17$              & $-0.54 \pm 0.11$ \\
        HD 20010    & CRIRES         & 100\,000    & $6152 \pm  95$       &  $3.96 \pm 0.11$ &    $1.17 \pm 0.24$              & $-0.27 \pm 0.06$ \\
        10 Leo      & CRIRES         & 100\,000    & $4742 \pm  61$       &  $2.76 \pm 0.17$ &    $1.45 \pm 0.08$              & $-0.03 \pm 0.02$ \\
      \hline
    \end{tabular}
\end{table*}

\section{The creation of a NIR line list}
\label{sec:creation_line_list}

As discussed extensively in \cref{cha:theory} and \cref{cha:method}, an atomic line list is needed
for employing the method described here. This line list is made by neutral and ionized iron lines in
the NIR. This has already been mentioned in \cref{cha:theory}. Here the process will be explained in
greater detail below for the two versions presented in \citet{Andreasen2016} and
\citet{Andreasen2017a}, respectively.


\subsection{First version}
\label{sec:linelist_first}

The first version of a NIR iron line list for determining stellar atmospheric parameters of high
resolution and high S/N spectra were presented in \citet{Andreasen2016}. Other NIR line lists exists
such as the one from \citet{Onehag2012,Lindgren2016} for utilising the synthetic method described
above\unfinished{Add more line lists from the literature.}. Here all iron transitions between
\SIrange{10000}{25000}{\angstrom}\footnote{That is equivalent to the YJHK bands.} were downloaded
from the VALD database \citep{VALD1,VALD2}. This only includes \ion{Fe}{I} and \ion{Fe}{II}
transitions. In total were \num{50198} \ion{Fe}{I} and \num{28339} \ion{Fe}{II} lines respectively
acquired.

The EW of all lines were measured in the solar atlas by \citet{Hinkle1995}. The EWs were measured
using \ARES due to the large amount of lines and to be as consistent as possible in the
measurements. Since \ARES expect a 1D spectrum with equidistant wavelength step, the solar spectrum
was interpolated onto a regular wavelength grid of \SI{0.01}{\angstrom}. This did not change the
appearance, and hence not the EW, of the final spectrum. This wavelength step is equivalent to a
spectral resolution of \num{1500000} at \SI{15000}{\angstrom}.

The EWs were measured by fitting Gaussian profiles to the spectral lines. For a given line, \ARES
output the central wavelength of the line (provided in the line list), the FWHM maximum of the
fitted line, the number of lines fitted for the final result (in case of blending), the depth of the
line, the EW of the line, and the Gaussian coefficients of the line. In the latest version of \ARES,
the error of the EW is also provided \citep{Sousa2015a}.

More lines were discarded according to the following four criteria:

\begin{itemize}
  \item Lines with EW lower than \SI{5}{m\angstrom} as these lines can be problematic to see in
        spectra with lower S/N or spectra with many spectral features. Therefore the measurements
        of these lines are less reliable than stronger lines.
  \item Lines with EW higher than \SI{200}{m\angstrom} as these lines are too strong to be fitted
        with a Gaussian profile. These lines might also be saturated and do not contain information
        about the abundance (see \fref{fig:cog}).
  \item Lines where the total number of fitted lines are 10 or higher since they show indications of
        severe blending.
  \item If the fitted central wavelength is more than \SI{0.05}{\angstrom} away from the wavelength
        provided by VALD3 to avoid false identification. This should also remove some blended lines.
\end{itemize}
After this removal the number of \ion{Fe}{I} and \ion{Fe}{II} lines are reduced to \num{6060} and
\num{2735}, respectively.


\subsubsection{Visual removal of lines}

At this point a visual inspection were necessary. All absorption lines from all elements were
downloaded from the VALD3 database in a \SI{3}{\angstrom} window around each of the nearly
\num{9000} iron lines from the previous step. For each small spectral window, all absorption lines
(including the iron line(s)) were plotted on top of the solar spectrum. An iron line were discarded
if another line had the same central wavelength and/or the absorption line were severely blended.
Most of the discarded iron lines had high EP, which are weak compared to the low EP lines.
Therefore, it is fair to assume that many of these iron lines were falsely identified as another
stronger line from another element. After the visual removal the number of \ion{Fe}{I} and
\ion{Fe}{II} lines were reduced to a mere 593 and 22, respectively.

For some of the absorption lines it was not clear which element was the cause. These lines were
marked for further investigation with synthesis as described below.

\subsubsection{Synthetic investigation}

For the lines marked above for further investigation an even broader window were used of
\SI{6}{\angstrom}. Once again, all lines were downloaded from the VALD3 database in these spectral
windows. \MOOG were used with the \emph{synth} driver to create a synthetic spectrum using a solar
atmosphere model with $T_\mathrm{eff}=\SI{5777}{K}$, $\log g=4.438$, $A(\mathrm{Fe})=7.47$, and
$\xi_\mathrm{micro}=\SI{1.00}{km/s}$. The iron abundance (7.47) is from \citet{Gonzalez2000}. The
overall metallicity for the solar atmosphere model is $[\ion{M}/\ion{H}]=0.00$ by definition. This
was done for all spectral windows for three different $[\ion{Fe}/\ion{H}]=\{-0.2, 0.0, 0.2\}$.
Before creating a synthetic spectrum all elements which are more than singly ionised were removed
since \MOOG does not allow these. An example of this can be seen in
\fref{fig:synthetic_investigation}. Here the neutral iron line at \SI{15550.439}{\angstrom} were
investigated. The three coloured curves are synthetic spectra with the three different
$[\ion{Fe}/\ion{H}]$. Note that while $[\ion{Fe}/\ion{H}]$ is used as a proxy for the over
metallicity, $[\ion{M}/\ion{H}]$, it is here specific only the iron abundance that is changed. The
upper plot shows the result with the full VALD3 line list, while in the lower plot the iron line
were removed from the line list. In this case it is clear that the iron line is the cause of the
absorption line. The grey curve is the solar atlas for reference.

\begin{figure}[htpb!]
    \centering
    \includegraphics[width=1.0\linewidth]{figures/synthetic_investigation.pdf}
    \caption{The three coloured curves represents different iron abundance, $\{-0.2, 0.0, 0.2\}$ dex
             compared to solar abundance. The grey curve is the solar atlas for reference. In this
             case the iron line at \SI{15550.439}{\angstrom} is investigated. \emph{Upper} plot:
             Synthetic spectra were computed using the full VALD line list in the spectral range for
             the three different iron abundances. \emph{Lower} plot: Same as the upper plot, but
             with the iron line removed from the line list. Since the synthetic spectra shows no
             features at this absorption line anymore, it is a fair assumption to say the iron line
             is the cause of this absorption line.}
    \label{fig:synthetic_investigation}
\end{figure}

If the three synthetic spectra shows variation at the iron line of interest, then it is assumed that
the iron line is the cause for the absorption line. As a simple test, the iron line was also removed
from the line list used to create the synthetic spectra. If the iron line in the synthetic spectra
disappeared, it was a clear signal that this line can be used in the final iron line list ( this can
be seen clearly in the lower plot of \fref{fig:synthetic_investigation}). In some cases two iron
lines had the same or very similar wavelength, and this technique was used to include the right iron
line. In cases where both iron lines causes the absorption line, they were both discarded since they
are blended.

In a few cases two iron lines had the same wavelength and EP but different $\log \mathit{gf}$. If
these both cause an iron line they can be combined into a single line by adding their $\mathit{gf}$
values. After this step, there was 414 \ion{Fe}{I} lines and 12 \ion{Fe}{II} lines, respectively.



\subsubsection{Calibrating the line list: astrophysical $\log \mathit{gf}$ values}

These lines were collected into a single line list in the format required by MOOG \citep{Sneden1973}
and the line abundance were measured by all lines using the solar atmosphere model described above.
If the derived abundance for a single line differs by more than 1.0 dex from the solar iron
abundance, the line would be discarded. The solar iron abundance used is 7.47 as found by
\citet{Gonzalez2000}. \unfinished{Why do we remove above 1.0dex?}

After the removal of the lines mentioned above, the final line list is almost compiled. At this
point the line list has to be calibrated. This is done by changing $\log \mathit{gf}$ so the line
abundance for all iron lines are 7.47 when using the solar atmosphere model mentioned above. As
mentioned in \sref{sec:linelist} there is a anti-correlation between the abundance of a line and
$\log \mathit{gf}$. This means a bisector minimization can be used to locate the $\log \mathit{gf}$
that gives the desired abundance.

This was done for all the iron lines at this stage. It is important to calibrate a line list if the
setup of parameter determination is changed somehow. This includes the type of model atmospheres
used (e.g. ATLAS9 or MARCS), the interpolation code to generate a model atmosphere from the grid, or
the settings of the radiative code, here \MOOG.


\subsubsection{Removal of high dispersion lines}

The line list calibrated above was used to derive parameters for HD 20010 (see \sref{sec:HD20010}),
however the derived parameters showed poor results when compared to the literature values. This lead
to the following test, where highly disperse lines would be removed.

A Gaussian distribution was made for the EW of each line centred on the EW itself,
\begin{align}
  f(x, EW, \sigma) = \frac{1}{\sqrt{2\pi\sigma^2}} e^{-\frac{(x-EW)^2}{2\sigma^2}},
\end{align}
where $\sigma$ is the error on the EW, expressed by \citet{Caryel1988}:
\begin{align}
  \sigma \simeq 1.6 \frac{\sqrt{\Delta\lambda EW}}{S/N},
\end{align}
where $\Delta\lambda=0.1$ and a S/N=50 is considered here, which is much lower than the actual S/N
of the solar atlas. 100 draws of the distribution above were made for each line and derive the
abundance using the solar atmosphere model. The mean absolute deviation (MAD) were calculated for
each line abundance. The MAD for each line as a function of the original measured EW can be seen in
\fref{fig:dispersive_lines}. The trend for the weaker lines is expected since a small absolute
change in the EW results in a large relative change in abundance. However, this does not mean these
lines have a high dispersion. Therefore the disperse lines are found in the de-trended MAD value,
where an exponential function is used for de-trending. A single point above $3\sigma$ is removed
iteratively in the de-trended data. After this process there are 334 \ion{Fe}{I} lines and 13
\ion{Fe}{II} lines.

\begin{figure}[htpb!]
    \centering
    \includegraphics[width=1.0\linewidth]{figures/disperse_lines.pdf}
    \caption{The most disperse lines. \emph{Upper} plot: The MAD versus the original EW. The red
             points are the outliers which were discarded during this process. \emph{Lower} plot:
             Same as above with the MAD being de-trended by the exponential fit as shown in the
             upper panel.}
    \label{fig:dispersive_lines}
\end{figure}



\subsection{Second version}
\label{sec:linelist_second}

As will be seen in \sref{sec:HD20010_first} the line list presented above were used to derive
atmospheric parameters of HD 20010. Even though the first test of the line list was success-full,
it left room for improvements. The errors on the derived parameters were quite high for the spectral
type compared results obtainable with a similar analysis utilising the optical spectrum (see
\sref{sec:HD20010_first} for details on this\unfinished{Make sure to actually discuss this!}).
Additionally, the derived metallicity was 0.10 dex higher than the literature values used.

If all derived parameters except metallicity are reliable (when compared to e.g. a literature
value), then it suggest that the measured EW are either over- or underestimated. However, when using
a line list for the first time like here, it can also suggest problems with the line list itself,
for example a bad calibration. This could have been wrong measurements of the EWs of the calibrator
star, Sun in this case. This combines to several cases:
\begin{itemize}
  \item Correct measurement of EW of calibrator star:
  \begin{itemize}
    \item Systematic lower measurement of EW of target star leads to underestimated $[\ion{M}/\ion{H}]$
    \item Systematic higher measurement of EW of target star leads to overestimated $[\ion{M}/\ion{H}]$
    \item Correct measurement of EW of target star leads to correct $[\ion{M}/\ion{H}]$
  \end{itemize}
  \item Systematic lower measurements of EW of calibrator star:
  \begin{itemize}
    \item Systematic lower measurement of EW of target star leads to underestimated $[\ion{M}/\ion{H}]$
    \item Systematic higher measurement of EW of target star can lead to correct or overestimated $[\ion{M}/\ion{H}]$
    \item Correct measurement of EW of target star leads to underestimated $[\ion{M}/\ion{H}]$
  \end{itemize}
  \item Systematic higher measurements of EW of calibrator star:
  \begin{itemize}
    \item Systematic lower measurement of EW of target star can lead to correct or underestimated $[\ion{M}/\ion{H}]$
    \item Systematic higher measurement of EW of target star leads to overestimated $[\ion{M}/\ion{H}]$
    \item Correct measurement of EW of target star leads to overestimated $[\ion{M}/\ion{H}]$
  \end{itemize}
\end{itemize}
It is important to note, that \emph{correct} has been used here, assuming a perfect setup, that
includes perfect model atmosphere, perfect radiative transfer code, etc. In reality the final
$[\ion{M}/\ion{H}]$ (and the other parameters) measured by different group will occasionally differ
regarding the setup used.

As described in \sref{sec:linelist_first} the EW of the Sun (calibrator star) was measured with
\ARES. A crucial option to set when using \ARES is the \code{rejt} parameter as mentioned in
\sref{sec:measureEW}. This option is used to place the continuum and will this directly affect the
measured EW. At the time of compiling the first version of the line list it seems the \code{rejt}
value used did not reflect the high S/N of the spectrum, thus placing the continuum too low and
thereby underestimate the EW.

In this second version of the line list, the goal is to:
\begin{enumerate}
  \item Make sure the EW measurements are as correct as possible by measuring them by hand
  \item Free of blended lines in cooler stars (K stars in this case)
\end{enumerate}
The second point is a similar exercise which was done in the optical by \citet{Tsantaki2013}, where
blended lines were removed from the larger line list by \citet{Sousa2008a}. This allowed to
determine the atmospheric parameters of stars colder than \SI{5000}{K}. However, the optical
spectrum still suffer for severely blended lines at low $T_\mathrm{eff}$, thus this method does not
work for M stars.

Both of the above steps were done at the same time, by measuring the EWs by hand using \code{IRAF}.
Whenever a line was difficult to reliably measure, i.e. a consistent measurement was not
possible/easy, it was discarded since it was blended. This can be seen in
\fref{fig:linelist_comparison} where the EW measurements from the first version is shown against
this version with the manual measurements. There are some measurements of EW higher than
\SI{150}{m\angstrom} which should have been removed. These lines have not been used during the
analysis, however they were kept since they might appear useful on a later stage.

\begin{figure}[htpb!]
    \centering
    \includegraphics[width=1.0\linewidth]{figures/linelist_comparison.pdf}
    \caption{Comparison of the EW from the first version of the line list, EW$_1$, and the second
             version, EW$_2$. The EWs are generally higher in the second version, with an average
             difference between the two version of \SI{2.1+-11.1}{m\angstrom}. The three horizontal
             lines show the average value and the standard deviation.}
    \label{fig:linelist_comparison}
\end{figure}

While the first version might seem careless since it is necessary with a second version, it is
important to stress the 1) limited access on high quality of NIR spectra, and 2) the gained
experience during the course of the thesis which proved valuable in the refining of the line list
by a second visual inspection of the line list on the solar spectrum.

After the new measurements of the EWs, the line list was re-calibrated according to the procedure
described above. The change in $\log \mathit{gf}$ is \SI{-0.062+-0.157}, i.e. on average the
oscillator strengths are higher in the second version compared to the first version of the line
list.

With the second version there are only 5 \ion{Fe}{II} lines which is a concern since these are
crucial for the derivation of the surface gravity. Therefore, when possible, it is important to
obtain the $\log g$ from other more reliable studies.


\section{HD20010}
\label{sec:HD20010}

HD 20010 is a star that has been analysed twice with the methodology described here in this thesis.
First time this star was analysed was in \citet{Andreasen2016} when the NIR line list was published
(see \sref{sec:linelist_first}). Later this line list was revised leading to a removal of several
lines (see \sref{sec:linelist_second}). This resulted in the second analysis in
\citet{Andreasen2017b}; both described below.

\subsection{First analysis}
\label{sec:HD20010_first}

To test the first version of the NIR line list a well-studied solar-type star is needed. The
spectrum for such a target needs to be available in the NIR at both high resolution and high S/N. An
ideal place to look for such a star is the CRIRES-POP database \citep{Lebzelter2012}. Here, the best
target for testing is HD 20010, an F8 subgiant star. At the time of writing this thesis and
\citet{Andreasen2016} the spectrum of HD 20010 was not fully reduced. Here it is still contaminated
by some telluric lines, and the wavelength solution used is not optimal. Both things are essential,
however a tedious and difficult task to accomplish.

HD 20010 star has been part of many surveys and is therefore well studied. Different parameters from
the literature are listed in \tref{tab:HD20010}.

\begin{table*}[htb!]
    \caption{Selection of literature values for the atmospheric parameters for HD20010. The mean and
             a $3 \sigma$ standard deviation is presented at the end of the table from the
             literature values included, which we use as a reference for our derived parameters.}
    \label{tab:HD20010}
    \centering
    \begin{tabular}{l|llll}
      \hline\hline
     Author                 & $T_\mathrm{eff}$ (K) & $\log g$ (dex)  & $[\ion{Fe}/\ion{H}]$ (dex)  & $\xi_\mathrm{micro}$ (km/s)  \\
      \hline
    \cite{Balachandran1990} & $6152$               & $4.15$          & $-0.27 \pm0.08$             & $1.60$                       \\
    \cite{Favata1997}       & $6000$               & \ldots          & $-0.35 \pm0.07$             & \ldots                       \\
    \cite{Santos2004}       & $6275\pm57$          & $4.40\pm0.37$   & $-0.19 \pm0.06$             & $2.41\pm0.41$                \\
    \cite{Gonzalez2010}     & $6170\pm35$          & $3.93\pm0.02$   & $-0.206\pm0.025$            & $1.70\pm0.09$                \\
    \cite{Ramirez2012}      & $6073\pm78$          & $3.91\pm0.03$   & $-0.30 \pm0.05$             & \ldots                       \\
    \cite{Mortier2013}      & $6114$               & \ldots          & $-0.19$                     & \ldots                       \\
      \hline
      Mean                  & $6131\pm255$         & $4.01\pm0.60$   & $-0.23 \pm0.14$             & $1.90\pm1.08$                \\
      \hline
    \end{tabular}
\end{table*}

The data available at CRIRES-POP are in the raw format and pipeline reduced, while three small
pieces of the spectra are fully reduced on the web
page\footnote{\url{http://www.univie.ac.at/crirespop/data.htm}}. The data is in the standard CRIRES
format with each fits file including four binary tables with the data from the four detectors. In
the future, the final reduced data will be presented by the CRIRES-POP team. In contrast to the
pipeline reduced data, this will be of higher quality, a better wavelength calibration, and telluric
correction. The EWs were measured of the pipeline reduced spectra, and where there was an overlap
with the fully reduced spectrum, we measured both as a consistency check. The measured EWs from the
fully reduced spectra were consistent with the measured EWs from the pipeline reduced spectra. As
mentioned above, the Y, J, H, and K-bands, which are all available for this star, were used in this
analysis. The spectra come in pieces of \SIrange{50}{120}{\angstrom}. These pieces overlap each
other, and up to five EW measurements were made for some lines. Unfortunately, wavelength
calibration is a difficult task for CRIRES owing to the rather small spectral regions measured on
each detector. Each calibration was performed separately for each detector and required the
availability of a sufficient number of calibration lines in the respective spectral region. This was
not always the case and a default linear solution was applied. A pipeline reduced spectrum shows up
as a stretched spectrum if the wavelength calibration is poor compared to a model spectrum or a
solar spectrum, for example. The wavelength calibration does not have any effect on the
signal-to-noise ratio, which is generally high for the spectrum of HD 20010. The signal-to-noise
varies between 200 and 400 for different chunks. However, the stretched spectra will most likely
have an effect on the measured EWs. The pipeline reduced spectra for HD 20010 contains tellurics and
the wavelength is shifted in radial velocity. All of these factors make the line identification very
difficult, so a program was developed\footnote{The program (and other small scripts) can be found
here\url{https://github.com/DanielAndreasen/astro_scripts}} to properly identify the lines, which
does the following:

\begin{enumerate}
  \item Plotting the observed spectrum
  \item Overplotting a model spectrum. In this particular case the solar spectrum was used since the
        atmospheric parameters are close enough, so the sun was able to serve as a model
  \item Overplotting a telluric spectrum from the TAPAS web page \citep{Bertaux2014}
  \item Overplotting vertical lines at the location of lines in the list
  \item Calculating the cross-correlation function (CCF) for the telluric spectrum with respect to
        the observed spectrum, locating the maximum value by a Gaussian fit, and using this to shift
        the telluric spectrum with the found RV
  \item Performing the same as step 5, but for the model spectrum
  \item Shifting the lines with the same RV as found for the model/solar spectrum
\end{enumerate}

The final plot shows the shifted spectra, and the CCFs at the sides. An example of the software in
use is shown in \unfinished{Make a plot of this}. The two radial velocities (for the telluric and
model spectrum) are part of the title of the plot. Once the lines were identified, the EWs were
measured with the \code{splot} routine in \emph{Image Reduction and Analysis Facility}
(\code{IRAF}). The reason not to choose \ARES for this task was to visually confirm the
identification of the line given the relative poor wavelength calibration from the automatic
pipeline. We were able to measure 249 \ion{Fe}{I} lines and 5 \ion{Fe}{II} lines compared to 344
\ion{Fe}{I} lines and 13 \ion{Fe}{II} lines for the Sun over the whole NIR spectral region in the
first version of the line list. Whenever there were more than one EW measurement of a line, the
average was used for the final EW. The stellar atmospheric parameters were derived using the
standard procedure (see \sref{sec:parameters}). Lines below \SI{5}{m\angstrom} were removed in order
to remove the lines which are most affected by contamination from either telluric or other line
blends. Additionally, a cut in EP at \SI{5.5}{eV} was made since the \ion{Fe}{I} and \ion{Fe}{II}
lines usually used for stellar parameter determination in the optical regime are also limited to
similar values \citep[see e.g][]{Sousa2008a}. Higher excitation potential lines are also more likely
to be affected by non-LTE effects.

The parameters were derived with one outlier in abundance removed iteratively (after a completed
minimization) until no outliers were present. Since we could only measure 5 \ion{Fe}{II} lines, the
parameter were also derived with $\log g=4.01$ dex fixed at the reference mean value (see
\tref{tab:HD20010}). The resulting atmospheric parameters and iron abundances are presented in
\tref{tab:HD20010_results}. The effective temperature, surface gravity, and metallicity agree within
the errors with the literature values. Similar parameters are obtained by fixing $\log g$ to the
average literature value or by leaving it free.

\begin{table*}[htb!]
    \caption{The derived parameters for HD20010 with and without fixed surface gravity.}
    \label{tab:HD20010_results}
    \centering
    \begin{tabular}{lllll}
      \hline\hline
                     & $T_\mathrm{eff}$ (K) &  $\log g$ (dex)  &   $\xi_\mathrm{micro}$ (km/s)  & [Fe/H] (dex)      \\
      \hline
        Literature   & $6131 \pm 255$       &  $4.01 \pm 0.60$ &    $1.90 \pm 1.08$              & $-0.23 \pm 0.14$ \\
      \hline
                     & $6116 \pm 224$       &  $4.21 \pm 0.58$ &    $2.45 \pm 0.45$              & $-0.14 \pm 0.14$ \\
                     & $6144 \pm 212$       &   4.01 (fixed)   &    $2.66 \pm 0.42$              & $-0.13 \pm 0.29$ \\
      \hline
    \end{tabular}
\end{table*}


The errors on the atmospheric parameters for HD 20010 are much higher than what is achievable with
other measurements in the literature, as presented above in \tref{tab:HD20010}. In order to explain
these errors, we calculated the abundances for all lines which have at least two measurements of the
EW. We then calculated the abundances for the highest measured EW and the lowest. The differences in
abundances are presented in \fref{fig:HD20010abundance}. The very large differences (more than 0.1
dex) translate to the high errors in the parameters.

\begin{figure}[htpb!]
    \centering
    \includegraphics[width=0.8\linewidth]{figures/HD20010abundance_error.pdf}
    \caption{Difference in abundance for HD 20010 when multiple measurements of EW were obtained.
             The differences are between the lowest and highest measured EW in case of multiple
             measurements. This is shown against the wavelength (\emph{upper panel}) and in a
             histogram (\emph{lower panel}).}
    \label{fig:HD20010abundance}
\end{figure}


This test strongly suggest that errors in the EWs, likely due to the telluric contamination and
non-optimal reduction of the spectrum (poor default wavelength calibration), are responsible for the
relatively large error in the derived stellar parameters. After the first analysis of HD 20010, the
CRIRES-POP team published a fully reduced spectrum of 10 Leo \citep{Nicholls2017} with telluric
correction and an optimal wavelength solution. The results obtained from this star (see
\sref{sec:10Leo}) are very promising for the method used here, and it encourage a complete re-visit
of HD 20010 once the reduction is optimal.


\subsection{Second analysis}
\label{sec:HD20010_second}

During the analysis of Arcturus (see \sref{sec:arcturus}) and 10 Leo (see \sref{sec:10Leo}) with the
refined line list presented in \sref{sec:linelist_second} it was a simple task to apply the updated
line list on HD 20010 a second time as a test whether it would perform similar, worse or better.
This is compared to results obtained from the literature.

Therefore HD 20010 was revisited, and the atmospheric stellar parameters was derived. The measured
EWs from the results above in \sref{sec:HD20010_first} was kept, however the lines which did not
make the cut in the second version of the line list was removed. Moreover, the $\log \mathit{gf}$
values was updated since these were re-calibrated.

\FASMA was used to obtain the results which are shown in \tref{tab:results} along with the results
for Arcturus and 10 Leo (details on their parameters will be found below). The agreement with the
adopted average literature values are better for HD 20010 compared to the results from above in
\sref{sec:HD20010_first} (especially $[\ion{Fe}/\ion{H}]$ and $\log g$), and smaller errors with the
updated results. This first test of the line list already shows promising results. The literature
values are slightly different here than compared to the first analysis of this star. This is solely
because other references were used, the PASTEL catalogue \citep{Soubiran2016}. However, this does
not change the improvement seen.


\begin{table}[htb!]
    \caption{Results for the three stars where first set of parameters are the literature values as
             presented in \tref{tab:stars}, second set of parameters are results with $\log g$ set
             to the same value during the minimization procedure as found in the literature (fixed),
             and last set of parameters are with all parameters free during the minimization
             procedure.}
    \label{tab:results}
    \centering
    \begin{tabular}{llll}
      \hline\hline
                                    & HD 20010          &  10 Leo           &  Arcturus        \\
      \hline
        Literature                  &                   &                   &                  \\
        $T_\mathrm{eff}$ (lit.)     & $6152 \pm  95  $  &  $4741 \pm  60  $ & $4300 \pm 110  $ \\
        $\log g$ (lit.)             & $3.96 \pm 0.19 $  &  $2.76 \pm 0.17 $ & $1.60 \pm 0.29 $ \\
        $[\ion{Fe}/\ion{H}]$ (lit.) & $-0.27 \pm 0.06$  &  $-0.03 \pm 0.02$ & $-0.54 \pm 0.11$ \\
        $\xi_\mathrm{micro}$ (lit.) & $1.17 \pm 0.24 $  &  $1.45 \pm 0.08 $ & $1.93 \pm 0.13 $ \\
      \hline
        $\log g$ fixed              &                   &                   &                  \\
        $T_\mathrm{eff}$            & $6161 \pm 164  $  &  $4761 \pm 118  $ & $4357 \pm  74  $ \\
        $\log g$                    & 3.96 (fixed)      &  2.76 (fixed)     & 1.60 (fixed)     \\
        $[\ion{Fe}/\ion{H}]$        & $-0.18 \pm 0.11$  &  $ 0.01 \pm 0.07$ & $-0.55 \pm 0.04$ \\
        $\xi_\mathrm{micro}$        & $1.72 \pm 0.44 $  &  $1.25 \pm 0.11 $ & $1.55 \pm 0.10 $ \\
      \hline
        All free                    &                   &                   &                  \\
        $T_\mathrm{eff}$            & $6162 \pm 184  $  &  $4805 \pm  98  $ & $4439 \pm  62  $ \\
        $\log g$                    & $4.08 \pm 0.77 $  &  $2.42 \pm 0.61 $ & $1.20 \pm 0.20 $ \\
        $[\ion{Fe}/\ion{H}]$        & $-0.18 \pm 0.11$  &  $-0.01 \pm 0.07$ & $-0.58 \pm 0.06$ \\
        $\xi_\mathrm{micro}$        & $1.59 \pm 0.49 $  &  $1.23 \pm 0.10 $ & $1.55 \pm 0.10 $ \\
        \hline\hline
    \end{tabular}
\end{table}




\section{Arcturus}
\label{sec:arcturus}

Arcturus is one of the brightest stars on the Northern hemisphere with a V magnitude of -0.05
\citep{Ducati2002}, and is well studied \citep[see e.g.][to mention just a
few]{Griffin1967,McWilliam1990,Ramirez2013}, and a benchmark star in current spectroscopic surveys
such as Gaia-ESO \citep{Jofre2014,Smiljanic2014}. Hence it is a prime target for testing with the
numerous measurements of the atmospheric parameters.

The atlas of Arcturus, acquired at Kitt Peak National Observatory using the FTS spectrograph at the
Mayall telescope by \citet{Hinkle1995a}, covers the spectral range of interest (YJHK bands). Strong
telluric features were identified with a spectrum from the TAPAS web page \citep{Bertaux2014} which
was useful during the line identification. The atlas also comes with a telluric standard and the
ratio of the two spectra in order to correct for the tellurics. The telluric spectrum from TAPAS is
only used for telluric line identification. Both the telluric corrected and non-corrected spectra
was used during the analysis, however the focus was on the non-corrected spectrum since it is simple
to identify the telluric lines in this spectrum.

The atlas consists of both a summer observation set and a winter observation set. The two data sets
have been obtained in order to minimise the effect of tellurics at different spectral regions. A
comparison between the two sets of measured EWs - both the manual measurements using \code{IRAF} and
the automatic measurements using \ARES - are shown in \fref{fig:EWcomp}. The automatic EW
measurements for the summer set and winter set show excellent agreement with a dispersion of
\SI{7}{m\angstrom}. This means that the two data sets are very similar, thus it was decided to only
manually measure the EWs for one set (summer). A few lines from the winter data set was measured to
verify the agreement. Since the EWs are very similar the parameters are only derived from from the
summer set with the EWs measured by \ARES. Parameters were derived with and without $\log g$ set to
a fixed value (1.60\,dex, the average literature value adopted). The derivation of the parameters
followed the same procedure as described above for HD 20010 using \FASMA. Again outliers in
abundance were removed one at the time iteratively until there were no more outliers. The final
results are presented in \tref{tab:results} together with mean parameters from the literature.

\begin{figure}[htpb!]
    \centering
    \includegraphics[width=1.0\linewidth]{figures/EWcomp.pdf}
    \caption{Top figure: Difference of the automatic EW measurements between the
             summer observations and winter observations from the Arcturus
             spectra. Bottom figure: Same as above, but with manual measurements
             from ARES (summer) and automatic measurements (summer).}
    \label{fig:EWcomp}
\end{figure}

There is overall good agreement between the derived parameters and the average values from the
literature adopted (see \tref{tab:results}). The only parameter being difficult to measure is the
surface gravity due to the low number of \ion{Fe}{II} lines in the NIR. This was already suspected
and mentioned in \sref{sec:linelist_second}. There are also some problems determining
$\xi_\mathrm{micro}$, however this is not a major concern, since it is not as important a parameter
as the other three. The derived metallicity is consistent with the literature here which is promising
since there were some problems with this parameter in \sref{sec:HD20010_first}.


\section{10 Leo}
\label{sec:10Leo}

The spectrum for 10 Leo was made available by the CRIRES-POP team \citep{Nicholls2017}. 10 Leo is
very similar to Arcturus, which is also one reason this star was the first to be fully reduced by
the CRIRES-POP team. The spectrum is divided into several pieces according to the atmospheric
windows in the NIR: YJ (only together), H, K, L, and M. Here the first three are used. Some small
gaps are present in the spectrum due to tellurics that could not be properly removed, low S/N, bad
pixels, etc. Rather than giving an uncertain interpolation, \citet{Nicholls2017} decided to leave
small gaps in the data. This has very little effect on this line by line analysis, however, due to
those gaps, one \ion{Fe}{II} line were not measured which are generally important to determine the
surface gravity.

The approach for determining the atmospheric stellar parameters for 10 Leo is identical to Arcturus.
We use \ARES on each band (YJ, H, and K-band) separately. For the small gaps in the spectrum, we
simply set the flux to 1, since the spectrum is already normalised. This will also prevent \ARES to
identify and measure any lines in these regions. The EWs from the three regions are combined to one
final line list used for the determination of the parameters. The final results can be seen in
\tref{tab:results}.

Generally the derived parameters are in excellent agreement with the literature values listed here.
For $T_\mathrm{eff}$ we were \SI{64}{K} off with $\log g$ set as a free parameter, well within the
errors. The only parameter that show a discrepancy compared to the literature value is
$\xi_\mathrm{micro}$ with a difference of \SI{0.22}{km/s}, which is at the limit of the errors
reported. We note that this parameter is not reported in the PASTEL database, and this was a derived
parameter from a empirical relation. We were able to derive good $\log g$ values, although with
larger errors compared to the results from the literature.


\section{Synthetic cool stars}
\label{sec:synthetic_spectra}



\section{SWEET-Cat and parameters for 50 planet hosts}
\label{sec:SWEET-Cat}




\section{Parameter dependence on EP cut}
\label{sec:EPcut}

It is common practise, as in this case, to make a cut in EP for a line list when deriving
parameters. This was suggested in \citet{Andreasen2016} \citep[later done in][]{Andreasen2017b} for
the NIR line list used here. This cut was made at \SI{5.5}{eV}, inspired by a similar cut in the
optical \citep{Sousa2008a}, including only lines below this limit. The lines with higher EP might
also be affected by non-LTE effects which is not treated in this methodology.

As the parameters are dependent on this cut in EP\unfinished{Make a figure that shows this}, it
seemed interesting to divide a line list in two with upper and lower EPs and analyse those
separately. Before going into that analysis it is important to note, that the parameters should not
depend on any cut in EP if the theory is right, the radiative transfer code is working properly, and
the atmospheric models are correct. Lines with different EP are likely to be formed in different
layers of the atmosphere as discussed in \sref{sec:stellar_atmosphere}, however this should not
effect the final derived abundance, which is the problem here.
