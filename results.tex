%!TEX root = thesis.tex
\chapter{Results for FGK stars}
\label{cha:results}
\epigraph{Don't cry because it's over, smile because it happened.}{Dr. Seuss}

It is time to apply the theory and methodology on some data. In this chapter results from studies
during the thesis will be presented. There is the analysis of three stars which lead to two papers,
HD20010 \citep{Andreasen2016} and Arcturus \& 10 Leo \citep{Andreasen2017b}. Additionally, there was
a analysis of how a cut in EP might affect the final parameters of a star. This will be discussed in
\sref{sec:EPcut}, and last the analysis of a range of synthetic spectra from the PHOENIX spectral
library \citep{Husser2013}, with $T_\mathrm{eff}$ lower than analysed in previous stars with the
methodology described above in \sref{sec:parameters}.


\section{Parameter dependence on EP cut}
\label{sec:EPcut}


\section{HD20010}
\label{sec:HD20010}

To test our new line list we search for a well-studied solar-type star. The spectrum for such a
target needs to be available in the NIR at both high resolution and high S/N. An ideal place to look
for such a star is the CRIRES-POP database (Lebzelter et al. 2012). Here, the best target for
testing is HD 20010, an F8 subgiant star. This star has been part of many surveys and is therefore
well studied. Different parameters from the literature are listed in Table 3. The data available at
CRIRES-POP are in the raw format and pipeline reduced, while three small pieces of the spectra are
fully reduced on the web page 3 . The data is in the standard CRIRES format with each fits file
including four binary tables with the data from the four detectors. In the future, the final reduced
data will be presented by the CRIRES-POP team. In contrast to the pipeline reduced data, this will
be of higher quality, a better wavelength calibration, and telluric correction. We measured the EWs
of the pipeline reduced spectra, and where there was an overlap with the fully reduced spectrum, we
measured both as a consistency check. The measured EWs from the fully reduced spectra were
consistent with the measured EWs from the pipeline reduced spectra. As mentioned above, we use the
Y, J, H, and K-bands which are all available for this star. The spectra come in pieces of 50 Å to
120 Å. These pieces over- lap each other, and we were able to measure the EW for a single line up to
five times. Unfortunately, wavelength calibration is a difficult task for CRIRES owing to the rather
small spectral regions measured on each detector. Each calibration was performed separately for each
detector and required the avail- ability of a sufficient number of calibration lines in the
respective spectral region. This was not always the case and a default linear solution was applied.
A pipeline reduced spectrum shows up as a stretched spectrum if the wavelength calibration is poor
compared to a model spectrum or a solar spectrum, for example. The wavelength calibration does not
have any effect on the signal-to-noise ratio, which is generally high for the spectrum of HD 20010.
The signal-to-noise varies between 200 and 400 for different chunks. The pipeline reduced spectra
for HD 20010 contains tellurics and the wavelength is shifted in radial velocity. All of these
factors make the line identification very difficult, and so we developed a program to properly
identify the lines, which does the following:

\begin{enumerate}
  \item Plotting the observed spectrum
  \item Overplotting a model spectrum. In this particular case the solar spectrum was used since the
        atmospheric parameters are close enough, so the sun was able to serve as a model
  \item Overplotting a telluric spectrum from the TAPAS web page \citep{Bertaux2014}
  \item Overplotting vertical lines at the location of lines in the list
  \item Calculating the cross-correlation function (CCF) for the telluric spectrum with respect to
        the observed spectrum, locating the maximum value by a Gaussian fit, and using this to shift
        the telluric spectrum with the found RV;
  \item Performing the same as step 5, but for the model
  \item Shifting the lines with the same RV as found for the model/solar spectrum.
\end{enumerate}

The final plot shows the shifted spectra, and the CCFs at the sides. An example of the software in
use is shown in Fig. 6. The two RVs are part of the title of the plot. Once the lines were
identified, the EWs were measured with the splot routine in Image Reduction and Analysis Facility
(IRAF). The reason not to choose ARES for this task was to visually confirm the identification of
the line given the relative poor wavelength calibration. We were able to measure 249 \ion{Fe}{I}
lines and 5 \ion{Fe}{II} lines compared to 344 Fe  lines and 13 Fe  lines for the Sun over the
whole NIR spectral region. Whenever we had more than one measurement of a line, the average was used
for the final EW. We derived the stellar parameters using the standard procedure (see Sect. 2.6) as
done for the Sun. Given the relatively low quality of the spectrum of HD 20010 (see below) and be-
cause it is not corrected for telluric contamination, we made a cut in EW at \SI{5}{m\AA{}} in order
to remove the lines which are most affected by contamination from either telluric or other line
blends. Additionally, we made a cut in EP at \SI{5.5}{eV} because the \ion{Fe}{I} and \ion{Fe}{II}
lines usually used for stellar parameter determination in the optical regime are also limited to
similar values \citep[see e.g][]{Sousa2008}. Higher excitation potential lines are also more likely
to be affected by non-LTE effects. When deriving the atmospheric parameters, we made a $3\sigma$
outlier removal in the abundance iteratively until there were no more outliers present. Since we
could only measure 5 \ion{Fe}{II} lines, for comparison we also decided to derive parameters using
the same method, but we fixed the surface gravity to the reference value. The resulting
atmospheric parameters and iron abundances are presented in Table 4. The effective temperature,
surface gravity, and metallicity agree within the errors with the literature values. Similar
parameters are obtained by fixing log g to the average literature value or by leaving it free.

The errors on the atmospheric parameters for HD 20010 are much higher than what is achievable with
other measurements in the literature, as presented above in Table 3. In order to explain these
errors, we calculated the abundances for all lines which have at least two measurements of the EW.
We then calculated the abundances for the highest measured EW and the lowest. The differences in
abundances are presented in Fig. 7. The very large differences (more than 0.1 dex) translate to the
high errors in the parameters.

The source of the large errors on the parameters can be seen more clearly where abundances are
compared to excitation potential or abundances versus reduced EW. Here the dispersion on the
abundances can be seen clearly, as shown in Fig. 8.

This test strongly suggest that errors in the EWs, likely due to the poor quality of this spectrum,
are responsible for the relatively large error bars in the derived stellar parameters. Systematic
errors (e.g. due to a possible non-optimal reduction of the spectrum) may be the reason for these
large error bars. As the CRIRES-POP team continue their great efforts in reducing the optimal
spectra, it will be interesting to re-visit this star once the entire spectrum has been fully
reduced.




\section{Arcturus}
\label{sec:arcturus}


\section{10 Leo}
\label{sec:10Leo}


\section{Synthetic cool stars}
\label{sec:synthetic_spectra}
