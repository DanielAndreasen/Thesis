%!TEX root = thesis.tex

\chapter{Deriving stellar parameters}
\label{cha:method}
\epigraph{The best way to learn is by doing. The only way to build a strong work ethic is getting
          your hands dirty.}{Alex Spanos}

There are different methods for obtaining stellar atmospheric parameters. Here follows a short
description of some of the most common methods, however the spectroscopic method will be explained
in much greater detail later in this chapter.


\section{Photometry}

Photometry can be used in different ways to estimate the effective temperature. In this section two
methods will be mentioned; a colour calibration, and asteroseismology. There are other methods, e.g.
SED fitting but this will not be discussed further.

\subsection{InfraRed Flux Method - IRFM}
\label{sec:irfm}

The InfraRed Flux Method (IRFM) was first described by \citet{Blackwell1977}. From IRFM it is
possible to measure the stellar radius and $T_\mathrm{eff}$ with a measurement of the angular
diameter, $\theta$, derived from infrared photometry. $T_\mathrm{eff}$ is derived from the angular
diameter from the simple relation
\begin{align}
  \sigma T_\mathrm{eff}^4 = \frac{4\mathcal{F}_E}{\theta^2}, \label{eq:irfm}
\end{align}
where $\mathcal{F}_E$ is the monochromatic flux measured at Earth, and $\sigma$ is Boltzmann's
constant. The angular diameter is calculated from the following equation:
\begin{align}
  \theta = 2\sqrt{\mathcal{F}_E/\mathcal{F}_S},
\end{align}
where $\mathcal{F}_S$ is the calculated monochromatic flux from the star. This flux is based on a
model atmosphere with an effective temperature based on the spectral energy distribution (SED). The
calculated flux show a strong dependence on $T_\mathrm{eff}$ in the visible, however this dependence
is much weaker in the infrared. Hence a poor first estimation of $T_\mathrm{eff}$ will lead to a
reliable angular diameter. From this new angular diameter $T_\mathrm{eff}$ can be re-derived, a new
model atmosphere can be used with a new set of $\mathcal{F}_S$ can be calculated. Iteratively the
angular diameter and $T_\mathrm{eff}$ can be calculated. If the distance $d$ is known of the star,
the stellar radius is $R_\ast = \frac{\theta d}{2}$. The solar flux, both measured and calculated,
from \citet{Blackwell1977} are shown in \fref{fig:IRFM}. Using the data provided the solar radius
and $T_\mathrm{eff}$ were derived using the equations above: $R=1.011R_\odot$ and
$T_\mathrm{eff}=\SI{5963}{K}$. This was simply done for each wavelength, and the results presented
here are just a simple average value.

\begin{figure}[htpb!]
    \centering
    \includegraphics[width=0.85\linewidth]{figures/IRFM.pdf}
    \caption{Measured and calculated flux from the Sun at infrared wavelengths. Data from Table 2 in
             \citet{Blackwell1977}. Mean solar radius from this data is $1.011R_\odot$, and mean
             solar $T_\mathrm{eff}=\SI{5963}{K}$ using \eref{eq:irfm}.}
    \label{fig:IRFM}
\end{figure}

The main drawbacks of the IRFM is the model dependence, a drawback many methods share, and the need
of high precision infrared photometry. For the model atmosphere a metallicity and surface gravity is
assumed which has an effect on $\mathcal{F}_S$, and hence on the final derived $T_\mathrm{eff}$ and
$R$. A more in-depth description of the IRFM can be see in e.g. \citet[][section 4]{Casagrande2006}.


\subsection{\texorpdfstring{$T_\mathrm{eff}$}{Teff}-colour-\texorpdfstring{$[\ion{Fe}/\ion{H}]$}{[Fe/H]} calibration}

Photometry can be used for deriving $T_\mathrm{eff}$ using existing colour calibrations like that of
for example \citet{Ramirez2005a} where adopted $T_\mathrm{eff}$ and $[\ion{Fe}/\ion{H}]$ in
combination with colours $X$ such as $B-V$, $V-S$, etc. are used to fit a polynomial such that the
$T_\mathrm{eff}$ can easily be estimated with a simple relation:
\begin{align}
  \theta_\mathrm{eff} = a_0+a_1 X+a_2 X^2+a_3X[\ion{Fe}/\ion{H}] + a_4[\ion{Fe}/\ion{H}] + a_5[\ion{Fe}/\ion{H}]^2, \label{eq:irfm}
\end{align}
where $\theta_\mathrm{eff}=5040/T_\mathrm{eff}$. A polynomial fit was performed on the residuals in
order to deal with spectroscopic features such as the Balmer lines, the Paschen jump, etc. This
polynomial fit, $P(X, [\ion{Fe}/\ion{H}])$ is added to \eref{eq:irfm} so the calibration finally
reads:
\begin{align}
  T_\mathrm{eff} = \frac{5040}{\theta_\mathrm{eff}} + P(X, [\ion{Fe}/\ion{H}]),
\end{align}

After obtaining the coefficients ($a_i$) for different combinations of colours, it is trivial to
obtain $T_\mathrm{eff}$ if the metallicity and a colour is known of the star.

\subsection{Asteroseismology}
\label{sec:asteroseismology}

Asteroseismology is the study of stellar pulsations. For main sequence stars (which will be the only
focus here), these pulsations propagate as sounds waves throughout a star, their origin and
amplitude is determined by the characteristics of the star. Hence the study of the pulsations, which
are seen on the surface, will thus be a study of the stellar properties. In order to to study these
a time series is needed. This can both be radial velocities as it was used in the recent results
from the SONG telescope \citep{Grundahl2017}, or photometry like the numerous results from e.g. the
space telescopes \emph{CoRoT} and \emph{Kepler} \citep[see
e.g.][]{Christensen-Dalsgaard2010,Chaplin2011,Huber2014}. The analysis is identical for either time
series, however the amplitudes in the power spectrum will be different.

After determining the frequencies of a range of pulsations from a power spectrum of the time series,
a pattern emerge at every $\Delta\nu$ (the so-called large frequency separation). A finer pattern
also occur described by $\delta\nu$. Last the frequency at maximum power is also measured from the
power spectrum, $\nu_\mathrm{max}$. The frequency at maximum power is used to obtain $\log g$ via
\begin{align}
  \nu_\mathrm{max} &\propto \frac{g}{\sqrt{T_\mathrm{eff}}} \\
                   &= \frac{M/M_\odot}{(R/R_\odot)^2 \sqrt{T_\mathrm{eff}/5777}}\; \SI{3.05}{mHz},\label{eq:scaling1}
\end{align}
where $\nu_{\mathrm{max},\odot}=\SI{3.05}{mHz}$ is the frequency of maximum amplitude for the Sun. A
similar equation exists for the determination of the stellar density:
\begin{align}
  \Delta\nu = (M/M_\odot)^{-1/2} (R/R_\odot)^{-3/2}\; \SI{134.9}{\micro\hertz}.
\end{align}
These simple scaling relation are described in detail in \citet{Kjeldsen1995} for main sequence
stars. These two equations can be used together to determine the mass and radius of the star, often
to a high precision. These scaling relations are applicable for stars which shows solar-like
oscillations, mostly found in main sequence FGK stars, but can also be found in red giant stars. The
mass and radius for a range of $\nu_\mathrm{max}$ and $\Delta\nu$ can be seen in \fref{fig:scaling},
where $T_\mathrm{eff}=\SI{5777}{K}$. The star located at $\{\Delta\nu=\SI{134.9}{\micro
Hz};\;\nu_\mathrm{max}=\SI{3.05}{mHz}\}$ is the Sun.

The small frequency separation, $\delta\nu$, is sensitive to the sound-speed gradient in the core
which in turn is sensitive to the composition. Thus the small frequency separation is a very
important diagnostic for stellar evolution. An interesting case is that of \citet{Bedding2011},
where it was shown it is possible to distinguish between hydrogen- and helium-burning cores in red
giant stars.

\begin{figure}[htpb!]
    \centering
    \includegraphics[width=0.85\linewidth]{figures/scaling_relation.pdf}
    \caption{Mass and radius from asteroseismic scaling relation. The colour is the mass and radius
             for the upper and lower panel, respectively. The location of the Sun is added for
             reference.}
    \label{fig:scaling}
\end{figure}

A drawback of asteroseismology is the dependence of $T_\mathrm{eff}$ in \eref{eq:scaling1} which has
to be provided from another method. Ideally this will come from spectroscopy for which the
determination of $T_\mathrm{eff}$ is often reliable. This drawback is minor compared to the weak
model dependence which is one of the strongest advantages of asteroseismology.

For both of the mentioned photometric methods to determine some atmospheric parameters a
disadvantage is the dependence on the knowledge of other atmospheric parameters which usually comes
from spectroscopy (e.g. metallicity). However, as will be discussed in
\sref{sec:method_spectroscopy} $\log g$ is often difficult to determine reliably and synergies are
welcomed between different methods.

\section{Spectroscopy}
\label{sec:method_spectroscopy}

A spectrum can be analysed with a range of different methods. The method finally chosen depend on
quality of the spectrum, i.e. high/low resolution and high/low S/N, the spectral type of the star,
the region that was observed, e.g. UV, optical, NIR, etc., some stellar properties, e.g.
fast-rotator, activity, etc. In practise, no methods work for all cases, but sometime several
methods can be used for on case.

Since this thesis is focused on FGKM stars, and mainly dwarfs, three methods will be described;
synthesis, spectral indices, and the EW method. The latter method will be described in a separate
section since this is the main method used for the analysis in this thesis.



\subsection{Synthesis}
\label{sec:synthesis}

The synthesis fitting method is a standard method for obtaining stellar atmospheric parameters from
a wide range of spectra, that is with different spectral resolution, spectral parameters such as
$T_\mathrm{eff}$, $\log g$, $v\sin i$ the projected rotational velocity, etc., and S/N  \citep[see
e.g.][]{Tsantaki2017}. The synthetic fitting method is in simple terms a comparison between the
observed spectrum and a synthetic spectrum, which is either calculated on the fly like Spectroscopy
Made Easy (SME) \citep{Valenti1996}, or using a pre-calculated grid like Starfish
\citep{Czekala2015}. By analysing the
\begin{align}
  \chi^2 = \sum_i^N\frac{(y_\mathrm{obs,i}-y_\mathrm{model,i})^2}{\sigma_i},
\end{align}
the synthetic spectrum that best match the observed spectrum can be found. Here the $y_\mathrm{obs}$
is the observed spectrum, $y_\mathrm{model}$ is the synthetic spectrum, and $\sigma$ is the error on
the measurement.

The synthetic fitting can be done by utilising small windows around sensitive spectral features such
as ionized lines or hydrogen lines for obtaining the surface gravity, iron lines for obtaining the
effective temperature, a series of different atomic lines for obtaining the overall metallicity.
This approach is used by \code{SME} and \code{FASMA}
\citep[][respectively]{Valenti1996,Tsantaki2017} and is a compromise between fitting the entire
spectral range and calculating the synthetic spectra on the fly which is time consuming. On the
other hand the entire spectrum can be fitted if a pre-calculated grid of synthetic spectra are
available. By masking small windows, one can also exclude different features that are troublesome,
this can be telluric lines, bad reduction of the spectra, or real spectral features where there
currently is poor atomic/molecular data such as the oscillator strength and thus it is not possible
to reliably fit this feature.

This method is affected by the different approaches one can use, that is which atmosphere models are
used (ATLAS, MARCS, etc.), atomic data and whether this has been calibrated, the radiative transfer
code in the case the synthetic spectra are calculated on the fly, and the minimization procedure
chosen. With these things in mind it is important to stress the wide range spectra and spectral
classes this method works with.

The advantage of the synthetic fitting method over the curve-of-growth analysis (see
\sref{sec:parameters}) is that it allows for the analysis of lower resolution and with a higher
rotational velocity, $v\sin i$.


\subsection{The EW method and \code{FASMA}}
\label{sec:parameters}

The EW method or curve-of-growth analysis is another standard method for obtaining stellar
atmospheric parameters from spectra as the synthetic fitting method (\sref{sec:synthesis}). Since
this is the method used throughout this thesis it will be explained in detail. This analysis follow
a chain of tasks, each has been made automatic in the software ``Fast Analysis of Spectra Made
Automatically" (\code{FASMA}\footnote{Greek for spectrum}) which was developed during this thesis
\citep{Andreasen2017a}. \code{FASMA} is made of three \code{drivers}:
\begin{enumerate}
  \item EW measurement driver
  \item Obtain stellar atmospheric parameters driver
  \item Abundance driver
\end{enumerate}
An additional driver is under development; a synthetic fitting driver \citep{Tsantaki2017}.
\code{FASMA} has been made available to the community via a web application at
\url{http://www.iastro.pt/fasma/}.


\subsubsection{Ingredients}

\code{FASMA} is written in the Python programming language and glue together other software and
model atmospheres necessary for obtaining stellar atmospheric parameters from high quality spectra.
These software and models are described in greater detail in the following sections. In short, the
curve-of-growth analysis require measured EWs where the latest version\footnote{The latest version
can be found here: \url{https://github.com/sousasag/ARES}} of \code{ARES} is used
\citep{Sousa2015a}. These EWs are used to derive line abundances using model atmosphere like the
ATLAS9 \citep{Kurucz1993}, MARCS models \citep{Gustafson2008}, or PHOENIX models\footnote{The
PHOENIX models are currently not a part of \code{FASMA}, however it is planned to implement these
models with \code{MOOG}.} \citep{Husser2013} to mention the most popular for this analysis. Note that the
PHOENIX models are relative new and not as widely used yet. In tandem with model atmospheres a
radiative transfer code is also needed. \code{FASMA} uses \code{MOOG} \citep{Sneden1973} for this.
The model atmosphere usually comes in a pre-calculated grid in the $\{T_\mathrm{eff},\,\log
g,\,[\ion{Fe}/\ion{H}]\}$ parameter space. These are interpolated in order to access the requested
combination of parameters. Last, \code{FASMA} consist of a minimization routine which looks for the
best matching parameters given a spectrum.



\subsubsection{Wrapper for \code{ARES}}
\label{sec:measureEW}

There are two ways to measure the EW of an absorption line, ``manually'' or automatically. There are
advantages and disadvantages for both approaches: For the manual, an advantage is that we can
inspect the lines and try to measure lines in different ways (which is useful if a absorption line
is blended). We have more control over how blended lines are fitted, and which profiles are used.
Disadvantages are that it is very time consuming, and it is prone to errors, as a measurement might
change drastically by the eyes measuring it. Even for the same person, the measurement can change.
By mentioning the advantages and disadvantages of the manual method, it should be clear that the
advantages and disadvantages of the automatic method is the opposite of those. Especially the time
to measure the lines are orders of magnitudes faster, which is crucial when dealing with more than a
handful of spectra. However, most important is the fact that the EWs of the lines are measured
consistently throughout the entire spectral range, allowing a homogeneous analysis of the lines.

When a line is measurement by hand (manually) it is in this thesis done using the \code{splot}
command in \code{IRAF}. Here the deblending mode is used whenever necessary. It is often necessary
to fit one spectral lines with several Gaussians, as neighbouring lines might contaminate the line
of interest.


Throughout this thesis line EWs are automatically measured with \code{ARES}
\citep{Sousa2007,Sousa2015a}. When using \code{ARES} it is important to use a correct value of the
\code{rejt} parameter. This parameter is used for placing the continuum level, and is thus directly
related to the final measurement EW. It is difficult to get this parameter right, however the newest
version of \code{ARES} has the option to analyse a few absorption free regions and measure the S/N.
The \code{rejt} is then calculated as: \begin{align*} \mathtt{rejt} = 1 - \frac{1}{\mathrm{S/N}}.
\end{align*}

\code{ARES} is used via the first driver of \code{FASMA}. All the options available for \code{ARES}
can be accessed by \code{FASMA}. The options are
\begin{itemize}
  \item Setting the spectral window, $\lambda_\mathrm{min}$ and $\lambda_\mathrm{max}$
  \item RV correction to be applied or a mask to measure the RV and automatic make this correction
  \item Minimum and maximum EW to be considered ($\SI{5}{m\angstrom}$ and $\SI{150}{m\angstrom}$
        respectively by default)
  \item Minimum acceptable distance between two consecutive lines
  \item Smoothing applied with a \code{boxcar} filter before measuring the EWs. This is only for
        automatic line identification.
\end{itemize}
An in-depth description of these options can be found in
\citet{Sousa2007,Sousa2015a}.

Rarely \code{ARES} crash when measuring an absorption line. The reason is not clear, however when
dealing with a large amount of spectra, it is important that the analysis moves on. However, a
closer inspection usually reveal strange spectral features such as zero flux, sharp peaks in the
spectra, etc. To deal with this problem, \code{FASMA} finds the last line which \code{ARES} tried to
measure in the log file. This line is temporarily removed from the line list and \code{ARES} is
restarted. The line list used for deriving parameters consists of numerous iron lines, thus removing
one line will have a negligible effect on the final derived parameters.



\subsubsection{Interpolation of atmosphere models}
\label{sec:interpolation}

\code{FASMA} has access to both ATLAS9 models by \citet{Kurucz1993} and MARCS models by
\citet{Gustafson2008}, both are in a pre-calculated grid as described above. Let this grid be
described by $\{T_\mathrm{eff,g},\, \log g_g,\, [\ion{Fe}/\ion{H}]_g\}$, where subscript $g$ is one
of the grid points. Such a grid can be seen in \fref{fig:grid} for $[\ion{Fe}/\ion{H}]=0.00$ in the
$T_\mathrm{eff}$ range; \SIrange{3000}{10000}{K}. For visualisation the location of the Sun is
shown as well. The colour scale corresponds to the temperature in the first layer of each model
atmosphere, i.e. the uppermost layer. The requested value will be $\{T_\mathrm{eff,r},\,\log
g_r,\,[\ion{Fe}/\ion{H}]_r\}$. The task is now to find the surrounding grid points in the parameter
space of the requested parameters. For $\log g$ and $[\ion{Fe}/\ion{H}]$ two neighbouring grid point
are used, and for $T_\mathrm{eff}$ four surrounding grid point are used, in total
$4\times2\times2=16$ model atmospheres for the interpolation. \code{FASMA} use the four surrounding
grid points for $T_\mathrm{eff}$ instead of two, since the model atmosphere changes most with
$T_\mathrm{eff}$. This is common in other interpolations as well \citep[see e.g.][]{Valenti1996}.

\begin{figure}[htpb!]
    \centering
    \includegraphics[width=0.85\linewidth]{figures/model_atmosphere.pdf}
    \caption{Model atmosphere grid from \citet{Kurucz1993} at $[\ion{Fe}/\ion{H}]=0.00$ between
             \SI{3000}{K} and \SI{10000}{K}. The grid extends to higher $T_\mathrm{eff}$, but these
             are not considered in this thesis.}
    \label{fig:grid}
\end{figure}

When the 16 model atmosphere have been located, the interpolation goes through each layer of the
model atmosphere, where there typical are 72 layers, and each column of which there are six. The
columns are described in \sref{sec:atmospheremodels}. The interpolation are done using the
\code{griddata} function from \code{SciPy}\footnote{\url{https://scipy.org/}}. The interpolation is
linear in the parameter space. After the interpolation, the result is saved to a file in the format
expected by \code{MOOG}.




\subsubsection{Minimization}
\label{sec:minimization}

With the measured EWs for all the lines in the line list, we choose an atmosphere model to determine
the abundances. If there is no prior knowledge of the star it is common simply to choose an
atmosphere model with solar parameters as a starting point. Once the line abundances of all the iron
lines has been determined, the linear correlation between the abundances and the reduced EWs (RW)
$a_\mathrm{RW}$, and the abundances and the excitation potential $a_\mathrm{EP}$ is calculated. If
there is a correlation it means the model atmosphere used is wrong. Moreover, we also have to check
if the mean abundance of \ion{Fe}{I} and \ion{Fe}{II} lines are equal, and last if mean abundance of
the \ion{Fe}{I} lines is equal to the input $[\ion{M}/\ion{H}]$ of the atmosphere model\footnote{We
use \ion{Fe}{I} instead of \ion{Fe}{II} lines for this, since they are more numerous.}. If one of
these four criteria does not pass, then the atmosphere model is wrong, and we have to search for a
new one. A common way to do this, is by combining the indicators into a scalar value:
\begin{align}
  f(\{T_\mathrm{eff}, \log g, [\ion{Fe}/\ion{H}], \xi_\mathrm{micro}\}) &= \sqrt{a_\mathrm{EP}^2 + a_\mathrm{RW}^2 + \Delta\ion{Fe}{}^2},
\end{align}
where $a_\mathrm{EP}$ is the correlation between abundances and excitation potential,
$a_\mathrm{RW}$ is the correlation between abundances and RW, and $\Delta\ion{Fe}{}$ is the
difference between the mean abundances of \ion{Fe}{I} and \ion{Fe}{II}. This scalar function can be
minimized using standard minimization procedures as the simplex downhill among others. However,
there is another approach that takes into the account the information stored in these indicators.
For example, if $a_\mathrm{EP}$ is positive it means $T_\mathrm{eff}$ has to be increased by an
amount correlated by the numerical value of $a_\mathrm{EP}$. In the same way, a non-zero
$a_\mathrm{RW}$ means $\xi_\mathrm{micro}$ has to be changed, and $\Delta\ion{Fe}{}$ is an indicator
for $\log g$. In the end it is a vector function being minimized which are more difficult, however
we are not minimizing this using standard mathematical methods, but rather using the physical
knowledge. This minimization is useless for anything else, but it is excellent for this. The vector
function has the form:
\begin{align}
    f(\{T_\mathrm{eff}, \log g, [\ion{Fe}/\ion{H}], \xi_\mathrm{micro}\}) = \{a_\mathrm{EP}, a_\mathrm{RW}, \Delta\ion{Fe}, \ion{Fe}{I}\}.
\end{align}

The abundances of \ion{Fe}{I} lines versus EP and RW are shown in \fref{fig:eprw} for the planet
host star HATS-1. The three rows are for three different model atmospheres. From upper to lower:
\begin{itemize}
  \item Converged: $T_\mathrm{eff}=\SI{5959}{K}$,
                   $\log g=4.59$,
                   $[\ion{Fe}/\ion{H}]=-0.04$, and
                   $\xi_\mathrm{micro}=\SI{1.05}{km/s}$.
  \item Converged with \SI{0.5}{km/s} added to $\xi_\mathrm{micro}$.
  \item Converged with \SI{500}{K} added to $T_\mathrm{eff}$.
\end{itemize}
Left column show the abundances against the EP, and the right column is abundances against RW.

\begin{figure}[htpb!]
    \centering
    \includegraphics[width=0.85\linewidth]{figures/EP_RW_vs_abundance.pdf}
    \caption{The abundances of \ion{Fe}{I} for the planet host star: HATS-1.
             Upper plot: Converged parameters (see text for stellar parameters for this star).
             Middle plot: Converged parameters with \SI{0.5}{km/s} added to $\xi_\mathrm{micro}$.
             Lower plot: Converged parameters with \SI{500}{K} added to $T_\mathrm{eff}$.}
    \label{fig:eprw}
\end{figure}

\paragraph{Option: outliers - }

The minimization with the different options is depicted in \fref{fig:minimization}. In each
iteration where convergence is not reached, the input metallicity is changed to that of the average
output metallicity using the \ion{Fe}{I} lines. \code{FASMA} is able to set one or all of the four
atmospheric parameters to a fixed value, and when it reach convergence it checks if there are any
outliers in the abundances. These outliers are likely to come from a bad measurement of the EW of
the given line. This can be due to line blending or a poor spectral reduction. These can be removed,
either:
\begin{itemize}
  \item All outliers above $3\sigma$ once; minimization routine is restarted after removal of
        outliers.
  \item All outliers above $3\sigma$ iteratively; minimization routine is restarted after removal of
        outliers each time.
  \item One outlier above $3\sigma$ (with the highest deviation) is removed iteratively;
        minimization routine is restarted after removal of outliers each time.
\end{itemize}
It is optional to remove any outliers, but recommended.

All restarts of the minimization will start at the previous best found parameters. For the latter
two where outliers are removed iteratively, this will continue until no outliers are present. An
optical line list like the ones by \citet{Sousa2008a,Tsantaki2013} have been tested thoroughly and
due to the large amount of lines it is safe to remove a larger amount of lines and still obtain
reliable parameters, thus using the first option is common here. However, with a less tested line
list, like the one by \citet{Andreasen2016} (and refined in \citet{Andreasen2017b}), one should
remove outliers more carefully, and it is recommended that one outlier is removed iteratively.

\paragraph{Option: fix $\xi_\mathrm{micro}$ - }

Sometimes the minimization can not reach convergence with all parameters free. The first approach to
progress is to fix $\xi_\mathrm{micro}$ to a value. This parameter is known to depend on the
spectral type \citep[see e.g.][and references therein]{Tsantaki2013}. This is also shown in
\fref{fig:vtRelation} for a sample of 583 stars analysed with \code{FASMA}. \code{FASMA} use one of
two empirical relations to fix $\xi_\mathrm{micro}$ if this is close to either $0\si{km/s}$ or
$5\si{km/s}$ and $|a_\mathrm{RW}| > 0.050$ at the end of the minimization. The empirical relations
are:
\begin{align}
  \xi_\mathrm{micro} =
  \begin{cases}
    6.935 \cdot 10^{-4}\; T_\mathrm{teff} - 0.348 \log g - 1.437     & \text{For $\log g \ge 3.95$} \\
    2.72 - 0.457 \log g + 0.072 \cdot [\ion{Fe}/\ion{H}]             & \text{For $\log g < 3.95$},
  \end{cases}
\end{align}
where the first case is from \citet{Tsantaki2013} and the latter case is from \citet{Adibekyan2015}.
In this way $\xi_\mathrm{micro}$ is changed in each iteration according to one of these relations.
This option is called \code{autofixvt} in \fref{fig:minimization}.

\begin{figure}[htpb!]
    \centering
    \includegraphics[width=0.8\linewidth]{figures/vtRelation.pdf}
    \caption{$\xi_\mathrm{micro}$ dependence on $T_\mathrm{eff}$ and $\log g$ for a sample of 583
             stars.}
    \label{fig:vtRelation}
\end{figure}

\paragraph{Option: Change line list - }

After the minimization, it happens that the derived $T_\mathrm{eff}$ is below \SI{5200}{K}. At this
temperature it is well known that the line list by \citet{Sousa2008a} does not work well. Some of
the lines start to be blended for lower $T_\mathrm{eff}$, thus giving poor measurements of some EWs.
This is something that was corrected by \citet{Tsantaki2013}, who created a subset of the previous
iron line list. This smaller line list are able to successfully derive $T_\mathrm{eff}$ below
\SI{5200}{K}. Therefore, if the option \code{teffrange} is on (which is highly recommended), the
line list by \citet{Sousa2008a} will simply be converted to the smaller line list by
\citet{Tsantaki2013} if $T_\mathrm{eff}$ is below this threshold, and the minimization will be
restarted. This conversion happens after the convergence.

\paragraph{Option: Refine results - }

Last there is an option, \code{refine}. This apply more strict criteria for the indicators to reach
convergence, thus making the minimization less sensitive to the initial guess since it could
otherwise reach convergence from one ``side'' of the parameter space. The default criteria are:
\begin{align*}
  a_\mathrm{EP}     &= 0.001\\
  a_\mathrm{RW}     &= 0.003\\
  \Delta\mathrm{Fe} &= 0.001.
\end{align*}
The criteria for $a_\mathrm{RW}$ is not as strict as $a_\mathrm{EP}$ since this indicator can change
rapidly with small changes in $\xi_\mathrm{micro}$, thus a very strict criteria might never lead to
convergence. Convergence is reached once all of the above criteria are met, and the input and output
metallicity are identical. If one or more of the parameters are fixed, the corresponding criterion
is simply set to 0 and effectively ignored, thus not changing the parameter.

\paragraph{Stepping in each iteration - }

For each iteration, the change to be applied for the atmospheric parameters are defined by adding
the following:
\begin{align}
  T_\mathrm{eff}     &: \SI{2000}{K} \cdot a_\mathrm{EP}   \\
  \xi_\mathrm{micro} &: \SI{1.5}{km/s} \cdot a_\mathrm{RW} \\
  \log g             &: -\Delta\mathrm{Fe}
\end{align}
to each parameter. Note again that metallicity is simply changed to the the output metallicity of
the previous iteration. These are empirical relations. Note that by changing e.g. $T_\mathrm{eff}$
not only is $a_\mathrm{EP}$ affected, but the other indicators as well as seen in \fref{fig:eprw}.
This inter-dependency between the parameters is ignored by \code{FASMA} as it is not a simple
problem to solve. The stepping presented above is chosen to rapidly reach convergence, without
causing problems for the inter-dependency. This minimization is thus build to apply a standard
method, curve-of-growth analysis, using as few calls to the time consuming part, which is the
calculation of the abundances with \code{MOOG}.

\begin{figure}[htpb!]
    \centering
    \includegraphics[width=0.85\linewidth]{figures/FASMA_minimization.pdf}
    \caption{Overview of the minimization for \code{FASMA}. Credit: \citet{Andreasen2017a}.}
    \label{fig:minimization}
\end{figure}


\subsubsection{Error estimate}
\label{sec:error_estimate}

The error estimate is based on the same method presented in \citet{Neuforge1997}. The error on
$\xi_\mathrm{micro}$ corresponds to the $1\sigma$ statistical error on the slope of the linear
regression between \ion{Fe}{I} abundances and RW. The error on $T_\mathrm{eff}$ is the statistical
error on the slope between \ion{Fe}{I} abundances and EP as well as the uncertainty in
$\xi_\mathrm{micro}$. The error for $[\ion{Fe}/\ion{H}]$ corresponds to the dispersion of the
\ion{Fe}{I} abundances as well as the uncertainties in $\xi_\mathrm{micro}$ and $T_\mathrm{eff}$.
The error in $\log g$ corresponds to the dispersion in the pressure sensitive \ion{Fe}{II}
abundances.

\subsubsection{Testing \code{FASMA}}
\label{sec:fasma_test}

Parameters were derived for a 582 sample presented in \citet{Sousa2011} with \code{FASMA} as a test.
The results were compared with \citet{Sousa2011} since the method is the one previously used in the
Porto group, thus making a fair test.

\code{ARES} was used to measure the EWs. \code{ARES} can give an estimate of the S/N by analysing
the continuum in certain intervals. For solar-type stars the following intervals work well:
5764-5766 \AA{}, 6047–6053 \AA{}, and 6068–6076 \AA{}. From the estimated S/N, \code{ARES} can give
an estimate on the very important \code{rejt} parameters \citep[see][for more information]{Sousa2015a}.

After measuring the EWs with \code{ARES}, \code{FASMA} was used  to determine the stellar
atmospheric parameters. The results are presented in \fref{fig:fasma_test} which shows
$T_\mathrm{eff}$, $\log g$, $[\ion{Fe}/\ion{H}]$, and $\xi_\mathrm{micro}$ for \code{FASMA} against
those of \citet{Sousa2011}. The sample contains stars with $T_\mathrm{eff}$ too cold for the line
list used. As described in \sref{sec:minimization} the line list by \citet{Sousa2008a} should be
converted to the line list presented in \citet{Tsantaki2013}. However, since this line list was not
available when \citet{Sousa2011} derived parameters, the \code{teffrange} option was left off in
order to make a fair comparison for \code{FASMA}.

\begin{figure}[htpb!]
    \centering
    \includegraphics[width=1.0\linewidth]{figures/FASMAtest.pdf}
    \caption{Stellar atmospheric parameters derived by \code{FASMA} compared to the sample by
             \citet{Sousa2011}. The x-axis in all plots shows the results from \code{FASMA}, while
             the y-axis shows the parameters derived by \citet{Sousa2011}.}
    \label{fig:fasma_test}
\end{figure}


The mean of the difference between parameters from \citet{Sousa2011} and those by \code{FASMA} are
presented in \tref{tab:FASMATest}. The comparison is very consistent, as expected, and the small
offsets are within the errors except for metallicity. This can be due to different versions of
\code{MOOG}, measured line lists (i.e. using slightly different settings/version of ARES to measure
the EWs), interpolation of atmosphere grid, and minimization routine. Most likely the difference
will be due to the different \code{rejt} parameters used in \code{ARES}, which can alter the EWs
systematically and hence the metallicity. To test this hypothesis 20 randomly stars with different
$T_\mathrm{eff}$ were selected and the EWs directly from \citet{Sousa2011} were used to derive
parameters. The results are presented in the last column of \tref{tab:FASMATest}. Note that the
$\log \mathit{gf}$ values from the original line lists by \citet{Sousa2011}, which used the
\code{MOOG} 2002 version, were not changed for the 2014 version of \code{MOOG}. This might lead to
some errors as well. However, the offsets are very small and are compatible with the errors on
parameters normally obtained from high-quality spectra.

\begin{table}[htb!]
    \caption{Difference in derived parameters by \citet{Sousa2011} and \code{FASMA}. The second
             column is the mean difference with EWs measured by \code{ARES} in \code{FASMA}, while
             the third column is the mean difference using 20 randomly stars with the exact same EWs
             from \citet{Sousa2011}.}
    \label{tab:FASMATest}
    \centering
    \begin{tabular}{lrr}
      \hline\hline
      Parameter             &  Mean difference         & Same line list        \\
      \hline
      $T_\mathrm{eff}$      &  $\SI{16(36)}{K}$        & $\SI{21(11)}{K}$      \\
      $\log g$              &  $\SI{-0.04(7)}{dex}$    & $\SI{-0.007(9)}{dex}$ \\
      $[\ion{Fe}/\ion{H}]$  &  $\SI{0.03(2)}{dex}$     & $\SI{0.004(9)}{dex}$  \\
      $\xi_\mathrm{micro}$  &  $\SI{-0.04(14)}{km/s}$  & $\SI{0.04(2)}{km/s}$  \\
      \hline
    \end{tabular}
\end{table}
