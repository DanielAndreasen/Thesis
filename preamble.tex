\usepackage[utf8]{inputenc}        %så må vi bruge æøå
\usepackage[english]{babel} %dansk opsætning
\usepackage[T1]{fontenc}
\usepackage{amsmath,amssymb,bm}    %god matematik
\usepackage{mathtools}
\usepackage{graphicx}
\usepackage[english]{varioref}     %smartere referencer med \vref
\usepackage[separate-uncertainty = true]{siunitx}
\usepackage[version=3]{mhchem}     %reaktionsskemaer
\usepackage{wasysym}               %til bl.a. astronomiske symboler
\usepackage[final]{microtype}
\usepackage[hang,multiple]{footmisc}
\let\newfloat\relax                % Memoir har allerede defineret denne men det gør float pakken også
\usepackage{float}
\usepackage{placeins}              % Bruges med \FloatBarrier
\usepackage{xkvltxp}
\usepackage[draft]{fixme}
\usepackage[colorlinks, linkcolor=black]{hyperref}
\usepackage{color}
\usepackage{rotating}
\usepackage{threeparttable}

\usepackage{epigraph} %for quotes
\epigraphfontsize{\small\itshape}
%\setlength\epigraphwidth{8cm}
\setlength\epigraphrule{0pt}

%% TODONOTES
\usepackage{xargs}                      % Use more than one optional parameter in a new commands
\usepackage[pdftex,dvipsnames]{xcolor}  % Coloured text etc.
\usepackage[colorinlistoftodos,prependcaption,textsize=tiny]{todonotes}
\newcommandx{\change}[2][1=]{\todo[linecolor=blue,backgroundcolor=blue!25,bordercolor=blue,#1]{#2}}
\newcommandx{\reference}[2][1=]{\todo[linecolor=OliveGreen,backgroundcolor=OliveGreen!25,bordercolor=OliveGreen,#1]{#2}}
\newcommandx{\unfinished}[2][1=]{\todo[linecolor=red,backgroundcolor=red!25,bordercolor=red,#1]{#2}}
\newcommandx{\improvement}[2][1=]{\todo[linecolor=Plum,backgroundcolor=Plum!25,bordercolor=Plum,#1]{#2}}


%%%%%%%%%%% Litteraturlisten %%%%%%%%%%%%%%%
% When in doubt visit this page:
% http://adsabs.harvard.edu/abs_doc/aas_macros.html
\usepackage[sort,longnamesfirst]{natbib}
\def\aap{A\&A}
\def\aapr{Astronomy and Astrophysics Reviews}
\def\eprint{e-prints}
\def\apj{ApJ}
\def\apjs{ApJS}
\def\apjl{ApJL}
\def\mnras{MNRAS}
\def\aj{AJ}
\def\nat{Nature}
\def\aaps{A\&A Supp.}
\def\prd{Phys. Rev. D}
\def\prl{Phys. Rev. Lett.}
\def\araa{ARA\&A}
\def\actaa{Acta Astronomica}
\def\procspie{Proceedings of the SPIE}


\usepackage{xcolor} % Required for specifying custom colors
\usepackage{fix-cm} % Allows increasing the f

\setlength{\oddsidemargin}{0mm} % Adjust margins to center the colored title
%box
\setlength{\evensidemargin}{0mm} % Margins on even pages - only necessary if
%adding more content to this template

\newcommand{\HRule}[1]{\hfill \rule{0.2\linewidth}{#1}} % Horizontal rule at
%the bottom of the page, adjust width here

\definecolor{grey}{rgb}{0.9,0.9,0.9} % Color of the box surrounding the title -
%these values can be changed to give the box a different color

%%%%%%%%%%%%% Til forsiden %%%%%%%%%%%%%%%
%\setlrmarginsandblock{*}{2.5cm}{1.75} % højre og venstre
\setulmarginsandblock{3cm}{*}{2.2}    % top og bund


%\setlrmarginsandblock{3cm}{*}{1.25}
\checkandfixthelayout
\usepackage{mathpazo}              % palatino + matematik
\usepackage{soul}                  % lege lege
\sodef\an{}{0.2em}{.9em plus.6em}{1em plus.1em minus.1em}
\newcommand\stext[1]{\an{\scshape#1}}


%%%%%%%%%%%%% newcommands %%%%%%%%%%%%%%%
\newcommand{\cref}[1]{Chapter~\vref{#1}}
\newcommand{\sref}[1]{Section~\vref{#1}}
\renewcommand{\tref}[1]{\tablename~\vref{#1}}
\renewcommand{\fref}[1]{\figurename~\vref{#1}}
\newcommand{\aref}[1]{Appendix~\vref{#1}}
\newcommand{\eref}[1]{Equation~\vref{#1}}
\renewcommand{\epsilon}{\varepsilon}
\renewcommand{\bf}{\textbf}
\newcommand{\nicebreak}{\newline\newline\noindent}

%%%%%%%%%%%%%%%% Math %%%%%%%%%%%%%%%%%%%%
\newcommand{\F}{\mathcal{F}}
\newcommand{\tm}[1]{\textnormal{#1}}
\newcommand{\pd}[2]{\frac{\partial #1}{\partial #2}}

%%%%%%%%%%%%% Layout %%%%%%%%%%%%%%%%%%%%%
%% Chapter style from ftp://ftp.dante.de/tex-archive/info/MemoirChapStyles/MemoirChapStyles.pdf
\definecolor{nicered}{rgb}{.647,.129,.149}
\makeatletter
\newlength\dlf@normtxtw
\setlength\dlf@normtxtw{\textwidth}
\def\myhelvetfont{\def\sfdefault{mdput}}
\newsavebox{\feline@chapter}
\newcommand\feline@chapter@marker[1][4cm]{%
\sbox\feline@chapter{%
  \resizebox{!}{#1}{\fboxsep=1pt%
  \colorbox{nicered}{\color{white}\bfseries\sffamily\thechapter}%
}}%
\rotatebox{90}{%
  \resizebox{%
    \heightof{\usebox{\feline@chapter}}+\depthof{\usebox{\feline@chapter}}}%
  {!}{\scshape\so\@chapapp}}\quad%
  \raisebox{\depthof{\usebox{\feline@chapter}}}{\usebox{\feline@chapter}}%
}
\newcommand\feline@chm[1][4cm]{%
  \sbox\feline@chapter{\feline@chapter@marker[#1]}%
  \makebox[0pt][l]{% aka \rlap
  \makebox[1cm][r]{\usebox\feline@chapter}%
}}
\makechapterstyle{daleif1}{
  \renewcommand\chapnamefont{\normalfont\Large\scshape\raggedleft\so}
  \renewcommand\chaptitlefont{\normalfont\huge\bfseries\scshape\color{nicered}}
  \renewcommand\chapternamenum{}
  \renewcommand\printchaptername{}
  \renewcommand\printchapternum{\null\hfill\feline@chm[2.5cm]\par}
  \renewcommand\afterchapternum{\par\vskip\midchapskip}
  \renewcommand\printchaptertitle[1]{\chaptitlefont\raggedleft ##1\par}
}
\makeatother
\chapterstyle{daleif1}
\setsecnumdepth{subsubsection}       %Giver nummerering ned til og med subsection
\settocdepth{subsubsection}          %Giver nummerering ned til og med subsection i indholdsfortegnelsen



\captionnamefont{\small\bfseries}
\captiontitlefont{\small\itshape}



%% A&A stuff
\DeclareRobustCommand{\ion}[2]{\textup{#1\,\textsc{\lowercase{#2}}}}
\newcommand*\element[1][]{%
  \def\aa@element@tr{#1}%
  \aa@element
}
