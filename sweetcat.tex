%!TEX root = thesis.tex


\chapter{SWEET-Cat}
\label{sec:SWEET-Cat}

Part of the work during the thesis has been dedicated to regularly update
SWEET-Cat\footnote{\url{https://www.astro.up.pt/resources/sweet-cat/}}, a catalogue with all
discovered planet hosts, and the stellar parameters.

In this chapter a detailed description of SWEET-Cat will be presented. Moreover an analysis of 50
planet hosts was performed during the thesis with updated planetary parameters (mass and radius).


\section{What is SWEET-Cat?}

As mentioned above, SWEET-Cat is a catalogue of planet host stars. However, the strength of
SWEET-Cat is the homogeneously analysed stars utilising the method described in
\sref{sec:parameters} with \code{FASMA} or a similar tool before the creation of \code{FASMA}.

In the era with a large number of discovered exoplanets (more than 3500 confirmed exoplanet at the
moment of writing), the time for in-depth statistical studies has arrived. However, when conducting
these studies it is crucial to have consistent measurements of e.g. stellar atmospheric parameters.
This can be obtained by using a single analysis to obtain these parameters, as it is know that
different methods will lead to different results \citep[see e.g.][for a recent review]{Hinkel2016}.

To obtain stellar atmospheric parameters from one method is an on-going goal with SWEET-Cat, where
high quality spectra are obtained for stars hosting planets. These are used to determine the stellar
parameters in a homogeneous way. All stars in SWEET-Cat analysed with the method from our group are
marked with a flag showing whether it is analysed homogeneously or not. The columns provided in
SWEET-Cat are summarised in \tref{tab:sweetcat}. It is important to note that SWEET-Cat does not
include any planetary parameters.

\begin{table}[htb!]
    \caption{Columns in SWEET-Cat}
    \label{tab:sweetcat}
    \centering
    \begin{tabular}{lrl}
      \hline\hline
      Column                         & Unit      & Description \\
      \hline
      Name                           &           & Popular stellar name                                 \\
      HD number                      &           & HD number                                            \\
      RA                             & \si{deg}  & Right ascension                                      \\
      Dec                            & \si{deg}  & Declination                                          \\
      $\mathrm{Vmag}$                & \si{mag}  & V magnitude                                          \\
      $\sigma(\mathrm{Vmag})$        & \si{mag}  & Error on V magnitude                                 \\
      $\pi$                          & \si{mas}  & Parallax                                             \\
      $\sigma(\pi)$                  & \si{mas}  & Error on parallax                                    \\
      Source of $\pi$                &           & Source of parallax measurement                       \\
      $T_\mathrm{eff}$               & \si{K}    & Effective temperature                                \\
      $\sigma(T_\mathrm{eff})$       & \si{K}    & Error on effective temperature                       \\
      $\log g$                       & \si{dex}  & Surface gravity                                      \\
      $\sigma(\log g)$               & \si{dex}  & Error on surface gravity                             \\
      $\log g_{\mathrm{LC}}$         & \si{dex}  & Surface gravity corrected from light curves          \\
      $\sigma(\log g_{\mathrm{LC}})$ & \si{dex}  & Error on surface gravity corrected from light curves \\
      $\xi_\mathrm{micro}$           & \si{km/s} & Micro turbulence                                     \\
      $\sigma(\xi_\mathrm{micro})$   & \si{km/s} & Error on micro turbulence                            \\
      $[\ion{Fe}/\ion{H}]$           & \si{dex}  & Metallicity                                          \\
      $\sigma([\ion{Fe}/\ion{H}])$   & \si{dex}  & Error on metallicity                                 \\
      Mass                           & $M_\odot$ & Stellar mass                                         \\
      $\sigma(\mathrm{Mass})$        & $M_\odot$ & Error on stellar mass                                \\
      Reference                      &           & Reference for parameters                             \\
      Homogeneity flag               & 0/1       & 0 for not homogeneous analysis, 1 otherwise          \\
      Last updated                   & date      & Last updated                                         \\
      Comments                       &           & Any special remarks/comments (e.g. M star)           \\
      \hline
    \end{tabular}
\end{table}

SWEET-Cat is updated on a weekly basis if new planet hosts are discovered, and whenever planet hosts
have been analysed with the method from our group, as described in this thesis.


\section{Data for 50 planet hosts}

In this section the data for a large update to SWEET-Cat will be described. The majority of the data
comes as a result from proposals submitted for observational time, while some of the data was found
in the archive. In the next section the analysis of the 50 spectra will be presented along with the
results.


\subsection{Proposals for observation time}



\subsection{Data collected from proposals}



\subsection{Data collected from archive}



\section{Analysis}


The method of determining atmospheric parameters from the curve of growth analysis has been applied
several times in the optical \citep[see e.g.][]{Mortier2013b,Tsantaki2013,Sousa2011,Santos2013}.
When studying stars with planets and any correlations between stellar and planetary parameters it is
important to have a homogeneous characterisation of the stars. An effort to create such a sample was
initiated by \citet{Santos2013} with the SWEET-Cat database. The motivation to homogenise the
stellar hosts is mainly to compare the hosts and make statistical studies on one consistent scale.
When doing these statistical studies, the results might otherwise suffer from offsets between
different methods.

The skills acquired during the NIR studies as mentioned above were directly translated into deriving
parameters for a sample of 50 known planet host stars that were not previously analysed by our group
\citep{Andreasen2017a}. The spectra of these stars were required at UVES, FIES, HARPS, and ESPaDOnS
with the mean S/N higher than 200\unfinished{Write more about the data acquisition here}.

A Hertzsprung-Russell diagram of the sample can be seen in \fref{fig:sweetcat}. The sample covers a
large range of $T_\mathrm{eff}$, FGK, while there are both dwarf, sub-giant, and some giant stars.
The colours indicate the $\log g$. In order to determine the luminosity of each star the simple
relation
\begin{align*}
  L = 4\pi R^2 \sigma T^4_\mathrm{eff}
\end{align*}
is used, where $L$ is the luminosity, $R$ is the stellar radius, and $\sigma$ is the
Stefan-Boltzmann constant. In solar units this relation is simply:
\begin{align*}
  \frac{L}{L_\odot} = \left(\frac{R}{R_\odot}\right)^2 \left(\frac{T_\mathrm{eff}}{T_{\mathrm{eff},\odot}}\right)^4
\end{align*}
In order to determine the the stellar radius, the empirical relation from \citet{Torres2010} was
used.

\begin{figure}[htpb!]
    \centering
    \includegraphics[width=1.0\linewidth]{figures/HR.pdf}
    \caption{A Hertzsprung-Russell diagram of the sample of 50 planet host stars added to SWEET-Cat.
             The parameters were derived using optical high resolution and high S/N spectra in
             tandem with \code{FASMA} and an optical line list. The colour scale shows the derived
             $\log g$ for each star.}
    \label{fig:sweetcat}
\end{figure}

The parameters were derived using \code{FASMA} with the optical line list compiled by
\citet{Sousa2008a} and \citet{Tsantaki2013} for stars where $T_\mathrm{eff}$ was below \SI{5200}{K}.
All the new derived parameters were added to SWEET-Cat, available for the community.

With these updated parameters the completeness of SWEET-Cat for stars brighter than V magnitude 10
is 85\% (77\% for stars brighter than 12). For fainter stars it is time expensive to acquire spectra
of the quality needed for this method. Moreover, many of the fainter planet host stars have been
observed with the \emph{Kepler} space mission, where most stars are faint.


\subsection{Changes to planetary parameters}

As a results to the analysis above, it is expected that some planetary parameters will change
compared with the previous literature values.

Therefore the radius and mass of all the 50 new stars updated in SWEET-Cat were computed using the
empirical formula presented in \citet{Torres2010}. Some of the stars have radii derived from
different methods, usually from isochrones. These radii generally show a good correlation with radii
derived from \citet{Torres2010} if the literature parameters of $T_\mathrm{eff}$, $\log g$, and
$[\ion{Fe}/\ion{H}]$ are used. However, when comparing with the new radius derived using the
parameters presented here, the results can differ by up to 65\%. This is shown in \fref{fig:RR} how
the radius calculated from \citet{Torres2010} differs between the literature atmospheric parameters
and the new homogeneous atmospheric parameters presented here. Note that stellar radii are provided
by many of the authors from different discovery papers, but here the atmospheric parameters via the
derivation of the stellar radius, as described above, are compared, rather than comparing the
stellar radii from different methods.

\begin{figure}[tpb]
    \centering
    \includegraphics[width=0.8\linewidth]{figures/radiusVSradius.pdf}
    \caption{Stellar radius on both axes calculated based on \citet{Torres2010}. The x-axis shows
             the stellar radius based on the atmospheric parameters from the literature, while the
             y-axis indicates the new homogeneous parameters presented here. The colour and size
             indicate the surface gravity. This clearly shows that the disagreement is biggest for
             more evolved stars.}
    \label{fig:RR}
\end{figure}

In the sections below there follow a discussion of the systems (seven stars, eight exoplanets) where
the radius or mass of the stars changes more than 25\% and how this influences the planetary
parameters. The changes in radius for a star is primarily due to changes in $\log g$, which can be
used as an indicator of the evolutionary stage of a star.

The planetary radius, mass, and semi-major axis were re-derived when possible following the three
simple scaling relations based on Newton's law of gravity \citep{Newton1687} for deriving mass and
distance and simple geometry for radius \citep[see e.g.][]{Torres2008}
\begin{align}
    M_\mathrm{pl,new} &= \left(\frac{M_\mathrm{\ast,lit}}{M_\mathrm{\ast,new}}\right)^{-2/3} M_\mathrm{pl,lit}  \\
    R_\mathrm{pl,new} &= \left(\frac{R_\mathrm{\ast,lit}}{R_\mathrm{\ast,new}}\right) R_\mathrm{pl,lit} \\
    a_\mathrm{pl,new} &= \left(\frac{M_\mathrm{\ast,lit}}{M_\mathrm{\ast,new}}\right)^{1/3} a_\mathrm{pl,lit},
\end{align}
where the subscript ``lit'' denotes the value from the literature used in the comparison, the
subscript ``new'' indicates the new computed values, the subscript ``pl'' is short for planet, and
the subscript ``$\ast$'' is short for star; $M$, $R$, and $a$ are mass, radius, and semi-major axis,
respectively. Note that for the literature values, the values reported directly from the literature
were used and not the derived radius and mass from \citet{Torres2010}. To identify outliers, the
radii and masses were compared when derived from \citet{Torres2010} since this is a measure of how
the atmospheric parameters have changed.


\subsubsection{HAT-P-46}
\label{sub:HAT-P-46}

HAT-P-46 has two known exoplanets according to \citet{Hartmann2014}. The outer planet HAT-P-46 c is
not transiting, hence we do not have a radius for this planet. The results we present in this paper
for this star come from UVES/VLT data with a S/N of 208. \citet{Hartmann2014} derives the following
spectroscopic parameters: $T_\mathrm{eff}=\SI{6120(100)}{K}$, $\log g=4.25\pm0.11$, and
$[\ion{Fe}/\ion{H}]=0.30\pm0.10$. We note that for this star the asteroseismic correction we apply
(see Section~\ref{sec:results}) results in a corrected $\log g$ below 4.2dex, so we used the
spectroscopic $\log g$ for this star.

If we derive the mass and radius of HAT-P-46 b with our new parameters, we obtain $R_\mathrm{pl} =
0.93R_J$, while \citet{Hartmann2014} derived $R_\mathrm{pl} = 1.28R_J$. We see no change in mass
(\citealt{Hartmann2014} found $M_\mathrm{pl}=0.49M_J$); however, there is a decrease in the radius,
and we end up with a more dense planet, $\rho_\mathrm{pl}=\SI{0.76}{g/cm^3}$ from
$\rho_\mathrm{pl}=\SI{0.28\pm0.10}{g/cm^3}$.

As the secondary companion does not transit we only have a limit on the minimum mass for this
planet. Here we get $M\sin i_\mathrm{pl} = 1.97M_J$ and \citet{Hartmann2014} presented $M\sin
i_\mathrm{pl} = 2.00M_J$, so a very small change, as expected.


\subsubsection{HD 120084}
\label{sub:HD_120084}

The exoplanet orbiting this star with a period of 2082 days and a quite eccentric orbit at 0.66 was
discovered by \citet{Sato2013}. The atmospheric parameters were derived by \citet{Takeda2008} using
a similar method to that described in this paper. The quality of the spectra they analysed, however,
were not as high as those used here. Using the HIDES spectrograph at the 188cm reflector at NAOJ,
\citet{Takeda2008} reported an average S/N for their sample of 100-300 objects at a resolving power
of 67 000. We used data from ESPaDOnS with a resolving power of 81 000, and with a S/N for this star
of 850. With our new parameters we obtain a slightly lower stellar mass for the star at
$1.93M_\odot$ compared to $2.39M_\odot$ obtained by \cite{Takeda2008}, hence the minimum planetary
mass is also slightly lower, from $m_\mathrm{pl}\sin i=4.5M_J$ to $m_\mathrm{pl}\sin i=3.9M_J$. We
see a 28\% decrease in the stellar radius, from $9.12R_\odot$ to $7.81R_\odot$. Since there are no
observations of the planet transiting, the planetary radius has not been computed.


\subsubsection{HD 233604}
\label{sub:HD_233604}

HD 233604 b was discovered by \citet{Nowak2013}, while the atmospheric parameters of the star were
derived by \citet{Zielinski2012}, who used the same method as described in this paper using the HRS
spectrograph at HET with a resolving power of 60 000 with a typical S/N at 200-250. We obtained the
spectrum for this star using the FIES spectrograph with a slightly higher resolution at 67 000, and
similar but also slightly higher S/N at 320 for this star.

This planet is in a very close orbit with a semi-major axis of $\sim 15R_\ast$ ($R_\ast$ is the
stellar radius) using the parameters from \citet{Nowak2013}. Using the updated parameters presented
in this paper we see a slight increase in the stellar mass from $1.5M_\odot$ to $1.9M_\odot$, and a
decrease in stellar radius from $10.5R_\odot$ to $8.6R_\odot$. This increases the semi-major axis to
$\sim 21R_\ast$. We note that the correct stellar radii are used to describe the semi-major axis in
both cases. The increase in stellar mass leads to an increase in the minimum planetary mass, from
$m_\mathrm{pl}\sin i=6.58M_J$ to $m_\mathrm{pl}\sin i=7.79M_J$.

\citet{Nowak2013} found a high \ion{Li}{} abundance at $A(\ion{Li}{})_\mathrm{LTE}=1.400\pm0.042$
for this star and speculated that this star might have engulfed a planet. A more likely explanation
is that this star has not yet reached the first dredge-up process \citep{Nowak2013}. We found a much
lower value, $A(\ion{Li}{})_\mathrm{LTE}=0.92$, and hence do not find the star to be \ion{Li}{}
rich. The \ion{Li}{} abundance we find is in excellent agreement with \citet{Adamow2014}. Even
applying a NLTE correction, as was done in \citet{Adamow2014} ($A(\ion{Li}{})_\mathrm{NLTE}=1.08$),
this star is not \ion{Li}{} rich.


\subsubsection{HD 5583}
\label{sub:HD_5583}

This exoplanet was discovered by \citet{Niedzielski2016} with an orbital period of 139 days around a
K giant. This exoplanet was discovered with the radial velocity technique, and we do not have a
planetary radius. The stellar parameters were derived in a similar manner to that presented here
\citep[see][and references therein]{Niedzielski2016}; our biggest disagreement is in the surface
gravity. We derive a $\log g$ that is higher by 0.34 dex, which gives a stellar radius that is
smaller by 37\%. The derived mass is 15\% higher, which in turn increases the minimum planetary mass
from $m_\mathrm{pl}\sin i=5.78M_J$ to $m_\mathrm{pl}\sin i=8.63M_J$. Even with the increase in mass,
it is still within the planetary regime for most inclinations, as was noted by
\citet{Niedzielski2016}.



\subsubsection{HD 81688}
\label{sub:HD81688}

This exoplanet was discovered by \citet{Sato2008} with the RV method. The host star is a metal-poor
K giant. The atmospheric parameters presented in \citet{Sato2008} are obtained via the same method
as presented in this paper, and the agreement is quite good. Once again the big disagreement is in
the surface gravity: ours is 0.48 dex higher. Even though the stellar parameters, and hence the
planetary parameters, do change, the radius and mass we derive are not far from the values presented
in the paper by \citet{Sato2008}. This is a case where the star was marked as an outlier, due to the
comparison between the radius and mass derived from \citet{Torres2010}.

The new stellar mass is the same as before, $2.1M_\odot$. The stellar radius changed from
$13.0R_\odot$ to $10.8R_\odot$. Since a transit of this star has not been observed and the stellar
mass remains the same, we do not see any change in the planetary parameters.

We note that this system is in an interesting configuration with a very close orbit around an
evolved star. This system, among others, has been the subject of work on planet engulfment
\citep[see e.g.][]{Kunitomo2011}.


\subsubsection{HIP 107773}
\label{sub:HIP_107773}

The planetary companion was presented in \citet{Jones2015} as an exoplanet around an
intermediate-mass evolved star. The stellar parameters were obtained from the analysis by
\citet{Jones2011} using the same method as presented here, but with a different line list, which
might lead to some disagreements. For this star we derive a higher $\log g$ (2.83 dex compared to
2.60 dex), thus we derive a slightly smaller star with $11.6R_\odot$ to $9.2R_\odot$ and
$2.4M_\odot$ to $2.1M_\odot$ for radius and mass of the star, respectively. The other atmospheric
parameters are very similar to those derived by \citet{Jones2011}. This leads to a reduced minimum
mass of the planetary companion from $m\sin i=1.98M_J$ to $m\sin i=1.78M_J$. The planetary radius
has not been measured.



\subsubsection{WASP-97}
\label{sub:WASP-97}

The exoplanet orbiting WASP-97 was discovered by \citet{Hellier2014}. The host star parameters were
derived using a similar method to that described in this paper after co-adding several spectra from
the CORALIE spectrograph. They reach a S/N of 100 with a spectral resolution of 50 000. The
parameters presented here come from the UVES spectrograph with a S/N of more than 200.

The parameters do not change much for this planet. The planetary mass changes from
$m_\mathrm{pl}\sin i=1.32M_J$ to $m_\mathrm{pl}\sin i=1.37M_J$ and the radius from $1.13R_J$ to
$1.42R_J$. This affects the density quite strongly; it changes from $\SI{1.13}{g/cm^3}$ to
$\SI{0.59}{g/cm^3}$. This exoplanet is then in the same category as Saturn; its density is lower
than water, but it is slightly larger than Jupiter.

\subsubsection{$\omega$ Serpentis (ome Ser)}
\label{sub:ome_Ser}

The exoplanet orbiting this star with a period of 277 days and an eccentric orbit at 0.11 was also
presented by \citet{Sato2013}. The atmospheric parameters were derived in the same way as for HD
120084. We used data from FIES with a resolving power of 67 000, and with a S/N for this star of
1168. With our new parameters we obtain a slightly higher stellar mass for the star at $2.19M_\odot$
compared to the value of $2.17M_\odot$ obtained by \cite{Takeda2008}. This change is not significant
enough to change the minimum planetary mass at $m_\mathrm{pl}\sin i=1.7M_J$. The stellar radius
decreases by more than one solar radius, from $12.3R_\odot$ to $11.1R_\odot$. However, since there
are no observations of transiting exoplanets, we cannot see the change in the planetary radius.



\subsubsection{o Ursa Major (omi UMa)}
\label{sub:omiUMa}

omi UMa b was discovered by \citet{Sato2012} using the RV method. The stellar parameters are from
\citet{Takeda2008}, as discussed above. The spectrum used for this star is from ESPaDOnS with a S/N
of more than 500 compared to the value of 100-300 reached for the large sample presented in
\citet{Takeda2008}. The luminosity and mass for omi UMa were obtained from theoretical evolutionary
tracks \citep[see][and references therein]{Sato2012}. The radius was then estimated using the
Stefan-Boltzmann relationship, using the measured luminosity and $T_\mathrm{eff}$.

The parameters presented here mainly differ in the surface gravity: ours is 0.72 dex higher at $\log
g=3.36$. This leads to a big change in stellar mass and radius from $3.1M_\odot$ to $1.6M_\odot$ and
$14.1R_\odot$ to $4.5R_\odot$, respectively. \citet{Sato2012} have reported that omi UMa b is the
first planet candidate around a star more massive than $3M_\odot$. With these updated results, the
minimum mass of the planet is now $m\sin i=2.7M_J$, whereas previously it was $m\sin i=4.1M_J$
\citep{Sato2012}. The exoplanet is not reported to transit, as seen from Earth, so we do not have a
radius for this exoplanet, which would have changed a great deal with these new results.





\section{Discovering two giant planet populations}


SWEET-Cat was recently combined with planetary masses to see two distinctive populations for giant
planets by \citet{Santos2017}. This can be seen in the mass histogram in \fref{fig:giantpopulations}
for the full sample of giant planets, with masses higher than 1 Jupiter mass and lower than 20
Jupiter masses, and for a sample constrained by: $\SI{4000}{K}\leq T_\mathrm{eff} \leq\SI{6500}{K}$
in order to have reliably atmospheric parameters from spectroscopic data, orbital periods above
\SI{10}{days} to avoid hot jupiters whose formation and migration process is debated \citep[see
e.g.]{Ngo2016}, orbital periods below 5 years to allow for the sample to be reasonable complete.
Last only stars brighter than 13 magnitude were included to ensure that the planetary masses can
have been derived with reasonable confidence using the radial velocities.

\begin{figure}[htpb!]
    \centering
    \includegraphics[width=1.0\linewidth]{figures/giantPopulation.pdf}
    \caption{Giant planet masses for the full sample and constrained sample (see text for details).
             This study was performed by \citet{Santos2017} to distinct two giant planet populations.}
    \label{fig:giantpopulations}
\end{figure}

By separating the distribution into two at $4M_{Jup}$, it can be shown \citep[see][for
details]{Santos2017} that the stars hosting the more massive giant planets are in average more
metal-poor compared to the stars hosting the lower mass giant planets. This suggest two different
stellar populations forming giant planets.


\section{Future work}
