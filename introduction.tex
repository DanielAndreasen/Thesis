\chapter{Introduction}

Effective temperature ($T_\mathrm{eff}$), surface gravity ($\log g$),
and metallicity ([M/H], where iron is normally used as a proxy)
are fundamental atmospheric parameters necessary to characterise a single
star, and to determine other indirectly fundamental parameters
such as mass, radius, and age from stellar evolution models
\citep[see e.g.][]{Girardi2000,Dotter2008,Baraffe2015}.
Precise and accurate stellar parameters are also essential in
exoplanet searches. Planetary radius and mass are mainly found from
transit lightcurve analysis and radial velocity analysis, respectively. The
determination of the mass of the planet implies a knowledge of the
stellar mass, while the measurement of the radius of the planet
is dependent on our capability to derive the radius of the star
\citep[see e.g.][]{Torres2008,Ammler2009,Torres2012}.

The derivation of precise stellar atmospheric parameters is not a simple task.
Different approaches often lead to discrepant results
\citep[see e.g.][]{Torres2010,Lebzelter2012b,Santos13}. Interferometry is
usually considered  an accurate method for deriving stellar radii
\citep[see e.g.][]{Boyajian2012}; however, it is only applicable for bright
nearby stars. Asteroseismology, on the other hand, reveals the inner stellar
structure by observing the stellar pulsations at the surface. From
asteroseismology it is possible to measure the surface gravity and mean density,
and therefore to calculate mass and radius with high precision \citep[see
e.g.][]{Kjeldsen1995}. However, for stars on the main sequence asteroseismic
methods can typically only be applied to FG stars, since the oscillation modes
of K and M dwarfs are likely too weak to be detected even with high precision
spectroscopy or photometry. Moreover, the effective temperature is needed when
applying asteroseismology in order to obtain the surface gravity and the mean
density.

A crucial parameter for the indirect determination of stellar bulk properties is
the $T_\mathrm{eff}$. In that respect, the infrared flux method (IRFM) has
proven to be reliable for FGK dwarf and subgiant stars. For higher accuracy the
IRFM needs a priori knowledge of the bolometric flux, reddening, surface
gravity, and stellar metallicity
\citep{Blackwell1977,Ramirez2005b,Casagrande2010}.

Finally, the use of high resolution spectroscopy along with stellar atmospheric
models is an extensively tested method that allows the derivation of the
fundamental parameters of a star \citep[see e.g.][]{Valenti2005,Santos13}. The
procedure depends on the quality of the spectra, their resolution, and
wavelength region. A fit to the overall spectrum can be applied for all spectral
resolutions, but are often time consuming \citep[see e.g.][]{Recio2006,Tsantaki2014}.
For resolutions higher than $\lambda/\Delta\lambda \sim 20\,000$ we can apply the
equivalent width (EW) method \citep[see e.g.][for details]{Tsantaki2013,Andreasen2017a}.
However, while the latter approach is often faster than the synthetic fitting,
it requires higher quality spectra, and the star to be slow rotating (below
$\SI{10}{km/s}$ to $\SI{15}{km/s}$).

Standard procedures are often used to derive stellar atmospheric parameters from
high quality spectra in the optical \citep[see e.g.][]{Valenti2005,Sousa2008a}.
With the advancement of high resolution near-infrared (NIR) instruments, we will
now be able to use a similar technique to that used in the optical part of the
spectrum \citep[see e.g.][]{Melendez1999,Sousa2008a,Tsantaki2013,Mucciarelli2013,Bensby2014}.
At the moment, the GIANO spectrograph installed at \emph{Telescopio Nazionale
Galileo} (TNG) is already available \citep{GIANO}, as is the \emph{infrared
Doppler instrument} (IRD) installed at the Subaru telescope \citep{IRD},
\emph{Calar Alto high-Resolution search for M dwarfs with Exoearths with
Near-infrared and optical Échelle Spectrographs} (CARMENES) for the \SI{3.5}{m}
telescope at Calar Alto Observatory \citep{CARMENES}, and iShell at the
\emph{InfraRed Telescope Facility} \citep{ishell1,ishell2}. Three new
spectrographs are planned for the near future: 1) The \emph{CRyogenic InfraRed
Echelle Spectrograph Upgrade Project} (CRIRES+) at the \emph{Very Large
Telescope} (VLT) \citep{CRIRESp} with expected first light in 2017, 2) \emph{un
SpectroPolarimètre Infra-Rouge A Near-InfraRed Spectropolarimeter} (SPIRou) at
\emph{The Canada-France-Hawaii Telescope} (CFHT) \citep{SPIROU1,SPIROU2} with
expected first light in 2017 as well, and 3) \emph{Near Infrared Planet
Searcher} NIRPS at the ESO 3.6m telescope in La Silla \citep{NIRPS}. The
spectral resolutions for these spectrographs range between $50\,000$ and
$100\,000$.

With the advance of the next generation NIR spectrographs, we are still
preparing the data analysis of stellar spectra, in particular how to get
reliable atmospheric parameters \citep[see e.g.][]{Onehag2012,Lindgren2016,Andreasen2016}.
The analysis of stellar spectra is well understood for FGK stars in the optical
part of the spectrum, however some work still needs to be done for the NIR part.

We continue our series of studies to explore the use of the NIR domain to derive
stellar parameters for FGK and M stars. In particular, here we analyse the atlas
of Arcturus and the spectrum of 10 Leo. For the analysis we use the iron line
list presented in \citet{Andreasen2016} (referred to as Paper I). In Paper I we
successfully tested our method on a slightly hotter star than the Sun, while in
this work we aim to test the method on cooler stars. The strength of the NIR
domain over the optical becomes clear when we move towards the cooler stars.
Here we see less continuum depression and line blending due to in particular
molecular features. Moreover, the cooler stars emit more light in the NIR domain
than the optical, and with the lightest stars being intrinsically faint, we thus
obtain the majority of the flux here.



\section{Planet host stars}
\label{sec:Planet_host_stars}
With the present diversity of exoplanets ot becomes increasingly important
to get an accurate and precise characterization of the planets in order to
study them in samples and on an individual level. An accurate and precise
characterization can give us an idea wether the planet is rocky, composed
of water or gaseous.








\section{Atmospheric parameters}
\label{sec:Atmospheric_parameters}
