%!TEX root = thesis.tex
\chapter{Introduction}
\label{cha:introduction}
\epigraph{Every atom in your body came from a star that exploded. And, the atoms in your left hand
          probably came from a different star than your right hand. It really is the most poetic
          thing I know about physics: You are all stardust.}{Lawrence M. Krauss}

Ever since the dawn of time, the humankind have looked at the stars and wondered if we are alone in
this Universe. To answer this question, one must look toward the field of extrasolar planets
(exoplanets). This is a rapidly growing field in astronomy and science in general. Ever since the
first confirmed discovery of an exoplanet around a millisecond pulsar in 1992 by
\citet{Wolszczan1992} and three years later, exoplanet 51 Peg b was discovered around a solar-type
star by \citet{Mayor1995}, more than 3600 exoplanets have been discovered at the time of writing,
July 2017\footnote{\url{http://exoplanet.eu/}}.

With the discoveries of exoplanets, the main focus of observational exoplanetology is now mainly on
finding a twin of Earth; that is a planet that can harbour life as we know it. An Earth twin will be
of similar size (both in mass and radius), and be located in a distance from its host star were
liquid water may exists on the surface. When speaking of detecting similar life as we know it, it is
important to note that it is the biosignatures that life will cause on the exoplanet's atmosphere
that the astronomers are and will be looking for. These biosignatures are the subtle changes life
can cause on its host planets atmosphere. Generally we refer to e.g. animals here on Earth which
produce carbon-dioxide, while plants produce oxygen. Before even start the search for these
biosignatures on another exoplanet, the exoplanet must be carefully characterised so we do not end
up looking blindly for the biosignatures. Such a search will be extremely costly. Additionally, it
may not be necessary to find an Earth replica in order to find conditions for life (see e.g.
\citet{Alibert2014} for a discussion on habitable planets unlike the Earth). Lastly, are the
compositions and origins interesting in this era of exoplanetary science. However, in order to
accurately and precisely characterise an exoplanet, it is crucial to characterise its host star.
This is commonly known as ``know the star, to know the exoplanet''. It is only possible to obtain
planetary global parameters such as its radius and mass if the stellar parameters are accurately
known. With sufficiently high precision it is even possible to distinguish between different
planetary bulk compositions such as water-worlds, rocky planets, gaseous planets, iron planets, or
exotic combinations of the above \citep[e.g.][]{Thiabaud2014,Dorn2015}. \citet{Dorn2015} found that
stellar elemental abundances (\ion{Fe}, \ion{Si}, \ion{Mg}) are key to reduce the degeneracy in
interior structure models and mantle composition.

In this chapter there will be a general introduction to exoplanets, detection methods, and
characterisation (\sref{sec:exoplanets}). Then a throughout introduction on the exoplanet host stars
(\sref{sec:planet_host_stars}), which is the main focus of this thesis. In the end of this chapter,
the importance of stellar studies will be discussed along with its wide-spread applications
(\sref{sec:stars_application}) before a general overview of this thesis (\sref{sec:this_thesis}).



\section{Exoplanets}
\label{sec:exoplanets}

The holy grail in the field of exoplanets is to find the first exoplanet with biosignatures. This is
by no means an easy task. To give an idea of the difficulty of detecting biosignatures on an
exoplanet, it is important to understand all the difficulties when simply detect and confirm the
presence an exoplanet. This will shortly be described in the sections below where several detection
techniques will be introduced.

\subsection{Detecting exoplanets}
\label{sec:detecting_exoplanets}

There are six main techniques for detecting exoplanets around main sequence stars, some with
advantages over others. These advantages depends strongly on the advancement of the instruments used
for the different techniques, but also on the physical properties of the exoplanetary system. A lot
of information about the exoplanets can be extracted when more than one of these techniques is used
and combined. The success of the different methods can be seen in \fref{fig:detectionTypes}. Here
all the detected planets (with measured masses) are shown divided {into which group was used for
detecting the exoplanets} into different detection methods. The most exoplanets are discovered with
the transit method and the radial velocity method, both which will be explained in more detail
below. Additionally, the figure also includes Mercury, Earth, and Jupiter (size of the symbols are
not to scales) that are here presented as a reference for the comparison with the known exoplanets.

\begin{figure}[htpb!]
    \centering
    \includegraphics[width=1.0\linewidth]{figures/exoplanetDetectionType.pdf}
    \caption{Confirmed exoplanets with the derived masses (relative to Jupiter) and the semi-major
             axis. The position of Mercury, Earth, and Jupiter are also shown, however the symbols
             are not to scale. The size of the symbols for the different detection methods are
             different in order to clearly see the methods which did not yet detect many exoplanets
             such as \emph{Imaging} or \emph{Astrometry}. Note that the masses discovered solely by
             the radial velocity (RV) method are the minimum masses. The data is from
             \url{http://exoplanet.eu}}
    \label{fig:detectionTypes}
\end{figure}

It is important to note, that different phenomena might mimic planetary signals, such as stellar
activity, or the contaminating light from a fainter stellar companion \citep[see
e.g.][]{Queloz2001,Oshagh2013,Oshagh2014}. However, they will not be described in this thesis. For
more details on the description of these techniques see e.g. \citet{Seager2010}.


\subsubsection{Transit method}
\label{sec:transitMethod}

The most successful method, if based on numbers of exoplanets detected, is the transit method. The
way to use this method is to measure the integrated light of a star is continuously monitored in
order to detect a transit. This kind of data is called a photometric time-series, and has wide
applications in stellar astrophysics. It is not enough to monitor a star to detect a transiting
exoplanet. It has to be aligned with the line of sight from the observer, thus the probability of
detecting exoplanets depends on the geometry of the system.

Transit photometry is a well-known method in astronomy, however only used recently for detecting
exoplanets. Before this, it has been used extensively for finding and characterising eclipsing
binary stars. The difference here is, that the exoplanet is extremely small and does not radiate (or
at least emit very little radiation compared with the host star). An example of an exoplanet
transiting a star can be seen in \fref{fig:transitMethod} for K2-29 b. The data is from a summer
school\footnote{Data can be found here: \url{https://github.com/iastro-pt/AzoresTE1}}, which was
prepared from \citet{Santerne2016}.

\begin{figure}[htpb!]
    \centering
    \includegraphics[width=1.0\linewidth]{figures/transitMethod.pdf}
    \caption{\emph{Upper plot}: The lightcurve of a star with an exoplanet transiting.
             \emph{Lower plot}: The phase folded curve of the above lightcurve using the period
             of \SI{3.259}{days}.}
    \label{fig:transitMethod}
\end{figure}

When an exoplanet transits its host star, the total brightness observed from the system will
decrease, as the exoplanet blocks some of the light from the star. This dimming in brightness will
be seen periodically as the exoplanet orbits its host star. The decrease in brightness as the planet
transit the star is directly related to the ratio between the stellar radius $R_\ast$ and the
planetary radius $R_p$:
\begin{align}
  k = \sqrt{\frac{R_p}{R_\ast}},  \label{eq:transit}
\end{align}
where $k$ is the depth of the transit compared to the normalised stellar brightness. This in turn
means that the transit method is most sensitive to exoplanets with a large radius, and often in
close orbits as these exoplanets are more likely to be aligned with the observer (us), and its host
star. In \fref{fig:transitAll} all detected exoplanets with the transit method is shown. In
particular the radius against the period (two parameters that will always be measured by this
method). Especially due to the \emph{Kepler} space mission, exoplanets with low radius have been
found, but it is evident as the period for a exoplanet increases, the detectability decreases, hence
the desert in the high-period/low-radius quadrant in the figure.

\begin{figure}[htpb!]
    \centering
    \includegraphics[width=1.0\linewidth]{figures/transitAll.pdf}
    \caption{Period and radius of all exoplanets discovered by the transit method from the
             \url{http://exoplanet.eu} database. The desert at high-period/low-radius is clearly
             evident here. This is where an Earth twin would be found around a Sun twin.}
    \label{fig:transitAll}
\end{figure}

This method allows to determine the radius of the planet as far as the radius of the star is known.
Furthermore, detailed analysis of the phase curve\footnote{A phase curve is here a light curve cut
at every period of the planet and plotted on top of itself. This is equivalent to a single transit
at a very high sampling rate} of an exoplanet can additionally reveal the surface temperature of the
exoplanet. The transit described above is also known as the primary transit. The secondary transit
is when the exoplanet is behind its host star as seen from Earth. This lead to a faint dimming of
the total brightness observed from the system. This dimming is due to the reflected light missing
from the exoplanet and its self-irradiation caused by its surface temperature. It is difficult to
observe secondary transits. This is mainly due to the low flux from the exoplanet compared to its
host.

While transit photometry can be done from Earth, the best results are from space with e.g.
\emph{Kepler} \citep{Borucki2010} and CoRoT \citep{Baglin2006}, while ground-based surveys have been
successful as well, e.g. WASP \citep{Pollacco2006}, and HAT-P \citep{Bakos2004} to mention just a
few.



\subsubsection{Radial velocity method}
\label{sec:rvmethod}

When an exoplanet orbits a star, they both orbit the centre of mass. Since the star is more massive
than the exoplanet, the centre of mass is much closer to the star. The radial velocity method is the
study of the motion of the host star around the centre of mass using the Doppler effect caused by an
orbiting exoplanet\footnote{As the transit method, this method has been known for decades in the
study of e.g. binary stars.}. This together with the transit method described above are by far the
most successful methods to detect and characterise exoplanets  with the current instruments. The
periodic signal created by the exoplanet on the host star depends on the mass ratio between the star
$M_\ast$ and the planet $M_p$:
\begin{align}
  K = \frac{\SI{28.4329}{m/s}}{\sqrt{1-e^2}} \frac{M_p\sin i}{\Mjup} \left( \frac{M_\ast+M_p}{M_\odot} \right)^{-2/3} \left(\frac{P}{\SI{1}{year}}\right)^{-1/3}  \label{eq:rv}
\end{align}
where $K$ is the semi-amplitude of the sinusoidal, $e$ is the eccentricity, $i$ is the inclination
of the orbital plane compared to the line of sight from Earth, $P$ is the orbital period, and
$\Mjup$ is the mass of Jupiter. Since $M_\ast \gg M_p$, the term $M_\ast+M_p\simeq M_\ast$ in order
to simplify the equation. The sinusoidal motion of the star can be seen in \fref{fig:rvmethod} where
both the time series and the phase curve is presented for the exoplanet, K2-29 b. The data used is
from the same source as the lightcurve above in \sref{sec:transitMethod}.

\begin{figure}[htpb!]
    \centering
    \includegraphics[width=1.0\linewidth]{figures/RVmethod.pdf}
    \caption{\emph{Upper panel}: RV time series of K2-29 b from the SOPHIE spectrograph.
             \emph{Lower panel}: Phase curve of the time series above, using the period of
             \SI{3.259}{days}. The fit is a simplistic sinusoidal for visual guidance.}
    \label{fig:rvmethod}
\end{figure}

In order to apply the radial velocity method for detecting exoplanets, it is necessary to collect a
time series of spectra in order to cover most part of the phase of the orbit. These spectra need to
be obtained with high spectral resolution in order to obtain a high precision measurement of the RV.
The RV is obtained when the observed spectrum is compared with a reference spectrum of the star at
$RV=\SI{0}{km/s}$. The Doppler shift needed to be applied to the observed spectrum to cause the
maximum overlap between the two spectra corresponds to the measured RV. The instrument needs to be
stable for the same reason \citep[see e.g.][]{Bouchy2001}. These spectra can be used for the
analysis and characterisation of the host star if combined in order to increase the signal-to-noise
(S/N). The spectra traditionally used for exoplanet detection have a high resolution, which
typically is more than \num{50000}. The spectral resolution is key, as the precision is strongly
dependent on this parameter.

The currently most precise radial velocity spectrographs are HARPS (ESO) \citep{HARPS} and HARPS-N
(La Palma) \citep{HARPSN}, able to achieve long term precision below \SI{1}{m/s} \citep[see
e.g.][]{Pepe2013}.

Very interestingly the RV measurements might be used to detect what is known as the
Rossiter–McLaughlin effect. This effect is a combination of two things. First, the rotation of the
host star slightly cause a red-shift of one side (the side rotating away from us), while the other
side of the rotating star is slightly blue-shifted. The net effect is zero. Second, if an exoplanet
transit its host, it might block some of the blue-shifted light, thus the net effect will suddenly
be ever so slightly red-shifted. While the exoplanet transit over the surface of the star the net
effect will change from slightly red-shifted to slightly blue-shifted. This effect is used to
determine the orbital configuration, i.e. the orbital inclination and whether the orbit is prograde
or retrograde. This effect have been used in several studies \citep[][to mention just a
few]{Winn2005,Triaud2010}. An illustration on how they RV phase curve might look like is illustrated
in \fref{fig:RMeffect}.

\begin{figure}[htpb!]
    \centering
    \includegraphics[width=1.0\linewidth]{figures/RMeffect.pdf}
    \caption{A model of the Rossiter-McLaughlin effect on a phase folded RV time series. This is a
             simple circular prograde orbit.}
    \label{fig:RMeffect}
\end{figure}

\subsubsection{Other techniques}

The following four techniques have all detected exoplanets, however, they are currently not widely
used and neither have the same level of success as the two methods described above. This is
something that will change in the future with new and up-coming instruments.

\paragraph{Direct imaging}

Direct imaging is probably the easiest method to understand, however it is quite difficult to
actually use this technique due to the immense difference in luminosity between a star and an
exoplanet. In its core, by carefully blocking the light of a star, it is possible to directly image
the exoplanets around it. However, it is extremely difficult to block the light of the host star and
find the emitted light of the exoplanet(s) in orbit. One has to deal with residual light from the
host star, which are orders of magnitudes brighter than the exoplanets. With SPHERE at VLT it is
expected than more exoplanets will be found with this technique \citep{Beuzit2008}. The first
exoplanet detection with SPHERE have already been made \citep[see
e.g.]{Vigan2016,Bonnefoy2016,Apai2016,Maire2016,Zurlo2016}.


\paragraph{Astrometry}

Astrometry is the measurement of the position of a star on the sky, i.e. its coordinates. The
coordinates for a star may change over time due to its relative motion compared to our Solar System.
Additionally, an exoplanet might cause minute changes in the coordinates due to the orbit of the
star and exoplanet around the centre of mass. Using astrometry to detect exoplanets is very similar
to the RV method described above in \sref{sec:rvmethod}. The Gaia mission \citep{GAIA} will provide
many exoplanets detected with astrometry. \citet{Perryman2014} estimates that $21\,000\pm6\,000$
high-mass ($1-15\Mjup$) long-period planets will be discovered to a distance of \SI{500}{pc} with
the Gaia mission for the nominal \SI{5}{year} mission.

This technique is more sensitive to massive, long period exoplanets as they cause a larger motion
compared to lighter, short period companions. See e.g. Figure 1 by \citet{Perryman2014} for the
astrometric signatures produced by a range of exoplanets at different period.


\paragraph{Transit timing variation}

This technique of detecting exoplanets is a highly indirect method of detecting exoplanets. Here a
transiting exoplanet has to be detected first as explained in \sref{sec:transitMethod}. Then
variations in the occurrence of mid-transit can be detected if a second non-transiting exoplanet
interact with the primary transiting exoplanet (known as planet-planet interaction). This
interaction will periodically cause the mid-transit to happen ahead/behind of the time if only one
exoplanet would be present \citep{Agol2005,Holman2005}.

A careful analysis of the transit timing variations (TTV) can give the mass of the transiting
exoplanet. Most of these exoplanets pairs which shows TTV are in an orbital resonance. This
technique as well, is more sensitive to massive exoplanets as they will induce a higher signal.
However, it is also sensitive to exoplanets with comparable semi-major-amplitudes.


\paragraph{Microlensing}

This technique is very exotic and not widely used, however since a few exoplanets have been
discovered by this technique it deserves to be mentioned. The core theory in this technique is the
well-known General Relativity by \citet{Einstein1916}. With this method an observer looks at a
distant star as a star between the observer and the distant star passes in between the line of
sight. The intermediate star will act as a lens and increase the magnitude of the distant star. This
increase of magnitude will reach its maximum as the two stars are most aligned as seen from Earth.

To use this method for detecting exoplanet, there will have to be an exoplanet orbiting intermediate
star. The exoplanet act as a microlens, momentarily make a secondary increase in magnitude. The
amount of increase in magnitude is related to the mass of the exoplanet and its configuration in the
orbit around its host star. The higher the mass, the higher the effect, however it is possible to
detect quite low-mass exoplanets with this technique.

While this exotic technique is interesting and has detected exoplanets, it is not as useful as the
RV and transit detection methods in individual studies of planets since it only occurs once. The
stars observed with this technique are often faint, thus making follow-up RV detection very
difficult if not impossible with the current instruments.


\subsection{Towards an Earth twin}

The above mentioned techniques will be used to find an Earth twin. Especially will the two first
techniques (transit and RV method) be the ones finding the smallest exoplanets as a wide range of
instruments are being developed dedicated for this. Since the detection of the first exoplanet
around a solar-type star by \citet{Mayor1995}, the community has been able to detect lower mass
exoplanets as seen in \fref{fig:exoplanetMass} and the regime of Earth-mass exoplanets have been
reached very recently.

\begin{figure}[htpb!]
    \centering
    \includegraphics[width=0.8\linewidth]{figures/exoplanetMass.pdf}
    \caption{The mass of exoplanet since the detection of the first exoplanet until now. The
             horizontal lines indicate the mass of Jupiter (upper), Neptune (middle), and Earth
             (lower).}
    \label{fig:exoplanetMass}
\end{figure}

While the first many discoveries of exoplanets were the (at the time) exotic and strange hot
Jupiter-like exoplanets in close orbits, the time have come to detect exoplanets with both lower
mass and wider orbits. It is crucial to have high precision instruments and long surveys to detect
these exoplanets. Missions as \emph{Kepler} and CoRoT have been excellent for this, since they have
focused on few fields of the sky for a long time.

The first place to naively look for a Earth's twin would be in a system similar to our own Solar
system. That means around a G dwarf like our own Sun. However, due to the high surface temperature
of these stars\footnote{Very few effort have been made to look for exoplanets around OB stars as
these are both few in numbers and hostile environments for exoplanets.}, the habitable zone will be
far from the host star \citep[see e.g.][]{Kasting1993}. The habitable zone is the distance interval
from a star where liquid water can be found. This zone is not only dependent on stellar parameters,
but also on the atmospheres of exoplanets since a exoplanet with a dense atmosphere can keep the
heat, thus be further away from its host star and still have liquid water, compared to an exoplanet
with a lighter atmosphere. Indeed, to detect a copy of our own Sun-Earth system we would need to
detect the minute signature of an exoplanet in a \SI{1}{year} orbit around its host star (a solar
twin). For the RV method, this minute signature will be \SI{10}{cm/s}, while the current level of
precision is around \SI{1}{m/s}. For photometry a precision of \SI{100}{ppm} is required to detect
an Earth sized planet around a solar-type star \citep{Borucki2017}. These signatures can be measured
with the next generation of ultra stable high resolution spectrographs such as ESPRESSO
\citep{ESPRESSO} and future photometric space missions such as PLATO \citep{PLATO}. If detected with
the transit method, more than one transit is needed, hence this will take at least two years, and
probably even longer. Moreover, this is all assuming that the geometry of the system aligns in such
as way that a transit can even be seen from Earth. The endeavour to get the follow-up RV afterwards
will also be extremely challenging with today's technology, and only the next generation of
instruments will be able to detect these signals. In conclusion, even with perfect instruments, it
will take years to detect an Earth twin around a Sun-twin with the RV and transit methods.

Therefore, it is not a surprise that an effort has been towards detecting Earth-like planets around
the lightest stars. These stars (M stars) are also cooler, hence the habitable zone will be closer
to its host compared to the more massive and hotter stars \citep{Kasting1997}, and ultimately the
period for habitable exoplanets will be shorter. The nature has been kind, since it seems that the M
stars are prone to form rocky planets rather than giant gaseous planets
\citep{Bonfils2013,Delfosse2013}. The shorter period means that the surveys can be shorter for these
exoplanets. Moreover, since the host stars are smaller the signal from a transit will be easier to
detect (see \eref{eq:transit}). Similarly will the RV signal be larger for an Earth-like planet in
the habitable zone around an M star compared to a similar exoplanet around a G star\footnote{ For a
solar twin, with an Earth twin at \SI{1}{AU} the semi-amplitude will be close to \SI{10}{cm/s} while
it would be roughly \SI{50}{cm/s} for an Earth twin around a 0.3$M_\odot$ M dwarf at \SI{0.1}{AU}.}.
Both due to the lower period and due to the lower mass of the host star (see \eref{eq:rv}).

While M stars seems to be the place to look for the Earth's twin, there are still some challenges to
tackle. Foremost is the detailed characterisation of the host star, which are particular troublesome
for these stars. This is something that will be focused on in \sref{sec:planet_host_stars}.

\subsubsection{Detecting biosignatures on an exoplanet}

The best hope to indirectly detect life is by finding biosignatures \citep[see
e.g.]{Kasting2002,Snellen2013} in the exoplanet's atmosphere. Transmission spectroscopy and
lightcurves in different pass bands are the techniques to study the atmosphere of exoplanets. For
transmission spectroscopy a spectrum is obtained of the star during transit. This spectrum will
thereby contain signals of the atmosphere of the transiting exoplanet. Later, a spectrum of the star
can be obtained, e.g. during the secondary transit. The difference in these two spectra will reveal
the exoplanetary spectrum. This was e.g. done by \citet{Charbonneau2002}.

An interesting test to look for biosignatures from the Earthshine on the Moon was performed by
\citet{Arnold2002} where they clearly see the blue colour of the Earth's atmosphere due to Rayleigh
scattering. This was done by obtaining a Moon spectrum and an Earthshine spectrum using the
\SI{80}{cm} telescope at OHP. They also observed signatures for oxygen, ozone, and water vapours;
all important biosignatures in a planetary atmosphere supporting life.

These signatures, especially oxygen, are from microorganisms through photosynthesis
\citep[see e.g.][]{Kasting2002}. However, there might be conditions outside the habitable zone which
might sustain life. Here on Earth, extremophiles such as the water bears are known to thrive in
extreme places such as boiling water, acid, ice, etc. This might eventually lead to a new window of
opportunity in the search of extraterrestrial life \citep{Cavicchioli2002}.



\section{Planet host stars}
\label{sec:planet_host_stars}

With the present diversity of exoplanets it becomes increasingly important to get an accurate and
precise characterisation of the exoplanets in order to study them in samples and on an individual
level. An accurate and precise characterisation can give us an idea whether the planet is rocky,
composed of water, gaseous, or some other more exotic combination, by comparing fixed density
profiles (based on different compositions) to the planetary mass and radius
\citep[see][e.g.]{Dorn2015}. As mentioned above, it is crucial to characterise the host stars in
order to characterise the exoplanets. The host stars discovered so far are mainly FGK dwarf stars,
although some are more evolved. This can be seen in \fref{fig:hostDistribution} where the effective
temperature of all host stars are shown against the luminosity. The colour represents the
temperature, while the size of the symbols is a measure of the radius, or with other words, the
evolutionary state. The smallest points are the dwarf stars. The luminosity was derived combining
the following equations:
\begin{align*}
  g &= \frac{GM}{R^2} \\
  L &= 4\pi R^2 \sigma T_\mathrm{eff}^4,
\end{align*}
into
\begin{align}
  \frac{L}{L_\odot} = \frac{M}{M_\odot} \left(\frac{T_\mathrm{eff}}{\SI{5777}{K}}\right)^4 10^{4.44-\log g},
\end{align}
where $\sigma$ is Boltzmann's constant, $G$ is Newton's gravitational constant, and $\log g$ is the
surface gravity. The data for \fref{fig:hostDistribution} is obtained from SWEET-Cat\footnote{
\url{https://www.astro.up.pt/resources/sweet-cat/}}.

\begin{figure}[htpb!]
    \centering
    \includegraphics[width=1.0\linewidth]{figures/hostDistribution.pdf}
    \caption{The effective temperature of all host stars are shown against the luminosity. The
             colour represents the temperature, while the size of the symbols is a measure of the
             radius, or with other words, the evolutionary state. The smallest points are the dwarf
             stars. The green star in the plot show the location of the Sun.}
    \label{fig:hostDistribution}
\end{figure}

To make an in-depth characterisation of stars it is common to use several different methods to gain
knowledge about different aspects of a star. If the effective temperature is needed, the reliable
determination comes from the analysis of a high resolution and high signal-to-noise (S/N) spectrum
or the infrared flux method (see \sref{sec:irfm} for more details). Spectroscopy is also used to
identify chemical abundances of the photosphere of the star, while a method like
asteroseismology\footnote{Asteroseismology is the study of stellar oscillations. See
\sref{sec:asteroseismology} for details on this method.} is used to determine the mass and radius of
a star with higher precision; two parameters crucial to characterise the orbiting exoplanet. More of
these methods are described in greater detail in \cref{cha:method}.

In the example given above, the effective temperature is needed before asteroseismology can be used
to determine the mass and radius. In order to perform a successful detailed asteroseismic analysis a
long time series is needed like the ones obtained from CoRoT and \emph{Kepler} which are months and
years long time series. The light curves provided by these dedicated missions are used both for the
detection of exoplanets and stellar oscillation studies. Likewise will the spectra obtained from the
RV method to detect/confirm an exoplanet be used for the stellar characterising afterwards by
combining them to increase the S/N. This combined high S/N spectrum is ideal for the spectral
analysis of stars.

The synergy between a spectroscopic analysis and asteroseismology have proven very successful
\citep[see e.g.][]{Huber2013} for characterising an exoplanet system (host star and exoplanet).
However, it does have its limitations. It can be difficult to detect solar-like oscillations as the
stars get colder than the Sun. The community has yet to detect any solar-like oscillations in M
dwarf stars \citep{Rodriguez2016,Berdinas2017}. Many of the detected exoplanets are from the
\emph{Kepler} mission, where many of the host stars are very faint. While it is not impossible to
make high resolution and high S/N spectroscopic observations, it is extremely time consuming.
Therefore brighter targets are often prioritised, unless there is an exceptional case.

With the search for an Earth twin around the small cool M dwarf stars, it is important to develop
reliable methods for the analysis of these stars. The methods to analysis typical planet hosts (FGKM
stars) currently works best for the FGK dwarf stars compared to M dwarf stars. To illustrate this
the median errors on the atmospheric parameters can be seen in \tref{tab:standardErrors}. The
parameters are from SWEET-Cat, and for the FGK stars only parameters analysed by the Porto group are
presented. It is important to note that there are 138 M stars while there are 538 FGK stars (after
only using stars analysed here). More details on SWEET-Cat will follow in \cref{cha:SWEET-Cat}.
\begin{table*}[htb!]
    \caption{The median precision errors on the stellar atmospheric parameters for FGK planet
             hosts and M planet hosts as taken from SWEET-Cat.}
    \label{tab:standardErrors}
    \centering
    \begin{tabular}{lllllll}
      \hline\hline
        Atmospheric parameter      & FGK    & M    \\
      \hline
        $T_\mathrm{eff} [\si{K}]$  & 44     & 60   \\
        $\log g [\si{cgs}]$        & 0.090  & 0.04 \\
        $[\ion{Fe}/\ion{H}]$       & 0.035  & 0.10 \\
      \hline
    \end{tabular}
\end{table*}

The detailed characteristics of M dwarfs are currently a problem, however, with the advance of
NIR spectrographs this is slowly changing. It is general believed that a NIR analysis is needed to
characterise M stars. The reason is simple that these stars are so intrinsic faint, that it is
important to collect as much flux as possible. This happens in the NIR. Since M stars are
intrinsically faint in the visible, it is advantageous to use NIR spectrographs where M stars emit
most of their light. Moreover, for spectroscopic studies, the optical spectrum of these stars are
severely contaminated by molecular absorption lines, which depress the continuum. It is crucial to
get the continuum placement correct during spectroscopic studies. In the NIR the continuum
depression is less severe, however still challenging. This can clearly be seen in
\fref{fig:opticalVSnir} where the optical and NIR part of the spectrum for HD 79210, a K7 dwarf
star\footnote{Note that there are very little difference between a K7V and M0V. For the latter case,
the situation raised in \fref{fig:opticalVSnir} will be more clear.}, is plotted. The spectra were
obtained by CARMENES simultaneously and have therefore the same exposure time. During a
spectroscopic analysis, which will be the applied method in this thesis, it is important to observe
isolated absorption lines. The more contaminated and blended a absorption line is, will result in
more uncertain results at the end of the analysis. This will be described in detail in
\sref{sec:parameters}. In \fref{fig:opticalVSnir} it is clear that the NIR spectrum is far less
contaminated by absorption lines.

\begin{figure}[htpb!]
    \centering
    \includegraphics[width=1.0\linewidth]{figures/opticalVSnir.pdf}
    \caption{Comparison between an optical and NIR part of the spectrum of HD 79210 obtained by
             the CARMENES spectrograph. This clearly illustrate why NIR spectra are preferred over
             optical spectra for cool stars, where absorption lines are less blended. HD 79210 is a
             K7 dwarf star.}
    \label{fig:opticalVSnir}
\end{figure}

The main goal of this thesis is to work towards a consistent derivation of stellar atmospheric
parameters for M stars in NIR. Before tackling the M stars, it is important to have a method that
work well for solar-like stars (FGK), which are better known during countless studies.

\subsection{Star-planet correlations}
\label{sec:starPlanetCorrelation}

While the stellar parameters are used to determine the planetary parameters, they can also be used
to reveal correlations between planets and their host stars. This might give insight in the
formation and evolution of the planets. Indeed, several important correlations have been found
already.

\paragraph{Giant planet and metallicity correlation}

Since the first discoveries of exoplanets, it became evident that giant planets were systematically
orbiting more metal-rich stars compared to stars with no planets. This correlation have been
confirmed by several studies \citep{Gonzalez1997,Santos2004,Fischer2005,Sousa2008a,Mortier2013b} and
can be seen in \fref{fig:fehCorrelation} using the data from \citet{Sousa2008a}. This correlation in
turn establish that core accretion is likely the main formation mechanism among giant exoplanets
\citep{Pollack1996,Ida2004,Mordasini2012} and not disc instability \citep{Boss2002}.

\begin{figure}[htpb!]
    \centering
    \includegraphics[width=1.0\linewidth]{figures/fehCorrelation.pdf}
    \caption{Metallicity correlation for giant planets, comparing planet hosts and non host stars
             from the sample of \citet{Sousa2008a}. It is clear that planet host stars have higher
             metallicities than non hosting stars. The histograms are normalised.}
    \label{fig:fehCorrelation}
\end{figure}

It is important to note that recent studies find this correlation for Neptune-like and super-Earth
planets as well \citep{Adibekyan2012a,Wang2015,Zhu2016} even if a debate exists
\citep{Sousa2011,Buchhave2012}.

It has been shown that metal-poor stars harbouring gas planets are enhanced in alpha
elements\footnote{These are elements formed by fusion of a helium core; C, N, O, Ne, Mg, Si, S, Ar,
Ca, and Ti.} \citep[see e.g.][]{Adibekyan2012a}. This means that other elements than iron also have
a role to play in planet formation in iron-poor environments.


\paragraph{Lithium and the presence of planets}

It has been shown by \citet{Israelian2004,Delgado2014,Figueira2014a,Gonzalez2015,Takeda2005} that
stars hosting planets are significantly more \ion{Li}{} depleted compared to stars without planets.
This correlation seems to be physically related with the occurrence of planets and not e.g. age,
mass, or metallicity of the stars \citep{Sousa2010}. Other studies does not find any correlation
between \ion{Li}{} depletion and stars with planets \citep{Baumann2010,Ramirez2012}.




\section{Applications from knowing the stars}
\label{sec:stars_application}

While ``know the star, to know the exoplanet'' is the main motivation behind this thesis, it is
obviously not the only application behind detailed stellar characterisation. Working towards a
better understanding of especially M stars, which consist of to 70\% of all the stars in the Milky
Way \citep{Bochanski2010}, will also open a new window into the study of different galactic
components and galactic chemical evolution. Such a study was done by e.g.
\citet{Adibekyan2012,Delgado2017} where spectra of 1111 FGK dwarf stars from the HARPS GTO sample
was used to derive chemical abundances of 12 refractory elements. It is possible to separate
different galactic populations (thin disk, thick disk, and the halo) by studying the chemical
abundances of stars as was shown in \citet{Adibekyan2012}.

In order to do a detailed chemical analysis\footnote{See also references within
\citet{Adibekyan2012} for other similar studies.}, it is crucial to have reliable and homogeneously
derived stellar atmospheric parameters. If the parameters are homogeneously derived, i.e. with the
same method, then errors when comparing populations are minimised and in case of known offsets it is
easy to correct for them (this is the case for the method used throughout this thesis, where the
surface gravity will be corrected based on another method).


\section{This thesis}
\label{sec:this_thesis}

This thesis will be focused on deriving stellar atmospheric parameters for FGK stars, making the
bridge towards M stars. This task will be accomplished utilising high resolution and high S/N NIR
spectra; a methodology using a wavelength domain still in its infant stage. The theory of stellar
atmospheres in a nutshell is described in \cref{cha:theory}, setting up all the tools to derived
them as described in \cref{cha:method}. In \cref{cha:method} the description of other useful and
commonly used methods for deriving parameters are also presented. Thereafter the knowledge will all
be used in \cref{cha:results}. First by obtaining a NIR line list containing \ion{Fe}{I} and
\ion{Fe}{II} lines, then by the derivation of parameters for HD 20010 (F sub-giant). Before deriving
parameters for two K giants, the NIR line list will be revisited.

After focusing on the NIR spectra, the optical counterpart of the method described below will be
used to derive parameters for 50 planet-host stars for an online catalogue of homogeneously derived
parameters (SWEET-Cat) in \cref{cha:SWEET-Cat}.

Last in \cref{cha:future} the future of the work established here will be discussed along with the
results already obtained. This will round of the thesis.

After the last chapter there will be a few appendices with large tables that would otherwise
distract the reader from the main points. One appendix, \cref{cha:fasmaML}, will be on derivation of
stellar atmospheric parameters using machine learning, which was a side project during the thesis.
This appendices are followed by the bibliography for this thesis.
