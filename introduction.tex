%!TEX root = thesis.tex
\chapter{Introduction}
\label{cha:introduction}

Ever since the dawn of time, the humankind have looked at the stars and wondered if we are alone in
this Universe. To answer this question, one must look toward the field of extrasolar planets
(exoplanets). This is a rapidly growing field in astronomy and science in general. Since the first
confirmed discovery of an exoplanet around a millisecond pulsar in 1992 by \citet{Wolszczan1992} and
three years later, the more interesting exoplanet 51 Peg b discovered around a solar-type star by
\citet{Mayor1995}, more than 3600 exoplanets have been discovered at the time of writing, July
2017\footnote{\url{http://exoplanet.eu/}}.

With the discoveries of exoplanets, the main focus is now mainly on finding the twin of Earth, that
is a planet that can harbour life as we know it. However, it is not enough to simply discover small
rocky exoplanets. Accurate and precise determination of the stellar parameters are crucial as the
planetary parameters (radius, mass, bulk density, etc.) are directly derived from their host's
parameters.

In this chapter there will be a general introduction to exoplanets, detection methods, and
characterisation (\sref{sec:exoplanets}). Then a throughout introduction on the exoplanet host
stars (\sref{sec:planet_host_stars}), which is the main focus on this thesis. While learning about
host stars, and stars in general, the results have wide-spread applications, where some will briefly
be discussed in the end of this chapter (\sref{sec:stars_application}) before an introduction on
what this thesis will consists of (\sref{sec:this_thesis}).



\section{Exoplanets}
\label{sec:exoplanets}



\section{Planet host stars}
\label{sec:planet_host_stars}

With the present diversity of exoplanets it becomes increasingly important to get an accurate and
precise characterisation of the planets in order to study them in samples and on an individual
level. An accurate and precise characterisation can give us an idea whether the planet is rocky,
composed of water or gaseous.



\section{Applications from knowing the stars}
\label{sec:stars_application}





\section{This thesis}
\label{sec:this_thesis}
